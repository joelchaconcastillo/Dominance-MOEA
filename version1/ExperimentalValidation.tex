In this section the experimental validation is carried out, showing that controlling the diversity in the variable space is a way to improve further some of the results obtained by the state-of-art-MOEAs at least in long-term executions.
%
The analysis carried out in this section is carefully designed to get a clear understanding of the balance induced between exploration and exploitation in the \VSDMOEA{}.
%
First, several technical specifications of the implemented agorithms, and test problems are described.
%
Thereafter, a core comparison between \VSDMOEA{} and the state-of-the-art is presented.
%
Additionally, to have a broad perception of the \VSDMOEA{} three extra experiments are driven.
%
Such analyses are aimed to test the scalability in the decision variable space, the diversity parameter required by \VSDMOEA{}, and several stopping criterion settings.
%
Particularly, the latter is probed with several terms-execution, i.e. from short-term to long-term executions.


%
%Firstly, several technical specifications taken into account in our comparison outline are explained.
%
%Thereafter, to have a broad perception of the \VSDMOEA{} some experiments are driven.
%
%Between them an analyzes which is designed to test the scalability in the decision variable space of each \MOEA{}.
%
%This analyzes is narrowed through some specific and well known problems.
%
%In the same line, with the intention to have a better understanding of the critical parameter that induces the initial amount of diversity is taken into account.
%
%This is carried out through several settings and with all the test-problems.
%
%Finally, since that the mechanism imposed in our proposal, which avoid the premature convergence, highly depends on the elapsed time, we reported the benefits of it considering short-term and long-term executions.
%
%The latter experiments shows that the \VSDMOEA{} has a decent performance in midterm executions.
%

This work is validated through some of the most popular bechmarks in the multi-objective field.
%
Such problems are the WFG~\cite{Joel:WFG}, DTLZ~\cite{Joel:DTLZ}, and UF~\cite{Joel:CEC2009}.
%
Furthermore, the experimental validation includes three well-known state-of-the-art MOEAs and the \VSDMOEA{}.
%
The MOEAs taken into account are the \NSGAII{}~\cite{Joel:jMetal} , the \MOEAD{}~\cite{MOEADCode}, and the \RMOEA{}~\cite{R2EMOACode}, that can be classified as based-dominance, based-decomposition, and based-indicators respectively.
%
Particularly, the \MOEAD{} implementation belongs to the first place in the ``Congress on Evolutionary Computation 2009'' (CEC) \cite{zhang2009performance}.
%
Given that all the considered algorithms are stochastic, each execution was repeated $35$ times with different seeds.
%
The common configuration in all the executions was the following: the stopping criterion was set to $250,000$ generations, the population size was fixed to $100$, the WFG test problems were configured with two and three objectives, which are set to $24$ parameters, where $20$ of them are distance parameters, and $4$ are position parameters.
%
Additionally, in the DTLZ test instances, the number of decision variables is set to $n=M+r-1$, where $r=\{5, 10, 20\}$ for DTLZ1, DTLZ2 to DTLZ6 and DTLZ7 respectively, as is suggested by the authors \cite{Joel:DTLZ}.  
% 
The UF benchmark is conformed by seven problems with two objectives (UF1-7) and three problems with three objectives (UF8-10), such problems are set with $30$ decision variables.
%
The operators employed on all the MOEAs are the Simulatied Binary Crossover (SBX), and the polynomial mutation~\cite{Joel:SBX1994, Joel:Mutation}.
%
In the crossover the probability and distribution index are set to $0.9$ and $2$ respectively.
%
Similarly, the mutation probability and distribution index are set to $1/n$ and $50$ respectively.
%
%
%The crossover and mutation probabilities are set to $0.9$ and $1/n$ respectively.
%
%In general, the crossover and mutation operators are the Simulated Binary Crossover (SBX), and polynomial~\cite{Joel:SBX1994, Joel:Mutation}, which probabilities are set to $0.9$ and $1/n$ respectively.
%
%Also, the crossover and mutation distribution indexes were assigned to $2$ and $50$ respectively.
%
In addition, the extra-parameterization of each algorithm is shown in the Table~\ref{tab:Parametrization}.


%In order, to validate the proposed MOEA, in this work are considered some of the most popular benchmark in the multi-objective field.
%
%Particularly, the WFG~\cite{Joel:WFG}, DTLZ~\cite{Joel:DTLZ}, and UF~\cite{Joel:CEC2009} test problems have been used for our purpose. 
%
%Through the literature several crossover operators have been proposed in MOEAs~\cite{Joel:ParentMeanCentricSelfAdaptation},  
%a popular operator is the Simulated Binary Crossover (SBX).4\cite{Joel:SBX1994}%, Joel:TAXONOMY_CROSSOVER, Joel:Kalyanmoy}
%Additionally, our experimental validation includes the \VSDMOEA{}, as well as three well-known state-of-the-art algorithms.
%
%In addition, out experimental validation includes the \VSDMOEA{}, as well three well-known state-of-the-art MOEAs.
%
%There are the \NSGAII{}~\cite{Joel:jMetal} , the \MOEAD{}~\cite{MOEADCode}, and the \RMOEA{}~\cite{R2EMOACode}, that can be classified as based-dominance, based-decomposition, and based-indicators respectively.
%
%The \MOEAD{} implementation taken into account belongs to the first place in the ``Congress on Evolutionary Computation 2009'' (CEC) \cite{zhang2009performance}.
%
%

%
% Please add the following required packages to your document preamble:
% \usepackage{multirow}
\begin{table}[t]
\centering
\caption{Parameterization of each MOEA considered}
\label{tab:Parametrization}
\begin{tabular}{c|c}
\hline
\textbf{Algorithm} & \textbf{Configuration} \\ \hline
\multirow{3}{*}{\textbf{MOEA/D}} &Max. updates by sub-problem ($\eta_r$) = 2, \\
 & tour selection = 10,   neighbor size = 10, \\
 & period utility updating = 30 generations \\ 
 & probability local selection ($\delta$) = 0.9,\\ \hline
\textbf{VSD-MOEA} & $D_I=0.4$ \\ \hline
\textbf{R2-EMOA} & $\rho=1$, offspring by iteration = $1$ \\ \hline
\end{tabular}
\end{table}


Despite the fact that the \MOEAD{}, and \RMOEA{} can be employed with the same utility function --in this case the Tchebycheff function-- each one of them is designed through a notorious dissimilar paradigm.
%
Thus, the weight vectors taken into consideration for each one of them were different.
%, i.e. the confias it was originally proposed by the authors. 
%
The main reason of this, is that the \RMOEA{} can be configured with a different population size than the number of weight vectors without been significantly affected.
%
Particularly, the \RMOEA{} employs $501$ and $496$ weight vectors for two and three objectives respectively.
%
Contrarily, in the \MOEAD{} each weight vector is identified as a sub-problem, therefore the population should correspond to the same number of weight vectors.
%
In addition, the weight vectors used in the \MOEAD{} should be uniformly scattered on the unit-simplex, however it can be a drawback since that the number of vectors required for this task increases non-linearly according the number of objectives.
%
Therefore, in this version is applied the method proposed in \cite{Joel:MOEAD_Uniform_Design, Joel:Kuhn_Munkres} where the uniform design (UD) \cite{Joel:Uniform_Design} and good lattice point method (GLP) are combined.
%
In this way, the number of weight vectors that is required by this \MOEA{} is not affected by the number of objectives.


Mainly, the experimental analyzes is carried out considering the hypervolume indicator (\HV{}).
%
The \HV{} metric measures the size of the objective space dominated by the approximated solutions given a reference point, so the solutions dominated by the reference point are not considered.
%
Particularly, the reference point is chosen to be a vector which values are sightly larger (ten percent) than the nadir point as is suggested in \cite{ishibuchi2017reference}.
%
Similarly that in \cite{li2015evolutionary}, and to have a fair comparison the normalized \HV{} is taken into account.
%
Specifically, the \HV{} reported is computed as the ratio between the \HV{} reached by a set of solutions and the \HV{} of the optimal Pareto Front.
%
In this way, the more approximated to the unity this metric is, the more converged are the solutions to the Pareto Front.
%



%%\begin{table}[t]
%%\centering
%%\caption{References points for the HV indicator}
%%\label{tab:ReferencePoints}
%%\begin{tabular}{cc}
%%\hline
%%\textbf{Instances} & \textbf{Reference Point} \\ \hline
%%WFG1-WFG9 & $[2.1, ...,2m+0.1]$ \\
%%DTLZ 1, 2, 4 & $[1.1, ..., 1.1]$ \\
%%DTLZ 3, 5, 6 & $[3, ..., 3]$ \\
%%DTLZ7 & $[1.1, ..., 1.1, 2m]$ \\
%%UF 1-10 & $[2, ..., 2]$ \\ \hline
%%\end{tabular}
%%\end{table}
%
In order to statistically compare the \HV{} results, a similar guideline than the proposed in~\cite{Joel:StatisticalTest} was used. 
%
First a Shapiro-Wilk test was performed to check whatever or not the values of the results followed a Gaussian distribution. 
%
If, so, the Levene test was used to check for the homogeneity of the variances. 
%
If samples had equal variance, an ANOVA test was done; if not, a Welch test was performed. 
%
For non-Gaussian distributions, the non-parametric Kruskal-Wallis test was used to test whether samples are drawn from the same distribution. 
%
An algorithm $X$ is said to win algorithm $Y$ when the differences between them are statistically significant, if the mean and median obtained by $X$ are higher than the mean and median achieved by $Y$.
%

In the Tables \ref{tab:StatisticsHV_2obj}, \ref{tab:StatisticsHV_3obj} are shown the normalized hypervolume with two an three objectives respectively.
%
From this empirical results, it is clear that the \VSDMOEA{} achieves the highest general mean \HV{} with both two and three objectives (last row).
%
Even more, the standard deviation is the lowest in almost all the problems, therefore this \MOEA{} shows to be stable, consequently it is able to attain similar results through several runs.
%
The best general mean considering two objectives is achieved by the \VSDMOEA{} with $0.955$.
%
Also, the second general mean is obtained by the \NSGAII{} with $0.886$.
%
It is important to remark that the general mean can be unsteady to atypical measurements, thus very low values could affect highly the general mean.
%
However, all the state-of-the-art-\MOEAS{} achieved a general mean of $0.88$ considering two objectives, whilst the \VSDMOEA{} obtained $0.95$.
%
Considering three objectives the performance of the \NSGAII{} is seriously affected, this might occurs since that the density estimator employed in the \NSGAII{} highly depends on the dominance-relation, thus in some problems (e.g. multi-frontal problems UF10) the solutions do not converge adequately to the Pareto front.
%
In the same line, the \VSDMOEA{} achieved the best \HV{} values in almost all the problems, in fact such values that are higher than $0.9$ are close enough to the Pareto Front.
%
The second best \MOEA{} based in the general mean is the \RMOEA{} with $0.855$, despite that it adopts the same utility function that the \MOEAD{}, the latter is lightly lower with $0.835$, this might occurs given that the \MOEAD{} has several parameters to be tuned, and the selected configuration could be insufficient to long-term executions.
%


In the Tables \ref{tab:Tests_HV_2obj} and \ref{tab:Tests_HV_3obj} are shown the statistical tests with two and three objectives respectively.
%
In the column tagged ``Diff'' is computed the difference between the mean of each algorithm and the best mean achieved.
%
Taking into account two objectives the \VSDMOEA{} attains the best score of $52$ wins, the second best score is attained by the \RMOEA{} with $34$ wins.
%
In spite that the \NSGAII{} achieves a better general mean, it wins less times that the remaining \MOEAS{}.
%
Meaning that the \NSGAII{} obtains values close enough than the best \MOEA{}, this can be viewed in the last row where is computed the total sum of all the problems.
%
Generally speaking, the best results are obtained by the \VSDMOEA{}, since that the total sum of the ``Diff'' column is $0.061$, thus when this algorithm does not achieved the best results, it obtained near solutions to the best results.
%
Furthermore, the algorithms \RMOEA{} and \MOEAD{} achieved a similar total ``Diff'' values.
%
Particularly, the worst ``Diff'' value achieved by the \VSDMOEA{} is with the problem WFG6.
%
This problem is uni-modal and non-separable, and this might occurs since that the initial factor distance is very high.
%
In fact through other experimental analyzes this problem was correctly solved with $D_I=0.1$ whose mean achieved  was of $0.917$, and $0.868$ for two and three objectives respectively.
%
In addition, a similar behavior can be seen with three objectives (Table \ref{tab:Tests_HV_3obj}), where the \VSDMOEA{} improves significantly in comparison to the state-of-the-art-MOEAs.
%
Particularly, the most complicated problems are better solved by the \VSDMOEA{} as are UF3, UF4, UF5, and UF6 composed by two objectives, and UF9, UF10 with three objectives.
%
The UF5 is considered as one of the most difficult problems since that the optimal Pareto front is conformed by $21$ points, also it has several sub-optimal regions where the solutions could suffer stagnation. 
%
Nevertheless, since that it is a disconnected Pareto front, the \MOEAS{} that considers weight vectors faces several difficulties (e.g. knee regions), consequently they have more chances to stagnate.
%
In fact in the latter problem the \NSGAII{} has a lower ``Diff'' value ($0.048$) than the \MOEAD{} and \RMOEA{} ($0.205$ and $0.122$).
%
Diversely, the UF10 is a multi-frontal problem, this means that there exists different sub-optimal non-dominated fronts that correspond to different locally optimal values \cite{huband2006review}, this characteristic increases the difficulties of the problem as the number of objectives increases.
%
However, the latter problem has notably converged better in \VSDMOEA{} ($0.627$) than in the remaining \MOEAS{} (second best $0.413$).
%


\subsection{Decision Variable Scalability Experiments}

 
The scalability of each MOEA is also evaluated respect with the number of decision variables \cite{Joel:ScalabilityStudy}. 
%
The figures \ref{fig:variable-decision-scalability-2obj} and \ref{fig:variable-decision-scalability-3obj} show the mean of \HV{} attained with $50$, $100$, $250$, $500$, and $1000$ variables respectively.
%
The scalability study was taken into consideration with some problems conformed by easy and difficult characteristics.
%
Particularly, the problems considered in this analyzes were the DTLZ4, UF5 which are conformed by two objectives, and DTLZ4, UF10 with three objectives.
%
In addition, each selected problem was repeated $35$ times, thus the mean of all the \HV{} values are reported.
%
The \VSDMOEA{} attained the best results in two objectives with both problems, however it is remarkable to notice that the \HV{} value reported by the \NSGAII{} improves as the number of decision variables is increased.
%
Specifically, the DTLZ4 is uni-modal, and its Pareto shape is concave, also it has a polynomial bias.
%
This interesting behavior is also showed by the \RMOEA{}, and can be explained as follows.
%
The probability of stagnation in certain sub-optimal regions can be avoided increasing the decision variables space.
%
It highly depends in the operators that are taken into account (in this case the SBX and polynomial mutation), thus in some circumstances the more bigger the variable decision space is, the better quality solutions are reached.
%
This can be lightly seen in the \MOEAD{} after $250$ variables, however we believe that this \MOEA{} is highly parameter-sensitive.
%
A differently effect is presented in the UF5, in this problem the more bigger the variable space is, the less quality solutions are obtained.
%
Despite this the \VSDMOEA{} still attains the best \HV{} values.
%
However, the \VSDMOEA{} is not the best \MOEA{} in all the circumstances, it can be seen considering three objectives (Figure \ref{fig:variable-decision-scalability-3obj}).
%
Specifically, where the number of variables is increased to one thousand in both problems (DTLZ4 an d UF10) its performance is highly degraded.
%
This might occurs for several reasons, perhaps the most important is related with the measurement employed in the variable space (Euclidean distance), this inconvenient is popularly known as \textit{The Curse of Dimensionality} \cite{trunk1979problem, beyer1999nearest}.
%
This drawback means that under certain broad conditions, as dimensionality increases, the distance to the nearest neighbor approaches the distance to the farthest neighbor.
%
In other words, the contrast in distances to different data points becomes no-existent.
%
The remaining \MOEAS{} show a similar behavior with the DTLZ4 with two and three objectives.
%
As previously mentioned the performance of \VSDMOEA{} is seriously deteriorated, however it is still better than the \NSGAII{}.
%
Also, the \RMOEA{} seems to be enough stable with three objectives in the problems DTLZ4 and UF10.

\begin{figure}[t]
\centering
\begin{tikzpicture}[scale=0.8]
\begin{axis}[
ymin=0.5,
%ymax=1.1,
%x label style={at={(current axis.left of origin)},anchor=north, below=10mm},
title={\textit{\textbf{Scalability with Two Objectives}}},
    xlabel=Number of Decision Variables,
  ylabel=Mean HV,
%  xlabel style={yshift=-0.2cm},
  xticklabel style = {rotate=30},
  legend style=
    {cells={anchor=east},legend pos=outer north east,},
  % enlargelimits = false,
  xticklabels from table={\ScalabilityTwoObj}{mean},xtick=data]
  
\addplot[red,solid,mark=square*] 
table [y=VSD_MOEA,x=X]{\ScalabilityTwoObj};
\addlegendentry{VSD-MOEA}
\addplot[blue,dotted,mark=triangle*] table[y= CPDEA,x=X]{\ScalabilityTwoObj};
\addlegendentry{CPDEA}]
		
\addplot[orange,dashed,mark=diamond*] table [y= R2_MOEA,x=X]{\ScalabilityTwoObj};
\addlegendentry{R2-EMOA}]
    
\addplot[black,loosely dotted,mark=pentagon*] table [y= MOEA_D,x=X]{\ScalabilityTwoObj};
\addlegendentry{MOEA/D}]
\end{axis}
\end{tikzpicture}
%\begin{tikzpicture}[scale=0.8]
%\begin{axis}[
%%ymin=0.5    ,
%%ymax=1.1,
%%x label style={at={(current axis.left of origin)},anchor=north, below=10mm},
%title={\textit{\textbf{Scalability UF5}}},
%    xlabel=Number of Decision Variables,
%  ylabel=Mean HV,
%  xticklabel style = {rotate=30},
%  legend style=
%    {cells={anchor=east},legend pos=outer north east,},
%  % enlargelimits = false,
%  xticklabels from table={\ScalabilityUFFiveTwoObj}{mean},xtick=data]
%  
%\addplot[red,solid,mark=square*] 
%table [y=VSD_MOEA,x=X]{\ScalabilityUFFiveTwoObj};
%\addlegendentry{VSD-MOEA}
%\addplot[blue,dotted,mark=triangle*] table [y= NSGA_II,x=X]{\ScalabilityUFFiveTwoObj};
%\addlegendentry{NSGA-II}]
%		
%\addplot[orange,dashed,mark=diamond*] table [y= R2_MOEA,x=X]{\ScalabilityUFFiveTwoObj};
%\addlegendentry{R2-MOEA}]
%    
%\addplot[black,loosely dotted,mark=pentagon*] table [y= MOEA_D,x=X]{\ScalabilityUFFiveTwoObj};
%\addlegendentry{MOEA/D}]
%\end{axis}
%\end{tikzpicture}
%\caption{Mean of the \HV{} (35 runs) considering two objectives.}

\label{fig:variable-decision-scalability-2obj}
\end{figure}

\begin{figure}[t]
\centering
\begin{tikzpicture}[scale=0.8]
\begin{axis}[
ymin=0.0,
title={\textit{\textbf{Scalability with Three Objectives}}},
    xlabel=Number of Decision Variables,
  ylabel=Mean HV,
%  xlabel style={yshift=-0.2cm},
  xticklabel style = {rotate=30},
  legend style=
    {cells={anchor=east},legend pos=outer north east,},
  xticklabels from table={\ScalabilityThreeObj}{mean},xtick=data]
  
\addplot[red,solid,mark=square*] 
table [y=VSD_MOEA,x=X]{\ScalabilityThreeObj};
\addlegendentry{VSD-MOEA}
\addplot[blue,dotted,mark=triangle*] table [y= NSGA_II,x=X]{\ScalabilityThreeObj};
\addlegendentry{NSGA-II}]
		
\addplot[orange,dashed,mark=diamond*] table [y= R2_MOEA,x=X]{\ScalabilityThreeObj};
\addlegendentry{R2-EMOA}]
    
\addplot[black,loosely dotted,mark=pentagon*] table [y= MOEA_D,x=X]{\ScalabilityThreeObj};
\addlegendentry{MOEA/D}]
\end{axis}
\end{tikzpicture}
%\begin{tikzpicture}[scale=0.8]
%\begin{axis}[
%ymin=0.0,
%title={\textit{\textbf{Scalability UF10}}},
%    xlabel=Number of Decision Variables,
%  ylabel=Mean HV,
%  xticklabel style = {rotate=30},
%  legend style=
%    {cells={anchor=east},legend pos=outer north east,},
%  xticklabels from table={\ScalabilityUFTenThreeObj}{mean},xtick=data]
%  
%\addplot[red,solid,mark=square*] 
%table [y=VSD_MOEA,x=X]{\ScalabilityUFTenThreeObj};
%\addlegendentry{VSD-MOEA}
%\addplot[blue,dotted,mark=triangle*] table [y= NSGA_II,x=X]{\ScalabilityUFTenThreeObj};
%\addlegendentry{NSGA-II}]
%		
%\addplot[orange,dashed,mark=diamond*] table [y= R2_MOEA,x=X]{\ScalabilityUFTenThreeObj};
%\addlegendentry{R2-MOEA}]
%    
%\addplot[black,loosely dotted,mark=pentagon*] table [y= MOEA_D,x=X]{\ScalabilityUFTenThreeObj};
%\addlegendentry{MOEA/D}]
%\end{axis}
%\end{tikzpicture}
%\caption{Mean of the \HV{} (35 runs) considering three objectives.}

\label{fig:variable-decision-scalability-3obj}
\end{figure}

\subsection{Analyzes of the Initial Factor Distance}

The \VSDMOEA{} induces the diversity of the decision variable space through the initial distance factor ($D_I$), such parameter is decreased as the number of generations elapses.
%
Given that this parameter influences the performance of the algorithm, a detailed empirical analyzes is taken into account as follows.
%
Since that this parameter is computed as a fraction of the main normalized diagonal which belongs to the unitary hyper-cube, the portions considered are $ D_I = \{0.0, 0.1, 0.2, 0.3, 0.4, 0.5, 0.6, 0.7, 0.8, 0.9\}$.
%
In the Figure \ref{fig:Initial-distance-factor} is shown the general mean of the \HV{} for each configuration with two and three objectives respectively.
%
Particularly, the initial parameter $D_I=0.0$, which does not promote diversity, should have a similar performance than a classic \MOEA{}.
%
In spite that none diversity is promoted, the \VSDMOEA{} achieves better general mean values than the remaining \MOEAS{}, those values are $0.905$, $0.895$ for two and three objectives respectively.
%
Even more, the benefits of promoting diversity are outstanding from $0.909$ and $0.895$ of two and three objectives to $0.936$ and $0.899$ with $D_I=0.1$.
%
It seems that such benefits are more notorious with two objectives than with three.
%
This might occurs given that the population size might not been enough to cover the entire objective space.
%
Based on this empirical results, the most suitable parameter configuration should be set with $D_I = 0.4$.
%
\begin{figure}[t]
\centering
\begin{tikzpicture}[scale=0.8]
\begin{axis}[
title={\textit{\textbf{Mean of the HV Value with Several Initial Threshold Values}}},
    xlabel=Initial Threshold Value,
  ylabel=HV,
%  xlabel style={yshift=-0.2cm},
  xticklabel style = {rotate=30},
  legend style=
    {cells={anchor=east},legend pos=outer north east,},
  xticklabels from table={\DIHV}{mean},xtick=data]
  
\addplot[red,solid,mark=square*] 
table [y=HV2obj,x=X]{\DIHV};
\addlegendentry{HV - 2obj}
\addplot[blue,dotted,mark=triangle*] table[y= HV3obj,x=X]{\DIHV};
\addlegendentry{HV - 3obj}
		
\end{axis}
\end{tikzpicture}
%%\begin{tikzpicture}[scale=0.8]
%%\begin{axis}[
%%ymin=0.0    ,
%%ymax=0.1,
%%title={\textit{\textbf{Mean of the IGD+ Value with Several Initial Distance Factor}}},
%%    xlabel=Initial Distance Factor,
%%  ylabel=IGD+,
%%  xticklabel style = {rotate=30},
%%  legend style=
%%    {cells={anchor=east},legend pos=outer north east,},
%%  xticklabels from table={\DIIGDP}{mean},xtick=data]
%%\addplot[red,solid,mark=square*] 
%%table [y=IGD2obj,x=X]{\DIIGDP};
%%\addlegendentry{IGD+ 2obj}
%%\addplot[blue,dotted,mark=triangle*] table [y=IGD3obj,x=X]{\DIIGDP};
%%\addlegendentry{IGD+ 3obj}
%%\end{axis}
%%\end{tikzpicture}
%\caption{Mean of Inidicator Considering All Instances with Several Initial Distance Factors}

\label{fig:Initial-distance-factor}
\end{figure}

\subsection{Diversity of the \MOEAS{} Through Generations}

In order, to have a better understanding of the diversity behavior some WFG problems have been selected.
%
The WFG problems divide the decision variables in two kinds of parameter: the distance parameters and the position parameters.
%
A parameter $x_i$ is a distance parameter when for all parameter vectors $\vec{F}(\mathbf{x})$, modifying $x_i$ in $\vec{F}(\mathbf{x})$ results in a parameter vector that dominates $\vec{F}(\mathbf{x})$, is equivalent to $\vec{F}(\mathbf{x})$, or is dominated by $\vec{F}(\mathbf{x})$.
%
However, if $x_i$ is a position parameter, modifying $x_i$ in $\vec{F}(\mathbf{x})$ always results in a vector that is incomparable or equivalent to $\vec{F}(\mathbf{x})$~\cite{huband2005scalable}.
%

In this section we show that state-of-the-art-\MOEAS{} do not always maintain high enough diversity.
%
Particularly, the selected problems are used to show that premature convergence appears in the set of distance parameters.
%
Consequently, the operators involved lose its exploratory strength.
%
We select the WFG1, WFG5, and WFG6 problems, because they have simple definition, but most MOEAs faces difficulties with them. 
%
In addition, those problems were taken into account given that the WFG1 and WFG5 were better solved by the \VSDMOEA{}.
%
Generally speaking, the WFG1 converged to the Pareto Front with our proposal, this since that the accuracy obtained was of $0.993$.
%
In contrast, the WFG5 was still far away of the Pareto Front which \HV{} mean  was of $0.923$.
%
Contrarily, the WFG6 was chosen since that the \VSDMOEA{} achieved the worst results.
%
Whereas the WFG1 and WFG6 are uni-modal, the WFG5 is a highly deceptive problem.
%
Moreover, the WFG1 and WFG5 are conformed by the separable properties of its objective functions.
%
In fact, the distance parameters values associated to Pareto optimal solutions for WFG1-WFG7 have exactly the same values in the distance parameters.
%
This values is shown as follows:
\begin{equation}
   x_{i=k+1:n} = 2i \times 0.35
\end{equation}
%
Taking into account the stochastic behavior of \MOEAS{}, $35$ independent executions were run by each selected problem.
%
In all of them, the stopping criterion was set to $250,000$ generations.
%
In order to analyze the diversity, the average Euclidean distance among individuals (ADI) is calculated, i.e. the mean value of all pairwise distances among individuals in the population is reported.
%
In the Figures \ref{fig:Diversity_WFG1}, \ref{fig:Diversity_WFG5}, and \ref{fig:Diversity_WFG6} are showed the evolution of diversity of the WFG1, WFG5, and WFG6 respectively.
%
In those figures each \MOEA{} is showed by dashed and solid lines that represents two and three objectives respectively.
%
Particularly, the \VSDMOEA{} properly maintains diversity in both kind of parameters, while in the remaining algorithms the diversity in the distance parameters is lost after the $10\%$ (generation $2500$) of total generations.
%
Thus, after those \MOEAS{} converges, they are basically modifying the position parameters, so the majority of the time is improving further the diversity in the objectives space and the convergence is neglected.
%
In fact, this diversity issue is also present in the WFG5.
%
However, all the \MOEAS{} have a minimum lower bound of diversity in the position parameters of those selected problems.
%
Contrastively, considering the WFG5 (Figure \ref{fig:Diversity_WFG5}) the \VSDMOEA{} does not converge at all in the distance parameters with three objectives, this might be an effect of promoting too much diversity.
%
Besides this issue, the \VSDMOEA{} still achieves the best \HV{} values.
%
In addition, this drawback is more notorious in the problem WFG6.
%
In fact in two and three objectives this \MOEA{} did not converges in the distance parameters, and the position variables are still more diversified as in WFG1.


\begin{figure}[t]
\centering
\begin{tikzpicture}[scale=0.8]
\begin{axis}[
title={\textit{\textbf{Diversity of the Position Variables with Two Objectives}}},
  xlabel=Generation,
  ylabel=ADI,
  xticklabel style = {rotate=30},
%  xlabel style={yshift=-0.2cm},
  legend style=
    {cells={anchor=east},legend pos=outer north east,}, nodes={scale=0.8, transform shape}    ,
  xticklabels from table={\DiversityTwoWFG}{mean},xtick=data]
 

\addplot[red,dotted,mark=square*] table [y=VSD_MOEA_Position_50,x=X]{\DiversityTwoWFG};
	\addlegendentry{VSD-MOEA (50)}

\addplot[blue,dotted,mark=triangle*] table[y= NSGA_II_Position_50,x=X]{\DiversityTwoWFG};
	\addlegendentry{NSGA-II (50)}
		
\addplot[orange,dotted,mark=diamond*] table [y= R2_MOEA_Position_50,x=X]{\DiversityTwoWFG};
	\addlegendentry{R2-EMOA (50)}
    
\addplot[black,dotted,mark=pentagon*] table [y= MOEA_D_Position_50,x=X]{\DiversityTwoWFG};
	\addlegendentry{MOEA/D (50)}


\addplot[red,solid,mark=square*] table [y=VSD_MOEA_Position_100,x=X]{\DiversityTwoWFG};
	\addlegendentry{VSD-MOEA (100)}

\addplot[blue,solid,mark=triangle*] table[y= NSGA_II_Position_100,x=X]{\DiversityTwoWFG};
	\addlegendentry{NSGA-II (100)}
		
\addplot[orange,solid,mark=diamond*] table [y= R2_MOEA_Position_100,x=X]{\DiversityTwoWFG};
	\addlegendentry{R2-EMOA (100)}
    
\addplot[black,solid,mark=pentagon*] table [y= MOEA_D_Position_100,x=X]{\DiversityTwoWFG};
	\addlegendentry{MOEA/D (100)}


\addplot[red,loosely dashdotted,mark=square*] table [y=VSD_MOEA_Position_250,x=X]{\DiversityTwoWFG};
	\addlegendentry{VSD-MOEA (250)}

\addplot[blue,loosely dashdotted,mark=triangle*] table[y= NSGA_II_Position_250,x=X]{\DiversityTwoWFG};
	\addlegendentry{NSGA-II (250)}
		
\addplot[orange,loosely dashdotted,mark=diamond*] table [y= R2_MOEA_Position_250,x=X]{\DiversityTwoWFG};
	\addlegendentry{R2-EMOA (250)}
    
\addplot[black,loosely dashdotted,mark=pentagon*] table [y= MOEA_D_Position_250,x=X]{\DiversityTwoWFG};
	\addlegendentry{MOEA/D (250)}



\end{axis}
\end{tikzpicture}


\begin{tikzpicture}[scale=0.8]
\begin{axis}[
title={\textit{\textbf{Diversity of the Distance Variables with Two Objectives}}},
  xlabel=Generation,
  ylabel=ADI,
%  xlabel style={yshift=-0.2cm},
  xticklabel style = {rotate=30},
  legend style=
    {cells={anchor=east},legend pos=outer north east,}, nodes={scale=0.8, transform shape}    ,
  xticklabels from table={\DiversityTwoWFG}{mean},xtick=data]

\addplot[red,dotted,mark=square*] table [y=VSD_MOEA_Distance_50,x=X]{\DiversityTwoWFG};
	\addlegendentry{VSD-MOEA (50)}

%\addplot[blue,dotted,mark=triangle*] table[y= NSGA_II_Distance_50,x=X]{\DiversityTwoWFG};
%	\addlegendentry{NSGA-II (50)}
		
\addplot[orange,dotted,mark=diamond*] table [y= R2_MOEA_Distance_50,x=X]{\DiversityTwoWFG};
	\addlegendentry{R2-EMOA (50)}
    
%\addplot[black,dotted,mark=pentagon*] table [y= MOEA_D_Distance_50,x=X]{\DiversityTwoWFG};
%	\addlegendentry{MOEA/D (50)}


\addplot[red,solid,mark=square*] table [y=VSD_MOEA_Distance_100,x=X]{\DiversityTwoWFG};
	\addlegendentry{VSD-MOEA (100)}

%\addplot[blue,solid,mark=triangle*] table[y= NSGA_II_Distance_100,x=X]{\DiversityTwoWFG};
%	\addlegendentry{NSGA-II (100)}
		
\addplot[orange,solid,mark=diamond*] table [y= R2_MOEA_Distance_100,x=X]{\DiversityTwoWFG};
	\addlegendentry{R2-EMOA (100)}
    
%\addplot[black,solid,mark=pentagon*] table [y= MOEA_D_Distance_100,x=X]{\DiversityTwoWFG};
%	\addlegendentry{MOEA/D (100)}


\addplot[red,loosely dashdotted,mark=square*] table [y=VSD_MOEA_Distance_250,x=X]{\DiversityTwoWFG};
	\addlegendentry{VSD-MOEA (250)}

%\addplot[blue,loosely dashdotted,mark=triangle*] table[y= NSGA_II_Distance_250,x=X]{\DiversityTwoWFG};
%	\addlegendentry{NSGA-II (250)}
		
\addplot[orange,loosely dashdotted,mark=diamond*] table [y= R2_MOEA_Distance_250,x=X]{\DiversityTwoWFG};
	\addlegendentry{R2-EMOA (250)}
    
%\addplot[black,loosely dashdotted,mark=pentagon*] table [y= MOEA_D_Distance_250,x=X]{\DiversityTwoWFG};
%	\addlegendentry{MOEA/D (250)}





\end{axis}
\end{tikzpicture}

%%\begin{tikzpicture}[scale=0.8]
%%\begin{axis}[
%%title={\textit{\textbf{Diversity of the Variables}}},
%%  xlabel=Generation,
%%  ylabel=ADI,
%%  xticklabel style = {rotate=30},
%%  legend style=
%%    {cells={anchor=east},legend pos=outer north east,},
%%  xticklabels from table={\DiversityTwoWFGOne}{mean},xtick=data]
%%
%%\addplot[red,dotted,mark=square*] table [y=VSD_MOEA,x=X]{\DiversityTwoWFGOne};
%%\addlegendentry{VSD-MOEA}
%%
%%\addplot[blue,dotted,mark=triangle*] table[y= NSGA_II,x=X]{\DiversityTwoWFGOne};
%%\addlegendentry{NSGA-II}
%%		
%%\addplot[orange,dotted,mark=diamond*] table [y= R2_MOEA,x=X]{\DiversityTwoWFGOne};
%%\addlegendentry{R2-MOEA}
%%    
%%\addplot[black,dotted,mark=pentagon*] table [y= MOEA_D,x=X]{\DiversityTwoWFGOne};
%%\addlegendentry{MOEA/D}
%%
%%
%%\addplot[red,solid,mark=square*] table [y=VSD_MOEA,x=X]{\DiversityThreeWFGOne};
%%\addlegendentry{VSD-MOEA}
%%
%%\addplot[blue,solid,mark=triangle*] table[y= NSGA_II,x=X]{\DiversityThreeWFGOne};
%%\addlegendentry{NSGA-II}
%%		
%%\addplot[orange,solid,mark=diamond*] table [y= R2_MOEA,x=X]{\DiversityThreeWFGOne};
%%\addlegendentry{R2-MOEA}
%%    
%%\addplot[black,solid,mark=pentagon*] table [y= MOEA_D,x=X]{\DiversityThreeWFGOne};
%%\addlegendentry{MOEA/D}
%%
%%	
%%\end{axis}
%%\end{tikzpicture}
%\caption{Evolution of Average Distance Individuals of the Problems WFG1-WFG7.}


\label{fig:Diversity_2obj}
\end{figure}


\begin{figure}[t]
\centering
\begin{tikzpicture}[scale=0.8]
\begin{axis}[
title={\textit{\textbf{Diversity of the Position Variables with Three Objectives}}},
  xlabel=Function Evaluations,
  ylabel=ADI,
%  xlabel style={yshift=-0.2cm},
 xticklabel style = {rotate=30},
  legend style=
   {cells={anchor=east},legend pos=outer north east,}, nodes={scale=0.8, transform shape}    ,
    %{cells={anchor=east},legend pos=outer north east,},
  xticklabels from table={\DiversityThreeWFG}{mean},xtick=data]
 

\addplot[red,dotted,mark=square*] table [y=VSD_MOEA_Position_50,x=X]{\DiversityThreeWFG};
	\addlegendentry{VSD-MOEA (50)}

\addplot[blue,dotted,mark=triangle*] table[y= CPDEA_Position_50,x=X]{\DiversityThreeWFG};
	\addlegendentry{CPDEA (50)}
		
\addplot[orange,dotted,mark=diamond*] table [y= R2_MOEA_Position_50,x=X]{\DiversityThreeWFG};
	\addlegendentry{R2-EMOA (50)}
    
\addplot[black,dotted,mark=pentagon*] table [y= MOEA_D_Position_50,x=X]{\DiversityThreeWFG};
	\addlegendentry{MOEA/D (50)}


\addplot[red,solid,mark=square*] table [y=VSD_MOEA_Position_100,x=X]{\DiversityThreeWFG};
	\addlegendentry{VSD-MOEA (100)}

\addplot[blue,solid,mark=triangle*] table[y= CPDEA_Position_100,x=X]{\DiversityThreeWFG};
	\addlegendentry{CPDEA (100)}
		
\addplot[orange,solid,mark=diamond*] table [y= R2_MOEA_Position_100,x=X]{\DiversityThreeWFG};
	\addlegendentry{R2-EMOA (100)}
    
\addplot[black,solid,mark=pentagon*] table [y= MOEA_D_Position_100,x=X]{\DiversityThreeWFG};
	\addlegendentry{MOEA/D (100)}


\addplot[red,loosely dashdotted,mark=square*] table [y=VSD_MOEA_Position_250,x=X]{\DiversityThreeWFG};
	\addlegendentry{VSD-MOEA (250)}

\addplot[blue,loosely dashdotted,mark=triangle*] table[y= CPDEA_Position_250,x=X]{\DiversityThreeWFG};
	\addlegendentry{CPDEA (250)}
		
\addplot[orange,loosely dashdotted,mark=diamond*] table [y= R2_MOEA_Position_250,x=X]{\DiversityThreeWFG};
	\addlegendentry{R2-EMOA (250)}
    
\addplot[black,loosely dashdotted,mark=pentagon*] table [y= MOEA_D_Position_250,x=X]{\DiversityThreeWFG};
	\addlegendentry{MOEA/D (250)}



\end{axis}
\end{tikzpicture}


\begin{tikzpicture}[scale=0.8]
\begin{axis}[
title={\textit{\textbf{Diversity of the Distance Variables with Three Objectives}}},
  xlabel=Function Evaluations,
  ylabel=ADI,
	ymax=0.6,
  xticklabel style = {rotate=30},
   ytick={ 0, 0.1, 0.2, 0.3, 0.4, 0.5, 0.6},
%  xlabel style={yshift=-0.2cm},
  legend style=
   {cells={anchor=east},legend pos=outer north east,}, nodes={scale=0.8, transform shape}    ,
    %{cells={anchor=east},legend pos=outer north east,},
  xticklabels from table={\DiversityThreeWFG}{mean},xtick=data]

\addplot[red,dotted,mark=square*] table [y=VSD_MOEA_Distance_50,x=X]{\DiversityThreeWFG};
	\addlegendentry{VSD-MOEA (50)}

\addplot[blue,dotted,mark=triangle*] table[y= CPDEA_Distance_50,x=X]{\DiversityThreeWFG};
	\addlegendentry{CPDEA (50)}
		
\addplot[orange,dotted,mark=diamond*] table [y= R2_MOEA_Distance_50,x=X]{\DiversityThreeWFG};
	\addlegendentry{R2-EMOA (50)}
    
%\addplot[black,dotted,mark=pentagon*] table [y= MOEA_D_Distance_50,x=X]{\DiversityThreeWFG};
%	\addlegendentry{MOEA/D (50)}


%\addplot[red,solid,mark=square*] table [y=VSD_MOEA_Distance_100,x=X]{\DiversityThreeWFG};
%	\addlegendentry{VSD-MOEA (100)}

%\addplot[blue,solid,mark=triangle*] table[y= CPDEA_Distance_100,x=X]{\DiversityThreeWFG};
%	\addlegendentry{CPDEA (100)}
		
%\addplot[orange,solid,mark=diamond*] table [y= R2_MOEA_Distance_100,x=X]{\DiversityThreeWFG};
%	\addlegendentry{R2-EMOA (100)}
    
%\addplot[black,solid,mark=pentagon*] table [y= MOEA_D_Distance_100,x=X]{\DiversityThreeWFG};
%	\addlegendentry{MOEA/D (100)}


\addplot[red,loosely dashdotted,mark=square*] table [y=VSD_MOEA_Distance_250,x=X]{\DiversityThreeWFG};
	\addlegendentry{VSD-MOEA (250)}

\addplot[blue,loosely dashdotted,mark=triangle*] table[y= CPDEA_Distance_250,x=X]{\DiversityThreeWFG};
	\addlegendentry{CPDEA (250)}
		
\addplot[orange,loosely dashdotted,mark=diamond*] table [y= R2_MOEA_Distance_250,x=X]{\DiversityThreeWFG};
	\addlegendentry{R2-EMOA (250)}
    
%\addplot[black,loosely dashdotted,mark=pentagon*] table [y= MOEA_D_Distance_250,x=X]{\DiversityThreeWFG};
%	\addlegendentry{MOEA/D (250)}





\end{axis}
\end{tikzpicture}

%%\begin{tikzpicture}[scale=0.8]
%%\begin{axis}[
%%title={\textit{\textbf{Diversity of the Variables}}},
%%  xlabel=Generation,
%%  ylabel=ADI,
%%  xticklabel style = {rotate=30},
%%  legend style=
%%    {cells={anchor=east},legend pos=outer north east,},
%%  xticklabels from table={\DiversityThreeWFGOne}{mean},xtick=data]
%%
%%\addplot[red,dotted,mark=square*] table [y=VSD_MOEA,x=X]{\DiversityThreeWFGOne};
%%\addlegendentry{VSD-MOEA}
%%
%%\addplot[blue,dotted,mark=triangle*] table[y= CPDEA,x=X]{\DiversityThreeWFGOne};
%%\addlegendentry{CPDEA}
%%		
%%\addplot[orange,dotted,mark=diamond*] table [y= R2_MOEA,x=X]{\DiversityThreeWFGOne};
%%\addlegendentry{R2-MOEA}
%%    
%%\addplot[black,dotted,mark=pentagon*] table [y= MOEA_D,x=X]{\DiversityThreeWFGOne};
%%\addlegendentry{MOEA/D}
%%
%%
%%\addplot[red,solid,mark=square*] table [y=VSD_MOEA,x=X]{\DiversityThreeWFGOne};
%%\addlegendentry{VSD-MOEA}
%%
%%\addplot[blue,solid,mark=triangle*] table[y= CPDEA,x=X]{\DiversityThreeWFGOne};
%%\addlegendentry{CPDEA}
%%		
%%\addplot[orange,solid,mark=diamond*] table [y= R2_MOEA,x=X]{\DiversityThreeWFGOne};
%%\addlegendentry{R2-MOEA}
%%    
%%\addplot[black,solid,mark=pentagon*] table [y= MOEA_D,x=X]{\DiversityThreeWFGOne};
%%\addlegendentry{MOEA/D}
%%
%%	
%%\end{axis}
%%\end{tikzpicture}
%\caption{Evolution of Average Distance Individuals of the Problems WFG1-WFG7.}

\label{fig:Diversity_3obj}
\end{figure}


%\begin{figure}[t]
%\centering
%\begin{tikzpicture}[scale=0.8]
\begin{axis}[
title={\textit{\textbf{Diversity of the Position Variables}}},
  xlabel=Generation,
  ylabel=ADI,
  xticklabel style = {rotate=30},
  legend style=
    {cells={anchor=east},legend pos=outer north east,},
  xticklabels from table={\DiversityTwoWFGSix}{mean},xtick=data]
 

\addplot[red,dotted,mark=square*] table [y=VSD_MOEA_Position,x=X]{\DiversityTwoWFGSix};
\addlegendentry{VSD-MOEA}

\addplot[blue,dotted,mark=triangle*] table[y= NSGA_II_Position,x=X]{\DiversityTwoWFGSix};
\addlegendentry{NSGA-II}
		
\addplot[orange,dotted,mark=diamond*] table [y= R2_MOEA_Position,x=X]{\DiversityTwoWFGSix};
\addlegendentry{R2-MOEA}
    
\addplot[black,dotted,mark=pentagon*] table [y= MOEA_D_Position,x=X]{\DiversityTwoWFGSix};
\addlegendentry{MOEA/D}


\addplot[red,solid,mark=square*] table [y=VSD_MOEA_Position,x=X]{\DiversityThreeWFGSix};
\addlegendentry{VSD-MOEA}

\addplot[blue,solid,mark=triangle*] table[y= NSGA_II_Position,x=X]{\DiversityThreeWFGSix};
\addlegendentry{NSGA-II}
		
\addplot[orange,solid,mark=diamond*] table [y= R2_MOEA_Position,x=X]{\DiversityThreeWFGSix};
\addlegendentry{R2-MOEA}
    
\addplot[black,solid,mark=pentagon*] table [y= MOEA_D_Position,x=X]{\DiversityThreeWFGSix};
\addlegendentry{MOEA/D}


\end{axis}
\end{tikzpicture}


\begin{tikzpicture}[scale=0.8]
\begin{axis}[
title={\textit{\textbf{Diversity of the Distance Variables}}},
  xlabel=Generation,
  ylabel=ADI,
  xticklabel style = {rotate=30},
  legend style=
    {cells={anchor=east},legend pos=outer north east,},
  xticklabels from table={\DiversityTwoWFGSix}{mean},xtick=data]

\addplot[red,dotted,mark=square*] table [y=VSD_MOEA_Distance,x=X]{\DiversityTwoWFGSix};
\addlegendentry{VSD-MOEA}

\addplot[blue,dotted,mark=triangle*] table[y= NSGA_II_Distance,x=X]{\DiversityTwoWFGSix};
\addlegendentry{NSGA-II}
		
\addplot[orange,dotted,mark=diamond*] table [y= R2_MOEA_Distance,x=X]{\DiversityTwoWFGSix};
\addlegendentry{R2-MOEA}
    
\addplot[black,dotted,mark=pentagon*] table [y= MOEA_D_Distance,x=X]{\DiversityTwoWFGSix};
\addlegendentry{MOEA/D}


\addplot[red,solid,mark=square*] table [y=VSD_MOEA_Distance,x=X]{\DiversityThreeWFGSix};
\addlegendentry{VSD-MOEA}

\addplot[blue,solid,mark=triangle*] table[y= NSGA_II_Distance,x=X]{\DiversityThreeWFGSix};
\addlegendentry{NSGA-II}
		
\addplot[orange,solid,mark=diamond*] table [y= R2_MOEA_Distance,x=X]{\DiversityThreeWFGSix};
\addlegendentry{R2-MOEA}
    
\addplot[black,solid,mark=pentagon*] table [y= MOEA_D_Distance,x=X]{\DiversityThreeWFGSix};
\addlegendentry{MOEA/D}

	
\end{axis}
\end{tikzpicture}

\begin{tikzpicture}[scale=0.8]
\begin{axis}[
title={\textit{\textbf{Diversity of the Variables}}},
  xlabel=Generation,
  ylabel=ADI,
  xticklabel style = {rotate=30},
  legend style=
    {cells={anchor=east},legend pos=outer north east,},
  xticklabels from table={\DiversityTwoWFGSix}{mean},xtick=data]

\addplot[red,dotted,mark=square*] table [y=VSD_MOEA,x=X]{\DiversityTwoWFGSix};
\addlegendentry{VSD-MOEA}

\addplot[blue,dotted,mark=triangle*] table[y= NSGA_II,x=X]{\DiversityTwoWFGSix};
\addlegendentry{NSGA-II}
		
\addplot[orange,dotted,mark=diamond*] table [y= R2_MOEA,x=X]{\DiversityTwoWFGSix};
\addlegendentry{R2-MOEA}
    
\addplot[black,dotted,mark=pentagon*] table [y= MOEA_D,x=X]{\DiversityTwoWFGSix};
\addlegendentry{MOEA/D}


\addplot[red,solid,mark=square*] table [y=VSD_MOEA,x=X]{\DiversityThreeWFGSix};
\addlegendentry{VSD-MOEA}

\addplot[blue,solid,mark=triangle*] table[y= NSGA_II,x=X]{\DiversityThreeWFGSix};
\addlegendentry{NSGA-II}
		
\addplot[orange,solid,mark=diamond*] table [y= R2_MOEA,x=X]{\DiversityThreeWFGSix};
\addlegendentry{R2-MOEA}
    
\addplot[black,solid,mark=pentagon*] table [y= MOEA_D,x=X]{\DiversityThreeWFGSix};
\addlegendentry{MOEA/D}

	
\end{axis}
\end{tikzpicture}

\caption{Evolution of the diversity for the problem WFG6}


%\label{fig:Diversity_WFG6}
%\end{figure}


\subsection{Performance of the \MOEAS{} Related with the Criteria Stop}

In spite that the \VSDMOEA{} is specially designed to attain quality solutions in long-term executions.
%
In this section is showed the performance of the \MOEAS{} modifying the criteria stop.
%
Mainly, three ranges of criteria stop were reviewed.
%
Each range was split in ten intervals, such ranges considered were $[300, 2500]$, $[2500, 25000]$ and $[25000, 250000]$ respectively.
%
In the Figure \ref{fig:Performance_time} are showed the mean \HV{} values attained with each \MOEA{} with two and three objectives respectively.
%
This shows that the \VSDMOEA{} attains the worst \HV{} values considering only $300$ generations.
%
However, as the number of generations increases the \HV{} values are improved significantly, in fact after $2500$ generations the \VSDMOEA{} has a notorious improvement respect to the state-of-the-art-\MOEAS{} in both two and three objectives.
%
It can be seen that as the maximum number of generations is higher than $25000$ the remaining \MOEAS{} seems to be stagnated, specifically considering two objectives.
%
Moreover, considering three objectives the \RMOEA{} shows improvements after $100000$ generations, however it still has lower values than the \VSDMOEA{}.
%
In addition, taking into account three objectives the best general mean value achieved was above $0.9$, there is still possible that increasing the number of generations the \VSDMOEA{} achieves better results.
%
However, in some circumstances, as seems to be in the case of two objectives, might not be able to achieve a better accuracy, thus an upper bound could be present.
%
It is significant important to note how the performance of the \NSGAII{} is degraded in three objectives, differently to the \RMOEA{}, and the \MOEAD{}.
%
Even more, the \MOEAD{} seems to be stagnated after $25000$ generations, this might occurs for the greedy selection approach that this \MOEA{} incorporates.
%
In fact the \MOEAD{} attained better results than the \RMOEA{} in short-term executions.
%

\begin{figure}[t]
\centering
\begin{tikzpicture}[scale=0.8]
\begin{axis}[
title={\textit{\textbf{Performance of \MOEAS{} with Two Objectives}}},
  xlabel= Maximum Number of Generations,
  ylabel=Mean HV,
%  ymin=0.3,
  ymax=1.0,
  xticklabel style = {rotate=90, anchor=east, font=\footnotesize,},
%  label style = {font=\small, },
%xlabel style={yshift=-1cm},
%  xmode=log,
%  log basis x={2},
  legend style=
    {cells={anchor=east},legend pos=outer north east,},
  xticklabels from table={\Performancetimetwo}{mean},xtick=data]
 

\addplot[red,dotted,mark=square*] table [y=VSD_MOEA,x=X]{\Performancetimetwo};
\addlegendentry{VSD-MOEA}

\addplot[blue,dotted,mark=triangle*] table[y= NSGA_II,x=X]{\Performancetimetwo};
\addlegendentry{NSGA-II}
		
\addplot[orange,dotted,mark=diamond*] table [y= R2_MOEA,x=X]{\Performancetimetwo};
\addlegendentry{R2-MOEA}
    
\addplot[black,dotted,mark=pentagon*] table [y= MOEA_D,x=X]{\Performancetimetwo};
\addlegendentry{MOEA/D}
\draw [red, dashed] (axis cs:10, 0) -- (axis cs:10, 1);
\draw [red, dashed] (axis cs:19, 0) -- (axis cs:19, 1);
\end{axis}
\end{tikzpicture}


\begin{tikzpicture}[scale=0.8]
\begin{axis}[
title={\textit{\textbf{Performance of \MOEAS{} with Three Objectives}}},
  xlabel= Maximum Number of Generations,
  ylabel=Mean HV,
%  ymin=0.5,
 % ymax=1.0,
  xticklabel style = {rotate=90, anchor=east, font=\footnotesize,},
  legend style=
    {cells={anchor=east},legend pos=outer north east,},
  xticklabels from table={\Performancetimethree}{mean},xtick=data]

\addplot[red,dashed,mark=square*] table [y=VSD_MOEA,x=X]{\Performancetimethree};
\addlegendentry{VSD-MOEA}

\addplot[blue,dashed,mark=triangle*] table[y= NSGA_II,x=X]{\Performancetimethree};
\addlegendentry{NSGA-II}
		
\addplot[orange,dashed,mark=diamond*] table [y= R2_MOEA,x=X]{\Performancetimethree};
\addlegendentry{R2-MOEA}
    
\addplot[black,dashed,mark=pentagon*] table [y= MOEA_D,x=X]{\Performancetimethree};
\addlegendentry{MOEA/D};
\draw [red, dashed] (axis cs:10, 0) -- (axis cs:10, 1);
\draw [red, dashed] (axis cs:19, 0) -- (axis cs:19, 1);
\end{axis}
\end{tikzpicture}



\caption{Performance of the \MOEAS{} considerin several maximum number of generations.}

\label{fig:Performance_time}
\end{figure}



% Please add the following required packages to your document preamble:
% \usepackage{graphicx}
\begin{table*}[]
\centering
\caption{Statistics HV with two objectives}
\label{tab:StatisticsHV_2obj}
\resizebox{\textwidth}{!}{%
\begin{tabular}{c c|c|c|c|c|c|c|c|c|c|c|c|c|c|c|c}
\cline{2-17}
 & \multicolumn{4}{c|}{\textbf{MOEA/D}} & \multicolumn{4}{c|}{\textbf{NSGA-II}} & \multicolumn{4}{c|}{\textbf{R2-MOEA}} & \multicolumn{4}{c}{\textbf{VSD-MOEA}} \\ \cline{2-17} 
 & \textbf{Min} & \textbf{Max} & \textbf{Mean} & \textbf{Std} & \textbf{Min} & \textbf{Max} & \textbf{Mean} & \textbf{Std} & \textbf{Min} & \textbf{Max} & \textbf{Mean} & \textbf{Std} & \textbf{Min} & \textbf{Max} & \textbf{Mean} & \textbf{Std} \\ \hline
\multicolumn{1}{c|}{\textbf{WFG1}} & 0.984 & 0.993 & 0.992 & 0.002 & 0.987 & 0.993 & 0.992 & 0.002 & 0.946 & 0.994 & 0.988 & 0.012 & 0.984 & 0.994 & 0.992 & 0.003 \\ \hline
\multicolumn{1}{c|}{\textbf{WFG2}} & 0.965 & 0.996 & 0.967 & 0.007 & 0.966 & 0.998 & 0.974 & 0.014 & 0.965 & 0.966 & 0.966 & 0.000 & 0.998 & 0.998 & 0.998 & 0.000 \\ \hline
\multicolumn{1}{c|}{\textbf{WFG3}} & 0.992 & 0.992 & 0.992 & 0.000 & 0.987 & 0.988 & 0.987 & 0.000 & 0.991 & 0.992 & 0.991 & 0.000 & 0.992 & 0.992 & 0.992 & 0.000 \\ \hline
\multicolumn{1}{c|}{\textbf{WFG4}} & 0.988 & 0.988 & 0.988 & 0.000 & 0.983 & 0.987 & 0.985 & 0.001 & 0.991 & 0.991 & 0.991 & 0.000 & 0.990 & 0.990 & 0.990 & 0.000 \\ \hline
\multicolumn{1}{c|}{\textbf{WFG5}} & 0.876 & 0.893 & 0.882 & 0.005 & 0.884 & 0.899 & 0.890 & 0.002 & 0.886 & 0.895 & 0.891 & 0.003 & 0.911 & 0.946 & 0.926 & 0.008 \\ \hline
\multicolumn{1}{c|}{\textbf{WFG6}} & 0.879 & 0.940 & 0.914 & 0.016 & 0.894 & 0.942 & 0.913 & 0.012 & 0.875 & 0.942 & 0.912 & 0.015 & 0.858 & 0.885 & 0.869 & 0.006 \\ \hline
\multicolumn{1}{c|}{\textbf{WFG7}} & 0.988 & 0.988 & 0.988 & 0.000 & 0.983 & 0.987 & 0.984 & 0.001 & 0.991 & 0.991 & 0.991 & 0.000 & 0.990 & 0.990 & 0.990 & 0.000 \\ \hline
\multicolumn{1}{c|}{\textbf{WFG8}} & 0.800 & 0.822 & 0.811 & 0.006 & 0.771 & 0.801 & 0.789 & 0.006 & 0.803 & 0.824 & 0.815 & 0.005 & 0.830 & 0.955 & 0.947 & 0.020 \\ \hline
\multicolumn{1}{c|}{\textbf{WFG9}} & 0.795 & 0.972 & 0.883 & 0.082 & 0.793 & 0.966 & 0.832 & 0.070 & 0.797 & 0.976 & 0.884 & 0.079 & 0.964 & 0.975 & 0.970 & 0.003 \\ \hline
\multicolumn{1}{c|}{\textbf{DTLZ1}} & 0.993 & 0.993 & 0.993 & 0.000 & 0.990 & 0.992 & 0.991 & 0.000 & 0.992 & 0.992 & 0.992 & 0.000 & 0.992 & 0.992 & 0.992 & 0.000 \\ \hline
\multicolumn{1}{c|}{\textbf{DTLZ2}} & 0.989 & 0.989 & 0.989 & 0.000 & 0.986 & 0.988 & 0.987 & 0.000 & 0.991 & 0.992 & 0.992 & 0.000 & 0.990 & 0.990 & 0.990 & 0.000 \\ \hline
\multicolumn{1}{c|}{\textbf{DTLZ3}} & 0.989 & 0.989 & 0.989 & 0.000 & 0.987 & 0.989 & 0.989 & 0.001 & 0.991 & 0.992 & 0.992 & 0.000 & 0.990 & 0.990 & 0.990 & 0.000 \\ \hline
\multicolumn{1}{c|}{\textbf{DTLZ4}} & 0.259 & 0.989 & 0.781 & 0.330 & 0.259 & 0.988 & 0.863 & 0.274 & 0.259 & 0.992 & 0.657 & 0.365 & 0.990 & 0.990 & 0.990 & 0.000 \\ \hline
\multicolumn{1}{c|}{\textbf{DTLZ5}} & 0.989 & 0.989 & 0.989 & 0.000 & 0.986 & 0.988 & 0.987 & 0.000 & 0.991 & 0.992 & 0.992 & 0.000 & 0.990 & 0.990 & 0.990 & 0.000 \\ \hline
\multicolumn{1}{c|}{\textbf{DTLZ6}} & 0.448 & 0.910 & 0.700 & 0.105 & 0.138 & 0.511 & 0.322 & 0.075 & 0.510 & 0.922 & 0.691 & 0.107 & 0.990 & 0.990 & 0.990 & 0.000 \\ \hline
\multicolumn{1}{c|}{\textbf{DTLZ7}} & 0.996 & 0.996 & 0.996 & 0.000 & 0.996 & 0.997 & 0.996 & 0.000 & 0.997 & 0.997 & 0.997 & 0.000 & 0.996 & 0.996 & 0.996 & 0.000 \\ \hline
\multicolumn{1}{c|}{\textbf{UF1}} & 0.991 & 0.993 & 0.992 & 0.000 & 0.986 & 0.989 & 0.988 & 0.000 & 0.978 & 0.994 & 0.990 & 0.005 & 0.992 & 0.995 & 0.994 & 0.000 \\ \hline
\multicolumn{1}{c|}{\textbf{UF2}} & 0.987 & 0.993 & 0.991 & 0.002 & 0.980 & 0.983 & 0.981 & 0.001 & 0.984 & 0.991 & 0.988 & 0.002 & 0.986 & 0.992 & 0.989 & 0.002 \\ \hline
\multicolumn{1}{c|}{\textbf{UF3}} & 0.481 & 0.674 & 0.597 & 0.043 & 0.678 & 0.871 & 0.784 & 0.048 & 0.531 & 0.704 & 0.589 & 0.041 & 0.805 & 0.909 & 0.867 & 0.025 \\ \hline
\multicolumn{1}{c|}{\textbf{UF4}} & 0.881 & 0.917 & 0.908 & 0.006 & 0.875 & 0.910 & 0.889 & 0.008 & 0.923 & 0.935 & 0.929 & 0.003 & 0.920 & 0.930 & 0.925 & 0.002 \\ \hline
\multicolumn{1}{c|}{\textbf{UF5}} & 0.035 & 0.792 & 0.484 & 0.165 & 0.256 & 0.766 & 0.641 & 0.104 & 0.123 & 0.792 & 0.566 & 0.192 & 0.586 & 0.762 & 0.658 & 0.043 \\ \hline
\multicolumn{1}{c|}{\textbf{UF6}} & 0.255 & 0.711 & 0.447 & 0.114 & 0.235 & 0.801 & 0.635 & 0.120 & 0.349 & 0.767 & 0.568 & 0.113 & 0.668 & 0.922 & 0.827 & 0.080 \\ \hline
\multicolumn{1}{c|}{\textbf{UF7}} & 0.987 & 0.991 & 0.990 & 0.001 & 0.980 & 0.983 & 0.981 & 0.001 & 0.557 & 0.991 & 0.910 & 0.150 & 0.975 & 0.991 & 0.988 & 0.003 \\ \hline
\multicolumn{1}{c|}{\textbf{Mean}} & 0.806 & 0.935 & 0.881 & 0.038 & 0.808 & 0.927 & 0.886 & 0.032 & 0.801 & 0.940 & 0.882 & 0.048 & 0.930 & 0.964 & 0.951 & 0.008 \\ \hline
\end{tabular}%
}
\end{table*}



%\begin{table*}
%\centering
%\caption{Statistics HV with two objectives}
%\label{tab:StatisticsHV_2obj}
%\resizebox{\textwidth}{!}{%
%\begin{tabular}{c|c|c|c|c|c|c|c|c|c|c|c|c|c|c|c|c} 
%\cline{2-17}
%\multicolumn{1}{c}{} & \multicolumn{4}{c|}{\textbf{MOEA/D} }                                 & \multicolumn{4}{c|}{\textbf{NSGA-II} }                                & \multicolumn{4}{c|}{\textbf{R2-MOEA} }                                & \multicolumn{4}{c}{\textbf{VSD-MOEA} }                                \\ 
%\cline{2-17}
%\multicolumn{1}{c}{} & \textbf{Min}    & \textbf{Max}    & \textbf{Mean}   & \textbf{Std}    & \textbf{Min}    & \textbf{Max}    & \textbf{Mean}   & \textbf{Std}    & \textbf{Min}    & \textbf{Max}    & \textbf{Mean}   & \textbf{Std}    & \textbf{Min}    & \textbf{Max}    & \textbf{Mean}   & \textbf{Std}     \\ 
%\hline
%\textbf{WFG1}         & 0.984           & 0.993           & 0.992           & 0.002           & 0.987           & 0.993           & 0.992           & 0.002           & 0.946           & 0.994           & 0.988           & 0.012           & 0.975           & 0.994           & \textbf{0.993 } & 0.003            \\ 
%\hline
%\textbf{WFG2}         & 0.965           & 0.996           & 0.967           & 0.007           & 0.966           & 0.998           & 0.974           & 0.014           & 0.965           & 0.966           & 0.966           & 0.000           & 0.998           & 0.998           & \textbf{0.998 } & 0.000            \\ 
%\hline
%\textbf{WFG3}         & 0.992           & 0.992           & 0.992           & 0.000           & 0.987           & 0.988           & 0.987           & 0.000           & 0.991           & 0.992           & 0.991           & 0.000           & 0.992           & 0.992           & \textbf{0.992 } & 0.000            \\ 
%\hline
%\textbf{WFG4}         & 0.988           & 0.988           & 0.988           & 0.000           & 0.983           & 0.987           & 0.985           & 0.001           & 0.991           & 0.991           & \textbf{0.991 } & 0.000           & 0.990           & 0.990           & 0.990           & 0.000            \\ 
%\hline
%\textbf{WFG5}         & 0.876           & 0.893           & 0.882           & 0.005           & 0.884           & 0.899           & 0.890           & 0.002           & 0.886           & 0.895           & 0.891           & 0.003           & 0.901           & 0.937           & \textbf{0.923 } & 0.008            \\ 
%\hline
%\textbf{WFG6}         & 0.879           & 0.940           & \textbf{0.914 } & 0.016           & 0.894           & 0.942           & 0.913           & 0.012           & 0.875           & 0.942           & 0.912           & 0.015           & 0.852           & 0.886           & 0.868           & 0.008            \\ 
%\hline
%\textbf{WFG7}         & 0.988           & 0.988           & 0.988           & 0.000           & 0.983           & 0.987           & 0.984           & 0.001           & 0.991           & 0.991           & \textbf{0.991 } & 0.000           & 0.990           & 0.990           & 0.990           & 0.000            \\ 
%\hline
%\textbf{WFG8}         & 0.800           & 0.822           & 0.811           & 0.006           & 0.771           & 0.801           & 0.789           & 0.006           & 0.803           & 0.824           & 0.815           & 0.005           & 0.945           & 0.959           & \textbf{0.953 } & 0.003            \\ 
%\hline
%\textbf{WFG9}         & 0.795           & 0.972           & 0.883           & 0.082           & 0.793           & 0.966           & 0.832           & 0.070           & 0.797           & 0.976           & 0.884           & 0.079           & 0.960           & 0.976           & \textbf{0.969 } & 0.004            \\ 
%\hline
%\textbf{DTLZ1}        & 0.993           & 0.993           & \textbf{0.993 } & 0.000           & 0.990           & 0.992           & 0.991           & 0.000           & 0.992           & 0.992           & 0.992           & 0.000           & 0.992           & 0.992           & 0.992           & 0.000            \\ 
%\hline
%\textbf{DTLZ2}        & 0.989           & 0.989           & 0.989           & 0.000           & 0.986           & 0.988           & 0.987           & 0.000           & 0.991           & 0.992           & \textbf{0.992 } & 0.000           & 0.990           & 0.990           & 0.990           & 0.000            \\ 
%\hline
%\textbf{DTLZ3}        & 0.989           & 0.989           & 0.989           & 0.000           & 0.987           & 0.989           & 0.989           & 0.001           & 0.991           & 0.992           & \textbf{0.992 } & 0.000           & 0.990           & 0.990           & 0.990           & 0.000            \\ 
%\hline
%\textbf{DTLZ4}        & 0.259           & 0.989           & 0.781           & 0.330           & 0.259           & 0.988           & 0.863           & 0.274           & 0.259           & 0.992           & 0.657           & 0.365           & 0.990           & 0.990           & \textbf{0.990 } & 0.000            \\ 
%\hline
%\textbf{DTLZ5}        & 0.989           & 0.989           & 0.989           & 0.000           & 0.986           & 0.988           & 0.987           & 0.000           & 0.991           & 0.992           & \textbf{0.992 } & 0.000           & 0.990           & 0.990           & 0.990           & 0.000            \\ 
%\hline
%\textbf{DTLZ6}        & 0.448           & 0.910           & 0.700           & 0.105           & 0.138           & 0.511           & 0.322           & 0.075           & 0.510           & 0.922           & 0.691           & 0.107           & 0.990           & 0.990           & \textbf{0.990 } & 0.000            \\ 
%\hline
%\textbf{DTLZ7}        & 0.996           & 0.996           & 0.996           & 0.000           & 0.996           & 0.997           & 0.996           & 0.000           & 0.997           & 0.997           & \textbf{0.997 } & 0.000           & 0.996           & 0.996           & 0.996           & 0.000            \\ 
%\hline
%\textbf{UF1}          & 0.991           & 0.993           & 0.992           & 0.000           & 0.986           & 0.989           & 0.988           & 0.000           & 0.978           & 0.994           & 0.990           & 0.005           & 0.994           & 0.995           & \textbf{0.994 } & 0.000            \\ 
%\hline
%\textbf{UF2}          & 0.987           & 0.993           & \textbf{0.991 } & 0.002           & 0.980           & 0.983           & 0.981           & 0.001           & 0.984           & 0.991           & 0.988           & 0.002           & 0.983           & 0.991           & 0.988           & 0.002            \\ 
%\hline
%\textbf{UF3}          & 0.481           & 0.674           & 0.597           & 0.043           & 0.678           & 0.871           & 0.784           & 0.048           & 0.531           & 0.704           & 0.589           & 0.041           & 0.822           & 0.904           & \textbf{0.881 } & 0.015            \\ 
%\hline
%\textbf{UF4}          & 0.881           & 0.917           & 0.908           & 0.006           & 0.875           & 0.910           & 0.889           & 0.008           & 0.923           & 0.935           & \textbf{0.929 } & 0.003           & 0.920           & 0.931           & 0.925           & 0.002            \\ 
%\hline
%\textbf{UF5}          & 0.035           & 0.792           & 0.484           & 0.165           & 0.256           & 0.766           & 0.641           & 0.104           & 0.123           & 0.792           & 0.566           & 0.192           & 0.628           & 0.787           & \textbf{0.688 } & 0.041            \\ 
%\hline
%\textbf{UF6}          & 0.255           & 0.711           & 0.447           & 0.114           & 0.235           & 0.801           & 0.635           & 0.120           & 0.349           & 0.767           & 0.568           & 0.113           & 0.813           & 0.919           & \textbf{0.888 } & 0.022            \\ 
%\hline
%\textbf{UF7}          & 0.987           & 0.991           & \textbf{0.990 } & 0.001           & 0.980           & 0.983           & 0.981           & 0.001           & 0.557           & 0.991           & 0.910           & 0.150           & 0.987           & 0.992           & \textbf{0.990 } & 0.001            \\ 
%\hline
%\textbf{Mean}         & \textbf{0.806}  & \textbf{0.935}  & \textbf{0.881}  & \textbf{0.038}  & \textbf{0.808}  & \textbf{0.927}  & \textbf{0.886}  & \textbf{0.032}  & \textbf{0.801}  & \textbf{0.940}  & \textbf{0.882}  & \textbf{0.048}  & \textbf{0.943}  & \textbf{0.964}  & \textbf{0.955}  & \textbf{0.005}   \\
%\hline
%\end{tabular}
%}
%\end{table*}
%


\begin{table*}
\centering
\caption{Statistics HV with three objectives}
\label{tab:StatisticsHV_3obj}
\begin{tabular}{c|c|c|c|c|c|c|c|c|c|c|c|c|c|c|c|c} 
\cline{2-17}
\multicolumn{1}{c}{} & \multicolumn{4}{c|}{\textbf{MOEA/D} }                                 & \multicolumn{4}{c|}{\textbf{NSGA-II} }                                & \multicolumn{4}{c|}{\textbf{R2-MOEA} }                                & \multicolumn{4}{c}{\textbf{VSD-MOEA} }                                \\ 
\cline{2-17}
\multicolumn{1}{c}{} & \textbf{Min}    & \textbf{Max}    & \textbf{Mean}   & \textbf{Std}    & \textbf{Min}    & \textbf{Max}    & \textbf{Mean}   & \textbf{Std}    & \textbf{Min}    & \textbf{Max}    & \textbf{Mean}   & \textbf{Std}    & \textbf{Min}    & \textbf{Max}    & \textbf{Mean}   & \textbf{Std}     \\ 
\hline
\textbf{WFG1}         & 0.958           & 0.969           & 0.966           & 0.002           & 0.925           & 0.945           & 0.935           & 0.005           & 0.968           & 0.979           & 0.975           & 0.002           & 0.979           & 0.984           & \textbf{0.982 } & 0.001            \\ 
\hline
\textbf{WFG2}         & 0.973           & 0.978           & 0.976           & 0.001           & 0.959           & 0.974           & 0.968           & 0.004           & 0.962           & 0.963           & 0.963           & 0.000           & 0.987           & 0.991           & \textbf{0.989 } & 0.001            \\ 
\hline
\textbf{WFG3}         & 0.992           & 0.992           & \textbf{0.992 } & 0.000           & 0.976           & 0.988           & 0.985           & 0.002           & 0.991           & 0.992           & 0.992           & 0.000           & 0.989           & 0.989           & 0.989           & 0.000            \\ 
\hline
\textbf{WFG4}         & 0.864           & 0.865           & 0.865           & 0.000           & 0.854           & 0.883           & 0.868           & 0.007           & 0.903           & 0.905           & 0.904           & 0.000           & 0.919           & 0.921           & \textbf{0.919 } & 0.001            \\ 
\hline
\textbf{WFG5}         & 0.795           & 0.804           & 0.797           & 0.002           & 0.806           & 0.836           & 0.821           & 0.008           & 0.843           & 0.853           & 0.848           & 0.002           & 0.835           & 0.859           & \textbf{0.853 } & 0.006            \\ 
\hline
\textbf{WFG6}         & 0.777           & 0.832           & 0.809           & 0.013           & 0.788           & 0.836           & 0.815           & 0.011           & 0.847           & 0.875           & \textbf{0.857 } & 0.007           & 0.825           & 0.856           & 0.835           & 0.009            \\ 
\hline
\textbf{WFG7}         & 0.864           & 0.865           & 0.865           & 0.000           & 0.858           & 0.889           & 0.875           & 0.008           & 0.901           & 0.905           & 0.904           & 0.001           & 0.918           & 0.920           & \textbf{0.919 } & 0.000            \\ 
\hline
\textbf{WFG8}         & 0.778           & 0.785           & 0.782           & 0.002           & 0.697           & 0.730           & 0.716           & 0.008           & 0.816           & 0.821           & 0.819           & 0.001           & 0.877           & 0.910           & \textbf{0.903 } & 0.008            \\ 
\hline
\textbf{WFG9}         & 0.726           & 0.851           & 0.819           & 0.039           & 0.720           & 0.833           & 0.746           & 0.027           & 0.773           & 0.895           & 0.872           & 0.038           & 0.813           & 0.881           & \textbf{0.874 } & 0.011            \\ 
\hline
\textbf{DTLZ1}        & 0.950           & 0.950           & 0.950           & 0.000           & 0.935           & 0.950           & 0.943           & 0.004           & 0.939           & 0.943           & 0.941           & 0.001           & 0.963           & 0.966           & \textbf{0.964 } & 0.001            \\ 
\hline
\textbf{DTLZ2}        & 0.899           & 0.899           & 0.899           & 0.000           & 0.871           & 0.901           & 0.886           & 0.007           & 0.913           & 0.916           & 0.915           & 0.001           & 0.929           & 0.930           & \textbf{0.930 } & 0.000            \\ 
\hline
\textbf{DTLZ3}        & 0.899           & 0.899           & 0.899           & 0.000           & 0.876           & 0.901           & 0.890           & 0.006           & 0.914           & 0.916           & 0.915           & 0.000           & 0.929           & 0.930           & \textbf{0.930 } & 0.000            \\ 
\hline
\textbf{DTLZ4}        & 0.151           & 0.899           & 0.813           & 0.238           & 0.871           & 0.904           & 0.888           & 0.007           & 0.151           & 0.916           & 0.675           & 0.298           & 0.928           & 0.930           & \textbf{0.930 } & 0.001            \\ 
\hline
\textbf{DTLZ5}        & 0.978           & 0.978           & 0.978           & 0.000           & 0.982           & 0.984           & 0.983           & 0.001           & 0.985           & 0.986           & \textbf{0.986 } & 0.000           & 0.986           & 0.986           & \textbf{0.986 } & 0.000            \\ 
\hline
\textbf{DTLZ6}        & 0.310           & 0.889           & 0.591           & 0.142           & 0.183           & 0.382           & 0.243           & 0.056           & 0.400           & 0.946           & 0.672           & 0.143           & 0.986           & 0.986           & \textbf{0.986 } & 0.000            \\ 
\hline
\textbf{DTLZ7}        & 0.914           & 0.914           & 0.914           & 0.000           & 0.907           & 0.935           & 0.924           & 0.006           & 0.837           & 0.893           & 0.860           & 0.014           & 0.962           & 0.966           & \textbf{0.964 } & 0.001            \\ 
\hline
\textbf{UF8}          & 0.151           & 0.830           & 0.773           & 0.107           & 0.324           & 0.646           & 0.463           & 0.069           & 0.578           & 0.917           & 0.898           & 0.057           & 0.905           & 0.925           & \textbf{0.918 } & 0.006            \\ 
\hline
\textbf{UF9}          & 0.753           & 0.916           & 0.846           & 0.067           & 0.368           & 0.782           & 0.728           & 0.096           & 0.778           & 0.954           & 0.844           & 0.079           & 0.937           & 0.975           & \textbf{0.963 } & 0.010            \\ 
\hline
\textbf{UF10}         & 0.145           & 0.555           & 0.341           & 0.162           & 0.060           & 0.391           & 0.242           & 0.067           & 0.143           & 0.578           & 0.413           & 0.166           & 0.469           & 0.762           & \textbf{0.627 } & 0.086            \\ 
\hline
\textbf{Mean}         & \textbf{0.730}  & \textbf{0.877}  & \textbf{0.835}  & \textbf{0.041}  & \textbf{0.735}  & \textbf{0.826}  & \textbf{0.785}  & \textbf{0.021}  & \textbf{0.771}  & \textbf{0.903}  & \textbf{0.855}  & \textbf{0.043}  & \textbf{0.902}  & \textbf{0.930}  & \textbf{0.919}  & \textbf{0.007}   \\
\hline
\end{tabular}
\end{table*}

%% Please add the following required packages to your document preamble:
% \usepackage{graphicx}
\begin{table*}[t]
\centering
\caption{Statistics IGD+ with two objectives}
\label{tab:StatisticsIGDP_2obj}
%\resizebox{\textwidth}{!}{%
\begin{tabular}{cc|c|c|c|c|c|c|c|c|c|c|c|c|c|c|c}
\cline{2-17}
 & \multicolumn{4}{c|}{\textbf{MOEA/D}} & \multicolumn{4}{c|}{\textbf{NSGA-II}} & \multicolumn{4}{c|}{\textbf{R2-EMOA}} & \multicolumn{4}{c}{\textbf{VSD-MOEA}} \\ \cline{2-17} 
 & \textbf{Min} & \textbf{Max} & \textbf{Mean} & \textbf{Std} & \textbf{Min} & \textbf{Max} & \textbf{Mean} & \textbf{Std} & \textbf{Min} & \textbf{Max} & \textbf{Mean} & \textbf{Std} & \textbf{Min} & \textbf{Max} & \textbf{Mean} & \textbf{Std} \\ \hline
\multicolumn{1}{c|}{\textbf{WFG1}} & 0.006 & 0.015 & 0.008 & 0.002 & 0.006 & 0.014 & 0.008 & 0.002 & 0.006 & 0.061 & 0.013 & 0.014 & 0.006 & 0.019 & 0.008 & 0.003 \\ \hline
\multicolumn{1}{c|}{\textbf{WFG2}} & 0.006 & 0.055 & 0.052 & 0.011 & 0.003 & 0.053 & 0.040 & 0.022 & 0.053 & 0.055 & 0.054 & 0.000 & 0.003 & 0.003 & 0.003 & 0.000 \\ \hline
\multicolumn{1}{c|}{\textbf{WFG3}} & 0.008 & 0.008 & 0.008 & 0.000 & 0.011 & 0.013 & 0.012 & 0.000 & 0.008 & 0.009 & 0.008 & 0.000 & 0.007 & 0.007 & 0.007 & 0.000 \\ \hline
\multicolumn{1}{c|}{\textbf{WFG4}} & 0.007 & 0.007 & 0.007 & 0.000 & 0.007 & 0.010 & 0.008 & 0.001 & 0.005 & 0.005 & 0.005 & 0.000 & 0.006 & 0.006 & 0.006 & 0.000 \\ \hline
\multicolumn{1}{c|}{\textbf{WFG5}} & 0.060 & 0.069 & 0.065 & 0.002 & 0.060 & 0.068 & 0.066 & 0.002 & 0.064 & 0.066 & 0.065 & 0.000 & 0.038 & 0.057 & 0.047 & 0.006 \\ \hline
\multicolumn{1}{c|}{\textbf{WFG6}} & 0.034 & 0.073 & 0.050 & 0.010 & 0.034 & 0.064 & 0.051 & 0.007 & 0.034 & 0.076 & 0.053 & 0.010 & 0.068 & 0.088 & 0.081 & 0.004 \\ \hline
\multicolumn{1}{c|}{\textbf{WFG7}} & 0.007 & 0.007 & 0.007 & 0.000 & 0.008 & 0.010 & 0.009 & 0.000 & 0.005 & 0.006 & 0.005 & 0.000 & 0.006 & 0.006 & 0.006 & 0.000 \\ \hline
\multicolumn{1}{c|}{\textbf{WFG8}} & 0.103 & 0.120 & 0.112 & 0.005 & 0.116 & 0.139 & 0.125 & 0.005 & 0.103 & 0.120 & 0.110 & 0.004 & 0.026 & 0.099 & 0.043 & 0.025 \\ \hline
\multicolumn{1}{c|}{\textbf{WFG9}} & 0.011 & 0.125 & 0.067 & 0.053 & 0.014 & 0.127 & 0.101 & 0.046 & 0.009 & 0.125 & 0.067 & 0.053 & 0.009 & 0.014 & 0.011 & 0.001 \\ \hline
\multicolumn{1}{c|}{\textbf{DTLZ1}} & 0.001 & 0.001 & 0.001 & 0.000 & 0.002 & 0.002 & 0.002 & 0.000 & 0.001 & 0.001 & 0.001 & 0.000 & 0.001 & 0.001 & 0.001 & 0.000 \\ \hline
\multicolumn{1}{c|}{\textbf{DTLZ2}} & 0.002 & 0.002 & 0.002 & 0.000 & 0.002 & 0.003 & 0.003 & 0.000 & 0.002 & 0.002 & 0.002 & 0.000 & 0.002 & 0.002 & 0.002 & 0.000 \\ \hline
\multicolumn{1}{c|}{\textbf{DTLZ3}} & 0.002 & 0.002 & 0.002 & 0.000 & 0.002 & 0.003 & 0.002 & 0.000 & 0.002 & 0.002 & 0.002 & 0.000 & 0.002 & 0.002 & 0.002 & 0.000 \\ \hline
\multicolumn{1}{c|}{\textbf{DTLZ4}} & 0.002 & 0.363 & 0.105 & 0.163 & 0.002 & 0.363 & 0.064 & 0.136 & 0.002 & 0.363 & 0.167 & 0.180 & 0.002 & 0.002 & 0.002 & 0.000 \\ \hline
\multicolumn{1}{c|}{\textbf{DTLZ5}} & 0.002 & 0.002 & 0.002 & 0.000 & 0.002 & 0.003 & 0.003 & 0.000 & 0.002 & 0.002 & 0.002 & 0.000 & 0.002 & 0.002 & 0.002 & 0.000 \\ \hline
\multicolumn{1}{c|}{\textbf{DTLZ6}} & 0.022 & 0.149 & 0.076 & 0.027 & 0.126 & 0.315 & 0.205 & 0.036 & 0.019 & 0.128 & 0.078 & 0.027 & 0.002 & 0.002 & 0.002 & 0.000 \\ \hline
\multicolumn{1}{c|}{\textbf{DTLZ7}} & 0.003 & 0.003 & 0.003 & 0.000 & 0.002 & 0.003 & 0.003 & 0.000 & 0.002 & 0.002 & 0.002 & 0.000 & 0.003 & 0.003 & 0.003 & 0.000 \\ \hline
\multicolumn{1}{c|}{\textbf{UF1}} & 0.004 & 0.004 & 0.004 & 0.000 & 0.005 & 0.006 & 0.006 & 0.000 & 0.003 & 0.005 & 0.004 & 0.001 & 0.003 & 0.003 & 0.003 & 0.000 \\ \hline
\multicolumn{1}{c|}{\textbf{UF2}} & 0.003 & 0.005 & 0.004 & 0.000 & 0.008 & 0.010 & 0.010 & 0.000 & 0.004 & 0.006 & 0.005 & 0.001 & 0.004 & 0.007 & 0.005 & 0.001 \\ \hline
\multicolumn{1}{c|}{\textbf{UF3}} & 0.141 & 0.237 & 0.180 & 0.022 & 0.052 & 0.127 & 0.084 & 0.020 & 0.119 & 0.210 & 0.183 & 0.021 & 0.038 & 0.095 & 0.057 & 0.013 \\ \hline
\multicolumn{1}{c|}{\textbf{UF4}} & 0.024 & 0.031 & 0.026 & 0.001 & 0.027 & 0.039 & 0.033 & 0.003 & 0.019 & 0.023 & 0.021 & 0.001 & 0.020 & 0.024 & 0.022 & 0.001 \\ \hline
\multicolumn{1}{c|}{\textbf{UF5}} & 0.079 & 0.593 & 0.265 & 0.120 & 0.091 & 0.254 & 0.142 & 0.033 & 0.079 & 0.521 & 0.215 & 0.131 & 0.088 & 0.154 & 0.132 & 0.014 \\ \hline
\multicolumn{1}{c|}{\textbf{UF6}} & 0.066 & 0.529 & 0.380 & 0.108 & 0.037 & 0.542 & 0.193 & 0.114 & 0.064 & 0.432 & 0.266 & 0.103 & 0.021 & 0.065 & 0.038 & 0.011 \\ \hline
\multicolumn{1}{c|}{\textbf{UF7}} & 0.003 & 0.005 & 0.004 & 0.000 & 0.007 & 0.008 & 0.007 & 0.000 & 0.003 & 0.242 & 0.046 & 0.082 & 0.003 & 0.009 & 0.004 & 0.001 \\ \hline
\multicolumn{1}{c|}{\textbf{Mean}} & 0.026 & 0.105 & 0.062 & 0.023 & 0.027 & 0.095 & 0.051 & 0.019 & 0.026 & 0.107 & 0.060 & 0.027 & 0.016 & 0.029 & 0.021 & 0.003 \\ \hline
\end{tabular}%
%}
\end{table*}



%%% Please add the following required packages to your document preamble:
%%% \usepackage{graphicx}
%%
%%\begin{table*}[t]
%%\caption{Statistics IGD+ with two objectives}
%%\label{tab:StatisticsIGDP_2obj}
%%%\resizebox{\textwidth}{!}{%
%%\begin{tabular}{c|c|c|c|c|c|c|c|c|c|c|c|c|c|c|c|c|}
%%\cline{2-17}
%% & \multicolumn{4}{c|}{\textbf{MOEA/D}} & \multicolumn{4}{c|}{\textbf{NSGA-II}} & \multicolumn{4}{c|}{\textbf{R2-MOEA}} & \multicolumn{4}{c|}{\textbf{VSD-MOEA}} \\ \cline{2-17} 
%% & \textbf{Min} & \textbf{Max} & \textbf{Mean} & \textbf{Std} & \textbf{Min} & \textbf{Max} & \textbf{Mean} & \textbf{Std} & \textbf{Min} & \textbf{Max} & \textbf{Mean} & \textbf{Std} & \textbf{Min} & \textbf{Max} & \textbf{Mean} & \textbf{Std} \\ \hline
%%\multicolumn{1}{|c|}{\textbf{WFG1}} & 0.006 & 0.015 & 0.008 & 0.002 & 0.006 & 0.014 & 0.008 & 0.002 & 0.006 & 0.061 & 0.013 & 0.014 & 0.006 & 0.025 & 0.007 & 0.003 \\ \hline
%%\multicolumn{1}{|c|}{\textbf{WFG2}} & 0.006 & 0.055 & 0.052 & 0.011 & 0.003 & 0.053 & 0.040 & 0.022 & 0.053 & 0.055 & 0.054 & 0.000 & 0.003 & 0.003 & 0.003 & 0.000 \\ \hline
%%\multicolumn{1}{|c|}{\textbf{WFG3}} & 0.008 & 0.008 & 0.008 & 0.000 & 0.011 & 0.013 & 0.012 & 0.000 & 0.008 & 0.009 & 0.008 & 0.000 & 0.007 & 0.007 & 0.007 & 0.000 \\ \hline
%%\multicolumn{1}{|c|}{\textbf{WFG4}} & 0.007 & 0.007 & 0.007 & 0.000 & 0.007 & 0.010 & 0.008 & 0.001 & 0.005 & 0.005 & 0.005 & 0.000 & 0.006 & 0.006 & 0.006 & 0.000 \\ \hline
%%\multicolumn{1}{|c|}{\textbf{WFG5}} & 0.060 & 0.069 & 0.065 & 0.002 & 0.060 & 0.068 & 0.066 & 0.002 & 0.064 & 0.066 & 0.065 & 0.000 & 0.033 & 0.053 & 0.040 & 0.005 \\ \hline
%%\multicolumn{1}{|c|}{\textbf{WFG6}} & 0.034 & 0.073 & 0.050 & 0.010 & 0.034 & 0.064 & 0.051 & 0.007 & 0.034 & 0.076 & 0.053 & 0.010 & 0.068 & 0.090 & 0.081 & 0.005 \\ \hline
%%\multicolumn{1}{|c|}{\textbf{WFG7}} & 0.007 & 0.007 & 0.007 & 0.000 & 0.008 & 0.010 & 0.009 & 0.000 & 0.005 & 0.006 & 0.005 & 0.000 & 0.006 & 0.006 & 0.006 & 0.000 \\ \hline
%%\multicolumn{1}{|c|}{\textbf{WFG8}} & 0.103 & 0.120 & 0.112 & 0.005 & 0.116 & 0.139 & 0.125 & 0.005 & 0.103 & 0.120 & 0.110 & 0.004 & 0.024 & 0.035 & 0.029 & 0.003 \\ \hline
%%\multicolumn{1}{|c|}{\textbf{WFG9}} & 0.011 & 0.125 & 0.067 & 0.053 & 0.014 & 0.127 & 0.101 & 0.046 & 0.009 & 0.125 & 0.067 & 0.053 & 0.009 & 0.015 & 0.011 & 0.001 \\ \hline
%%\multicolumn{1}{|c|}{\textbf{DTLZ1}} & 0.001 & 0.001 & 0.001 & 0.000 & 0.002 & 0.002 & 0.002 & 0.000 & 0.001 & 0.001 & 0.001 & 0.000 & 0.001 & 0.001 & 0.001 & 0.000 \\ \hline
%%\multicolumn{1}{|c|}{\textbf{DTLZ2}} & 0.002 & 0.002 & 0.002 & 0.000 & 0.002 & 0.003 & 0.003 & 0.000 & 0.002 & 0.002 & 0.002 & 0.000 & 0.002 & 0.002 & 0.002 & 0.000 \\ \hline
%%\multicolumn{1}{|c|}{\textbf{DTLZ3}} & 0.002 & 0.002 & 0.002 & 0.000 & 0.002 & 0.003 & 0.002 & 0.000 & 0.002 & 0.002 & 0.002 & 0.000 & 0.002 & 0.002 & 0.002 & 0.000 \\ \hline
%%\multicolumn{1}{|c|}{\textbf{DTLZ4}} & 0.002 & 0.363 & 0.105 & 0.163 & 0.002 & 0.363 & 0.064 & 0.136 & 0.002 & 0.363 & 0.167 & 0.180 & 0.002 & 0.002 & 0.002 & 0.000 \\ \hline
%%\multicolumn{1}{|c|}{\textbf{DTLZ5}} & 0.002 & 0.002 & 0.002 & 0.000 & 0.002 & 0.003 & 0.003 & 0.000 & 0.002 & 0.002 & 0.002 & 0.000 & 0.002 & 0.002 & 0.002 & 0.000 \\ \hline
%%\multicolumn{1}{|c|}{\textbf{DTLZ6}} & 0.022 & 0.149 & 0.076 & 0.027 & 0.126 & 0.315 & 0.205 & 0.036 & 0.019 & 0.128 & 0.078 & 0.027 & 0.002 & 0.002 & 0.002 & 0.000 \\ \hline
%%\multicolumn{1}{|c|}{\textbf{DTLZ7}} & 0.003 & 0.003 & 0.003 & 0.000 & 0.002 & 0.003 & 0.003 & 0.000 & 0.002 & 0.002 & 0.002 & 0.000 & 0.003 & 0.003 & 0.003 & 0.000 \\ \hline
%%\multicolumn{1}{|c|}{\textbf{UF1}} & 0.004 & 0.004 & 0.004 & 0.000 & 0.005 & 0.006 & 0.006 & 0.000 & 0.003 & 0.005 & 0.004 & 0.001 & 0.003 & 0.003 & 0.003 & 0.000 \\ \hline
%%\multicolumn{1}{|c|}{\textbf{UF2}} & 0.003 & 0.005 & 0.004 & 0.000 & 0.008 & 0.010 & 0.010 & 0.000 & 0.004 & 0.006 & 0.005 & 0.001 & 0.005 & 0.008 & 0.006 & 0.001 \\ \hline
%%\multicolumn{1}{|c|}{\textbf{UF3}} & 0.141 & 0.237 & 0.180 & 0.022 & 0.052 & 0.127 & 0.084 & 0.020 & 0.119 & 0.210 & 0.183 & 0.021 & 0.043 & 0.077 & 0.052 & 0.006 \\ \hline
%%\multicolumn{1}{|c|}{\textbf{UF4}} & 0.024 & 0.031 & 0.026 & 0.001 & 0.027 & 0.039 & 0.033 & 0.003 & 0.019 & 0.023 & 0.021 & 0.001 & 0.021 & 0.024 & 0.022 & 0.001 \\ \hline
%%\multicolumn{1}{|c|}{\textbf{UF5}} & 0.079 & 0.593 & 0.265 & 0.120 & 0.091 & 0.254 & 0.142 & 0.033 & 0.079 & 0.521 & 0.215 & 0.131 & 0.083 & 0.145 & 0.118 & 0.015 \\ \hline
%%\multicolumn{1}{|c|}{\textbf{UF6}} & 0.066 & 0.529 & 0.380 & 0.108 & 0.037 & 0.542 & 0.193 & 0.114 & 0.064 & 0.432 & 0.266 & 0.103 & 0.019 & 0.034 & 0.026 & 0.005 \\ \hline
%%\multicolumn{1}{|c|}{\textbf{UF7}} & 0.003 & 0.005 & 0.004 & 0.000 & 0.007 & 0.008 & 0.007 & 0.000 & 0.003 & 0.242 & 0.046 & 0.082 & 0.003 & 0.005 & 0.004 & 0.000 \\ \hline
%%\multicolumn{1}{|c|}{\textbf{Mean}} & \textbf{0.026} & \textbf{0.105} & \textbf{0.062} & \textbf{0.023} & \textbf{0.027} & \textbf{0.095} & \textbf{0.051} & \textbf{0.019} & \textbf{0.026} & \textbf{0.107} & \textbf{0.060} & \textbf{0.027} & \textbf{0.015} & \textbf{0.024} & \textbf{0.019} & \textbf{0.002} \\ \hline
%%\end{tabular}%
%%%}
%%\end{table*}
%%

%% Please add the following required packages to your document preamble:
% \usepackage{graphicx}
\begin{table*}[t]
\caption{Summary of the IGD+ results attained for problems with three objectives}
\label{tab:StatisticsIGDP_3obj}
\centering
%\resizebox{\textwidth}{!}{%
\begin{tabular}{cc|c|c|c|c|c|c|c|c|c|c|c|c|c|c|c}
\cline{2-17}
\textbf{}                           & \multicolumn{4}{c|}{\textbf{MOEA/D}}                       & \multicolumn{4}{c|}{\textbf{NSGA-II}}                      & \multicolumn{4}{c|}{\textbf{R2-EMOA}}                             & \multicolumn{4}{c}{\textbf{VSD-MOEA}}                            \\ \cline{2-17} 
                                    & \textbf{Min} & \textbf{Max} & \textbf{Mean} & \textbf{Std} & \textbf{Min} & \textbf{Max} & \textbf{Mean} & \textbf{Std} & \textbf{Min}   & \textbf{Max}   & \textbf{Mean}  & \textbf{Std}   & \textbf{Min}   & \textbf{Max}   & \textbf{Mean}  & \textbf{Std}   \\ \hline
\multicolumn{1}{c|}{\textbf{WFG1}}  & 0.080        & 0.100        & 0.090         & 0.005        & 0.142        & 0.179        & 0.160         & 0.010        & 0.058          & 0.098          & 0.079          & 0.010          & \textbf{0.049} & \textbf{0.070} & \textbf{0.058} & \textbf{0.006} \\ \hline
\multicolumn{1}{c|}{\textbf{WFG2}}  & 0.057        & 0.068        & 0.063         & 0.002        & 0.073        & 0.133        & 0.097         & 0.014        & 0.102          & 0.104          & 0.103          & 0.000          & \textbf{0.031} & \textbf{0.048} & \textbf{0.037} & \textbf{0.004} \\ \hline
\multicolumn{1}{c|}{\textbf{WFG3}}  & 0.023        & 0.023        & 0.023         & 0.000        & 0.031        & 0.061        & 0.039         & 0.005        & \textbf{0.022} & \textbf{0.023} & \textbf{0.022} & \textbf{0.000} & 0.033          & 0.033          & 0.033          & 0.000          \\ \hline
\multicolumn{1}{c|}{\textbf{WFG4}}  & 0.127        & 0.127        & 0.127         & 0.000        & 0.121        & 0.144        & 0.132         & 0.005        & 0.095          & 0.098          & 0.097          & 0.001          & \textbf{0.090} & \textbf{0.094} & \textbf{0.093} & \textbf{0.001} \\ \hline
\multicolumn{1}{c|}{\textbf{WFG5}}  & 0.177        & 0.184        & 0.181         & 0.002        & 0.160        & 0.186        & 0.170         & 0.005        & 0.147          & 0.158          & 0.153          & 0.003          & \textbf{0.140} & \textbf{0.150} & \textbf{0.146} & \textbf{0.003} \\ \hline
\multicolumn{1}{c|}{\textbf{WFG6}}  & 0.155        & 0.205        & 0.175         & 0.012        & 0.159        & 0.196        & 0.177         & 0.009        & \textbf{0.122} & \textbf{0.151} & \textbf{0.140} & \textbf{0.007} & 0.156          & 0.173          & 0.166          & 0.005          \\ \hline
\multicolumn{1}{c|}{\textbf{WFG7}}  & 0.127        & 0.127        & 0.127         & 0.000        & 0.113        & 0.138        & 0.123         & 0.007        & 0.094          & 0.102          & 0.097          & 0.001          & \textbf{0.092} & \textbf{0.094} & \textbf{0.094} & \textbf{0.001} \\ \hline
\multicolumn{1}{c|}{\textbf{WFG8}}  & 0.189        & 0.194        & 0.192         & 0.001        & 0.244        & 0.274        & 0.256         & 0.008        & 0.161          & 0.166          & 0.163          & 0.001          & \textbf{0.099} & \textbf{0.154} & \textbf{0.109} & \textbf{0.015} \\ \hline
\multicolumn{1}{c|}{\textbf{WFG9}}  & 0.130        & 0.240        & 0.154         & 0.036        & 0.138        & 0.246        & 0.224         & 0.025        & 0.099          & 0.211          & 0.119          & 0.037          & \textbf{0.099} & \textbf{0.210} & \textbf{0.118} & \textbf{0.036} \\ \hline
\multicolumn{1}{c|}{\textbf{DTLZ1}} & 0.014        & 0.014        & 0.014         & 0.000        & 0.017        & 0.020        & 0.018         & 0.001        & \textbf{0.013} & \textbf{0.014} & \textbf{0.014} & \textbf{0.000} & 0.014          & 0.014          & 0.014          & 0.000          \\ \hline
\multicolumn{1}{c|}{\textbf{DTLZ2}} & 0.027        & 0.027        & 0.027         & 0.000        & 0.030        & 0.036        & 0.032         & 0.001        & \textbf{0.023} & \textbf{0.024} & \textbf{0.023} & \textbf{0.000} & 0.024          & 0.025          & 0.024          & 0.000          \\ \hline
\multicolumn{1}{c|}{\textbf{DTLZ3}} & 0.027        & 0.027        & 0.027         & 0.000        & 0.027        & 0.032        & 0.030         & 0.001        & \textbf{0.023} & \textbf{0.023} & \textbf{0.023} & \textbf{0.000} & 0.024          & 0.025          & 0.024          & 0.000          \\ \hline
\multicolumn{1}{c|}{\textbf{DTLZ4}} & 0.027        & 0.595        & 0.092         & 0.181        & 0.028        & 0.036        & 0.032         & 0.001        & 0.023          & 0.595          & 0.190          & 0.225          & \textbf{0.024} & \textbf{0.025} & \textbf{0.024} & \textbf{0.000} \\ \hline
\multicolumn{1}{c|}{\textbf{DTLZ5}} & 0.003        & 0.003        & 0.003         & 0.000        & 0.003        & 0.003        & 0.003         & 0.000        & 0.002          & 0.002          & 0.002          & 0.000          & \textbf{0.002} & \textbf{0.002} & \textbf{0.002} & \textbf{0.000} \\ \hline
\multicolumn{1}{c|}{\textbf{DTLZ6}} & 0.022        & 0.163        & 0.087         & 0.032        & 0.126        & 0.224        & 0.187         & 0.027        & 0.003          & 0.136          & 0.069          & 0.033          & \textbf{0.002} & \textbf{0.002} & \textbf{0.002} & \textbf{0.000} \\ \hline
\multicolumn{1}{c|}{\textbf{DTLZ7}} & 0.045        & 0.045        & 0.045         & 0.000        & 0.038        & 0.052        & 0.044         & 0.003        & 0.060          & 0.087          & 0.079          & 0.008          & \textbf{0.027} & \textbf{0.029} & \textbf{0.028} & \textbf{0.000} \\ \hline
\multicolumn{1}{c|}{\textbf{UF8}}   & 0.048        & 0.365        & 0.069         & 0.051        & 0.093        & 0.220        & 0.178         & 0.031        & 0.027          & 0.159          & 0.033          & 0.022          & \textbf{0.025} & \textbf{0.034} & \textbf{0.029} & \textbf{0.002} \\ \hline
\multicolumn{1}{c|}{\textbf{UF9}}   & 0.041        & 0.151        & 0.086         & 0.049        & 0.106        & 0.314        & 0.139         & 0.049        & 0.025          & 0.137          & 0.094          & 0.053          & \textbf{0.022} & \textbf{0.028} & \textbf{0.024} & \textbf{0.001} \\ \hline
\multicolumn{1}{c|}{\textbf{UF10}}  & 0.163        & 0.565        & 0.294         & 0.125        & 0.198        & 0.658        & 0.261         & 0.080        & 0.159          & 0.553          & 0.257          & 0.131          & \textbf{0.070} & \textbf{0.187} & \textbf{0.103} & \textbf{0.026} \\ \hline
\multicolumn{1}{c|}{\textbf{Mean}}  & 0.078        & 0.170        & 0.099         & 0.026        & 0.097        & 0.166        & 0.121         & 0.015        & 0.066          & 0.150          & 0.093          & 0.028          & 0.054          & 0.074          & 0.059          & 0.005          \\ \hline
\end{tabular}%
%}
\end{table*}

%% Please add the following required packages to your document preamble:
%% \usepackage{graphicx}
%\begin{table*}[t]
%\caption{Statistics IGD+ with three objectives}
%\label{tab:StatisticsIGDP_3obj}
%%\resizebox{\textwidth}{!}{%
%\begin{tabular}{c|c|c|c|c|c|c|c|c|c|c|c|c|c|c|c|c|}
%\cline{2-17}
% & \multicolumn{4}{c|}{\textbf{MOEA/D}} & \multicolumn{4}{c|}{\textbf{NSGA-II}} & \multicolumn{4}{c|}{\textbf{R2-MOEA}} & \multicolumn{4}{c|}{\textbf{VSD-MOEA}} \\ \cline{2-17} 
% & \textbf{Min} & \textbf{Max} & \textbf{Mean} & \textbf{Std} & \textbf{Min} & \textbf{Max} & \textbf{Mean} & \textbf{Std} & \textbf{Min} & \textbf{Max} & \textbf{Mean} & \textbf{Std} & \textbf{Min} & \textbf{Max} & \textbf{Mean} & \textbf{Std} \\ \hline
%\multicolumn{1}{|c|}{\textbf{WFG1}} & 0.080 & 0.100 & 0.090 & 0.005 & 0.142 & 0.179 & 0.160 & 0.010 & 0.058 & 0.098 & 0.079 & 0.010 & 0.050 & 0.066 & 0.056 & 0.004 \\ \hline
%\multicolumn{1}{|c|}{\textbf{WFG2}} & 0.057 & 0.068 & 0.063 & 0.002 & 0.073 & 0.133 & 0.097 & 0.014 & 0.102 & 0.104 & 0.103 & 0.000 & 0.031 & 0.044 & 0.038 & 0.003 \\ \hline
%\multicolumn{1}{|c|}{\textbf{WFG3}} & 0.023 & 0.023 & 0.023 & 0.000 & 0.031 & 0.061 & 0.039 & 0.005 & 0.022 & 0.023 & 0.022 & 0.000 & 0.033 & 0.033 & 0.033 & 0.000 \\ \hline
%\multicolumn{1}{|c|}{\textbf{WFG4}} & 0.127 & 0.127 & 0.127 & 0.000 & 0.121 & 0.144 & 0.132 & 0.005 & 0.095 & 0.098 & 0.097 & 0.001 & 0.091 & 0.094 & 0.093 & 0.001 \\ \hline
%\multicolumn{1}{|c|}{\textbf{WFG5}} & 0.177 & 0.184 & 0.181 & 0.002 & 0.160 & 0.186 & 0.170 & 0.005 & 0.147 & 0.158 & 0.153 & 0.003 & 0.143 & 0.155 & 0.147 & 0.002 \\ \hline
%\multicolumn{1}{|c|}{\textbf{WFG6}} & 0.155 & 0.205 & 0.175 & 0.012 & 0.159 & 0.196 & 0.177 & 0.009 & 0.122 & 0.151 & 0.140 & 0.007 & 0.143 & 0.173 & 0.163 & 0.008 \\ \hline
%\multicolumn{1}{|c|}{\textbf{WFG7}} & 0.127 & 0.127 & 0.127 & 0.000 & 0.113 & 0.138 & 0.123 & 0.007 & 0.094 & 0.102 & 0.097 & 0.001 & 0.092 & 0.094 & 0.093 & 0.001 \\ \hline
%\multicolumn{1}{|c|}{\textbf{WFG8}} & 0.189 & 0.194 & 0.192 & 0.001 & 0.244 & 0.274 & 0.256 & 0.008 & 0.161 & 0.166 & 0.163 & 0.001 & 0.101 & 0.121 & 0.106 & 0.005 \\ \hline
%\multicolumn{1}{|c|}{\textbf{WFG9}} & 0.130 & 0.240 & 0.154 & 0.036 & 0.138 & 0.246 & 0.224 & 0.025 & 0.099 & 0.211 & 0.119 & 0.037 & 0.101 & 0.162 & 0.106 & 0.010 \\ \hline
%\multicolumn{1}{|c|}{\textbf{DTLZ1}} & 0.014 & 0.014 & 0.014 & 0.000 & 0.017 & 0.020 & 0.018 & 0.001 & 0.013 & 0.014 & 0.014 & 0.000 & 0.014 & 0.014 & 0.014 & 0.000 \\ \hline
%\multicolumn{1}{|c|}{\textbf{DTLZ2}} & 0.027 & 0.027 & 0.027 & 0.000 & 0.030 & 0.036 & 0.032 & 0.001 & 0.023 & 0.024 & 0.023 & 0.000 & 0.024 & 0.025 & 0.024 & 0.000 \\ \hline
%\multicolumn{1}{|c|}{\textbf{DTLZ3}} & 0.027 & 0.027 & 0.027 & 0.000 & 0.027 & 0.032 & 0.030 & 0.001 & 0.023 & 0.023 & 0.023 & 0.000 & 0.024 & 0.025 & 0.024 & 0.000 \\ \hline
%\multicolumn{1}{|c|}{\textbf{DTLZ4}} & 0.027 & 0.595 & 0.092 & 0.181 & 0.028 & 0.036 & 0.032 & 0.001 & 0.023 & 0.595 & 0.190 & 0.225 & 0.024 & 0.025 & 0.024 & 0.000 \\ \hline
%\multicolumn{1}{|c|}{\textbf{DTLZ5}} & 0.003 & 0.003 & 0.003 & 0.000 & 0.003 & 0.003 & 0.003 & 0.000 & 0.002 & 0.002 & 0.002 & 0.000 & 0.002 & 0.002 & 0.002 & 0.000 \\ \hline
%\multicolumn{1}{|c|}{\textbf{DTLZ6}} & 0.022 & 0.163 & 0.087 & 0.032 & 0.126 & 0.224 & 0.187 & 0.027 & 0.003 & 0.136 & 0.069 & 0.033 & 0.002 & 0.002 & 0.002 & 0.000 \\ \hline
%\multicolumn{1}{|c|}{\textbf{DTLZ7}} & 0.045 & 0.045 & 0.045 & 0.000 & 0.038 & 0.052 & 0.044 & 0.003 & 0.060 & 0.087 & 0.079 & 0.008 & 0.027 & 0.029 & 0.028 & 0.000 \\ \hline
%\multicolumn{1}{|c|}{\textbf{UF8}} & 0.048 & 0.365 & 0.069 & 0.051 & 0.093 & 0.220 & 0.178 & 0.031 & 0.027 & 0.159 & 0.033 & 0.022 & 0.026 & 0.034 & 0.029 & 0.002 \\ \hline
%\multicolumn{1}{|c|}{\textbf{UF9}} & 0.041 & 0.151 & 0.086 & 0.049 & 0.106 & 0.314 & 0.139 & 0.049 & 0.025 & 0.137 & 0.094 & 0.053 & 0.022 & 0.030 & 0.025 & 0.002 \\ \hline
%\multicolumn{1}{|c|}{\textbf{UF10}} & 0.163 & 0.565 & 0.294 & 0.125 & 0.198 & 0.658 & 0.261 & 0.080 & 0.159 & 0.553 & 0.257 & 0.131 & 0.061 & 0.168 & 0.099 & 0.026 \\ \hline
%\multicolumn{1}{|c|}{\textbf{Mean}} & \textbf{0.078} & \textbf{0.170} & \textbf{0.099} & \textbf{0.026} & \textbf{0.097} & \textbf{0.166} & \textbf{0.121} & \textbf{0.015} & \textbf{0.066} & \textbf{0.150} & \textbf{0.093} & \textbf{0.028} & \textbf{0.053} & \textbf{0.068} & \textbf{0.058} & \textbf{0.003} \\ \hline
%\end{tabular}%
%%}
%\end{table*}


% Please add the following required packages to your document preamble:
% \usepackage{graphicx}
\begin{table}[t]
\centering
\caption{Statistical Tests of HV with two objectives}
\label{tab:Tests_HV_2obj}

%\resizebox{\textwidth}{!}{%
\begin{tabular}{c c|c|c|c}
\cline{2-5}
                                        & \textbf{$\uparrow$} & \textbf{$\downarrow$} & \textbf{$\leftrightarrow$} & \textbf{Diff} \\ \hline
\multicolumn{1}{c|}{\textbf{MOEA/D}}   & 24                  & 36                    & 9                          & 1.615         \\ \hline
\multicolumn{1}{c|}{\textbf{NSGA-II}}  & 13                  & 49                    & 7                          & 1.496         \\ \hline
\multicolumn{1}{c|}{\textbf{R2-EMOA}}  & 34                  & 21                    & 14                         & 1.597         \\ \hline
\multicolumn{1}{c|}{\textbf{VSD-MOEA}} & 50                  & 15                    & 4                          & 0.059         \\ \hline
\end{tabular}%
%}
\end{table}


%% Please add the following required packages to your document preamble:
%% \usepackage{graphicx}
%\begin{table*}[t]
%\caption{Statistical Tests of HV with two objectives}
%\label{tab:Tests_HV_2obj}
%\centering
%%\resizebox{\textwidth}{!}{%
%\begin{tabular}{c|c|c|c|c|}
%\cline{2-5}
%                                        & \textbf{$\uparrow$} & \textbf{$\downarrow$} & \textbf{$\leftrightarrow$} & \textbf{Diff} \\ \hline
%\multicolumn{1}{|c|}{\textbf{MOEA/D}}   & 0                   & 40                    & 105                        & 14.000        \\ \hline
%\multicolumn{1}{|c|}{\textbf{NSGA-II}}  & 3                   & 20                    & 126                        & 13.000        \\ \hline
%\multicolumn{1}{|c|}{\textbf{R2-EMOA}}  & 5                   & 94                    & 49                         & 16.000        \\ \hline
%\multicolumn{1}{|c|}{\textbf{VSD-MOEA}} & 2                   & 140                   & 14                         & 5.000         \\ \hline
%\end{tabular}%
%%}
%\end{table*}
%


%% Please add the following required packages to your document preamble:
%% \usepackage{graphicx}
%\begin{table*}[t]
%\caption{Statistical Tests of HV with two objectives}
%\label{tab:Tests_HV_2obj}
%\centering
%%\resizebox{\textwidth}{!}{%
%\begin{tabular}{c c|c|c|c|c|c|c|c|c|c|c|c|c|c|c|c}
%\cline{2-17}
% & \multicolumn{4}{c|}{\textbf{MOEA/D}} & \multicolumn{4}{c|}{\textbf{NSGA-II}} & \multicolumn{4}{c|}{\textbf{R2-EMOA}} & \multicolumn{4}{c}{\textbf{VSD-MOEA}} \\ \cline{2-17} 
% & \textbf{$\uparrow$} & \textbf{$\downarrow$} & \textbf{$\leftrightarrow$} & \textbf{Diff} & \textbf{$\uparrow$} & \textbf{$\downarrow$} & \textbf{$\leftrightarrow$} & \textbf{Diff} & \textbf{$\uparrow$} & \textbf{$\downarrow$} & \textbf{$\leftrightarrow$} & \textbf{Diff} & \textbf{$\uparrow$} & \textbf{$\downarrow$} & \textbf{$\leftrightarrow$} & \textbf{Diff} \\ \hline
%\multicolumn{1}{c|}{\textbf{WFG1}} & 1 & 0 & 2 & 0.000 & 0 & 2 & 1 & 0.000 & 0 & 0 & 3 & 0.005 & 1 & 0 & 2 & 0.000 \\ \hline
%\multicolumn{1}{c|}{\textbf{WFG2}} & 1 & 2 & 0 & 0.032 & 2 & 1 & 0 & 0.024 & 0 & 3 & 0 & 0.033 & 3 & 0 & 0 & 0.000 \\ \hline
%\multicolumn{1}{c|}{\textbf{WFG3}} & 2 & 1 & 0 & 0.001 & 0 & 3 & 0 & 0.005 & 1 & 2 & 0 & 0.001 & 3 & 0 & 0 & 0.000 \\ \hline
%\multicolumn{1}{c|}{\textbf{WFG4}} & 1 & 2 & 0 & 0.003 & 0 & 3 & 0 & 0.006 & 3 & 0 & 0 & 0.000 & 2 & 1 & 0 & 0.001 \\ \hline
%\multicolumn{1}{c|}{\textbf{WFG5}} & 0 & 3 & 0 & 0.044 & 1 & 1 & 1 & 0.036 & 1 & 1 & 1 & 0.035 & 3 & 0 & 0 & 0.000 \\ \hline
%\multicolumn{1}{c|}{\textbf{WFG6}} & 1 & 0 & 2 & 0.000 & 1 & 0 & 2 & 0.001 & 1 & 0 & 2 & 0.002 & 0 & 3 & 0 & 0.045 \\ \hline
%\multicolumn{1}{c|}{\textbf{WFG7}} & 1 & 2 & 0 & 0.003 & 0 & 3 & 0 & 0.007 & 3 & 0 & 0 & 0.000 & 2 & 1 & 0 & 0.001 \\ \hline
%\multicolumn{1}{c|}{\textbf{WFG8}} & 1 & 2 & 0 & 0.136 & 0 & 3 & 0 & 0.158 & 2 & 1 & 0 & 0.133 & 3 & 0 & 0 & 0.000 \\ \hline
%\multicolumn{1}{c|}{\textbf{WFG9}} & 1 & 1 & 1 & 0.087 & 0 & 3 & 0 & 0.138 & 1 & 1 & 1 & 0.086 & 3 & 0 & 0 & 0.000 \\ \hline
%\multicolumn{1}{c|}{\textbf{DTLZ1}} & 3 & 0 & 0 & 0.000 & 0 & 3 & 0 & 0.002 & 2 & 1 & 0 & 0.001 & 1 & 2 & 0 & 0.001 \\ \hline
%\multicolumn{1}{c|}{\textbf{DTLZ2}} & 1 & 2 & 0 & 0.002 & 0 & 3 & 0 & 0.004 & 3 & 0 & 0 & 0.000 & 2 & 1 & 0 & 0.001 \\ \hline
%\multicolumn{1}{c|}{\textbf{DTLZ3}} & 1 & 2 & 0 & 0.002 & 0 & 3 & 0 & 0.003 & 3 & 0 & 0 & 0.000 & 2 & 1 & 0 & 0.001 \\ \hline
%\multicolumn{1}{c|}{\textbf{DTLZ4}} & 0 & 2 & 1 & 0.209 & 1 & 1 & 1 & 0.128 & 0 & 0 & 3 & 0.334 & 2 & 0 & 1 & 0.000 \\ \hline
%\multicolumn{1}{c|}{\textbf{DTLZ5}} & 1 & 2 & 0 & 0.002 & 0 & 3 & 0 & 0.004 & 3 & 0 & 0 & 0.000 & 2 & 1 & 0 & 0.001 \\ \hline
%\multicolumn{1}{c|}{\textbf{DTLZ6}} & 1 & 1 & 1 & 0.291 & 0 & 3 & 0 & 0.668 & 1 & 1 & 1 & 0.299 & 3 & 0 & 0 & 0.000 \\ \hline
%\multicolumn{1}{c|}{\textbf{DTLZ7}} & 0 & 3 & 0 & 0.001 & 2 & 1 & 0 & 0.001 & 3 & 0 & 0 & 0.000 & 1 & 2 & 0 & 0.001 \\ \hline
%\multicolumn{1}{c|}{\textbf{UF1}} & 1 & 1 & 1 & 0.002 & 0 & 3 & 0 & 0.006 & 1 & 1 & 1 & 0.004 & 3 & 0 & 0 & 0.000 \\ \hline
%\multicolumn{1}{c|}{\textbf{UF2}} & 3 & 0 & 0 & 0.000 & 0 & 3 & 0 & 0.010 & 1 & 1 & 1 & 0.003 & 1 & 1 & 1 & 0.002 \\ \hline
%\multicolumn{1}{c|}{\textbf{UF3}} & 0 & 2 & 1 & 0.270 & 2 & 1 & 0 & 0.084 & 0 & 2 & 1 & 0.279 & 3 & 0 & 0 & 0.000 \\ \hline
%\multicolumn{1}{c|}{\textbf{UF4}} & 1 & 2 & 0 & 0.020 & 0 & 3 & 0 & 0.040 & 3 & 0 & 0 & 0.000 & 2 & 1 & 0 & 0.003 \\ \hline
%\multicolumn{1}{c|}{\textbf{UF5}} & 0 & 3 & 0 & 0.175 & 1 & 0 & 2 & 0.018 & 1 & 0 & 2 & 0.092 & 1 & 0 & 2 & 0.000 \\ \hline
%\multicolumn{1}{c|}{\textbf{UF6}} & 0 & 3 & 0 & 0.380 & 2 & 1 & 0 & 0.192 & 1 & 2 & 0 & 0.258 & 3 & 0 & 0 & 0.000 \\ \hline
%\multicolumn{1}{c|}{\textbf{UF7}} & 2 & 0 & 1 & 0.000 & 1 & 2 & 0 & 0.009 & 0 & 3 & 0 & 0.079 & 2 & 0 & 1 & 0.001 \\ \hline
%\multicolumn{1}{c|}{\textbf{Total}} & 23 & 36 & 10 & 1.661 & 13 & 49 & 7 & 1.542 & 34 & 19 & 16 & 1.643 & 48 & 14 & 7 & 0.060 \\ \hline
%\end{tabular}%
%%}
%\end{table*}



%%% Please add the following required packages to your document preamble:
%%% \usepackage{graphicx}
%%\begin{table*}[t]
%%\caption{Statistical Tests of HV with two objectives}
%%\label{tab:Tests_HV_2obj}
%%\centering
%%%\resizebox{\textwidth}{!}{%
%%\begin{tabular}{c c|c|c|c|c|c|c|c|c|c|c|c|c|c|c|c}
%%\cline{2-17}
%%\textbf{} & \multicolumn{4}{c|}{\textbf{MOEA/D}} & \multicolumn{4}{c|}{\textbf{NSGA-II}} & \multicolumn{4}{c|}{\textbf{R2-MOEA}} & \multicolumn{4}{c}{\textbf{VSD-MOEA}} \\ \cline{2-17} 
%% & \textbf{$\uparrow$} & \textbf{$\downarrow$} & \textbf{$\leftrightarrow$} & \textbf{Diff} & \textbf{$\uparrow$} & \textbf{$\downarrow$} & \textbf{$\leftrightarrow$} & \textbf{Diff} & \textbf{$\uparrow$} & \textbf{$\downarrow$} & \textbf{$\leftrightarrow$} & \textbf{Diff} & \textbf{$\uparrow$} & \textbf{$\downarrow$} & \textbf{$\leftrightarrow$} & \textbf{Diff} \\ \hline
%%\multicolumn{1}{c|}{\textbf{WFG1}} & 1 & 1 & 1 & 0.000 & 0 & 2 & 1 & 0.001 & 0 & 1 & 2 & 0.005 & 3 & 0 & 0 & 0.000 \\ \hline
%%\multicolumn{1}{c|}{\textbf{WFG2}} & 1 & 2 & 0 & 0.032 & 2 & 1 & 0 & 0.024 & 0 & 3 & 0 & 0.033 & 3 & 0 & 0 & 0.000 \\ \hline
%%\multicolumn{1}{c|}{\textbf{WFG3}} & 2 & 1 & 0 & 0.001 & 0 & 3 & 0 & 0.005 & 1 & 2 & 0 & 0.001 & 3 & 0 & 0 & 0.000 \\ \hline
%%\multicolumn{1}{c|}{\textbf{WFG4}} & 1 & 2 & 0 & 0.003 & 0 & 3 & 0 & 0.006 & 3 & 0 & 0 & 0.000 & 2 & 1 & 0 & 0.001 \\ \hline
%%\multicolumn{1}{c|}{\textbf{WFG5}} & 0 & 3 & 0 & 0.041 & 1 & 1 & 1 & 0.033 & 1 & 1 & 1 & 0.032 & 3 & 0 & 0 & 0.000 \\ \hline
%%\multicolumn{1}{c|}{\textbf{WFG6}} & 1 & 0 & 2 & 0.000 & 1 & 0 & 2 & 0.001 & 1 & 0 & 2 & 0.002 & 0 & 3 & 0 & 0.046 \\ \hline
%%\multicolumn{1}{c|}{\textbf{WFG7}} & 1 & 2 & 0 & 0.003 & 0 & 3 & 0 & 0.007 & 3 & 0 & 0 & 0.000 & 2 & 1 & 0 & 0.001 \\ \hline
%%\multicolumn{1}{c|}{\textbf{WFG8}} & 1 & 2 & 0 & 0.141 & 0 & 3 & 0 & 0.163 & 2 & 1 & 0 & 0.138 & 3 & 0 & 0 & 0.000 \\ \hline
%%\multicolumn{1}{c|}{\textbf{WFG9}} & 1 & 1 & 1 & 0.086 & 0 & 3 & 0 & 0.137 & 1 & 1 & 1 & 0.085 & 3 & 0 & 0 & 0.000 \\ \hline
%%\multicolumn{1}{c|}{\textbf{DTLZ1}} & 3 & 0 & 0 & 0.000 & 0 & 3 & 0 & 0.002 & 2 & 1 & 0 & 0.001 & 1 & 2 & 0 & 0.001 \\ \hline
%%\multicolumn{1}{c|}{\textbf{DTLZ2}} & 1 & 2 & 0 & 0.002 & 0 & 3 & 0 & 0.004 & 3 & 0 & 0 & 0.000 & 2 & 1 & 0 & 0.001 \\ \hline
%%\multicolumn{1}{c|}{\textbf{DTLZ3}} & 1 & 2 & 0 & 0.002 & 0 & 3 & 0 & 0.003 & 3 & 0 & 0 & 0.000 & 2 & 1 & 0 & 0.001 \\ \hline
%%\multicolumn{1}{c|}{\textbf{DTLZ4}} & 0 & 2 & 1 & 0.209 & 1 & 1 & 1 & 0.128 & 0 & 0 & 3 & 0.334 & 2 & 0 & 1 & 0.000 \\ \hline
%%\multicolumn{1}{c|}{\textbf{DTLZ5}} & 1 & 2 & 0 & 0.002 & 0 & 3 & 0 & 0.004 & 3 & 0 & 0 & 0.000 & 2 & 1 & 0 & 0.001 \\ \hline
%%\multicolumn{1}{c|}{\textbf{DTLZ6}} & 1 & 1 & 1 & 0.291 & 0 & 3 & 0 & 0.668 & 1 & 1 & 1 & 0.299 & 3 & 0 & 0 & 0.000 \\ \hline
%%\multicolumn{1}{c|}{\textbf{DTLZ7}} & 0 & 3 & 0 & 0.001 & 2 & 1 & 0 & 0.001 & 3 & 0 & 0 & 0.000 & 1 & 2 & 0 & 0.001 \\ \hline
%%\multicolumn{1}{c|}{\textbf{UF1}} & 1 & 1 & 1 & 0.002 & 0 & 3 & 0 & 0.007 & 1 & 1 & 1 & 0.004 & 3 & 0 & 0 & 0.000 \\ \hline
%%\multicolumn{1}{c|}{\textbf{UF2}} & 3 & 0 & 0 & 0.000 & 0 & 3 & 0 & 0.010 & 1 & 1 & 1 & 0.003 & 1 & 1 & 1 & 0.003 \\ \hline
%%\multicolumn{1}{c|}{\textbf{UF3}} & 0 & 2 & 1 & 0.284 & 2 & 1 & 0 & 0.097 & 0 & 2 & 1 & 0.292 & 3 & 0 & 0 & 0.000 \\ \hline
%%\multicolumn{1}{c|}{\textbf{UF4}} & 1 & 2 & 0 & 0.020 & 0 & 3 & 0 & 0.040 & 3 & 0 & 0 & 0.000 & 2 & 1 & 0 & 0.003 \\ \hline
%%\multicolumn{1}{c|}{\textbf{UF5}} & 0 & 3 & 0 & 0.205 & 1 & 1 & 1 & 0.048 & 1 & 1 & 1 & 0.122 & 3 & 0 & 0 & 0.000 \\ \hline
%%\multicolumn{1}{c|}{\textbf{UF6}} & 0 & 3 & 0 & 0.442 & 2 & 1 & 0 & 0.253 & 1 & 2 & 0 & 0.320 & 3 & 0 & 0 & 0.000 \\ \hline
%%\multicolumn{1}{c|}{\textbf{UF7}} & 2 & 0 & 1 & 0.000 & 1 & 2 & 0 & 0.009 & 0 & 3 & 0 & 0.079 & 2 & 0 & 1 & 0.000 \\ \hline
%%\multicolumn{1}{c|}{\textbf{Total}} & \textbf{23} & \textbf{37} & \textbf{9} & \textbf{1.768} & \textbf{13} & \textbf{50} & \textbf{6} & \textbf{1.649} & \textbf{34} & \textbf{21} & \textbf{14} & \textbf{1.749} & \textbf{52} & \textbf{14} & \textbf{3} & \textbf{0.061} \\ \hline
%%\end{tabular}%
%%%}
%%\end{table*}
%%

% Please add the following required packages to your document preamble:
% \usepackage{graphicx}
\begin{table*}[t]
\caption{Statistical Tests of HV with Three Objectives}
\label{tab:Tests_HV_3obj}
\centering
%\resizebox{\textwidth}{!}{%
\begin{tabular}{c c|c|c|c|c|c|c|c|c|c|c|c|c|c|c|c}
\cline{2-17}
\textbf{} & \multicolumn{4}{c|}{\textbf{MOEA/D}} & \multicolumn{4}{c|}{\textbf{NSGA-II}} & \multicolumn{4}{c|}{\textbf{R2-MOEA}} & \multicolumn{4}{c}{\textbf{VSD-MOEA}} \\ \cline{2-17} 
 & \textbf{$\uparrow$} & \textbf{$\downarrow$} & \textbf{$\leftrightarrow$} & \textbf{Diff} & \textbf{$\uparrow$} & \textbf{$\downarrow$} & \textbf{$\leftrightarrow$} & \textbf{Diff} & \textbf{$\uparrow$} & \textbf{$\downarrow$} & \textbf{$\leftrightarrow$} & \textbf{Diff} & \textbf{$\uparrow$} & \textbf{$\downarrow$} & \textbf{$\leftrightarrow$} & \textbf{Diff} \\ \hline
\multicolumn{1}{c|}{\textbf{WFG1}} & 1 & 2 & 0 & 0.016 & 0 & 3 & 0 & 0.047 & 2 & 1 & 0 & 0.007 & 3 & 0 & 0 & 0.000 \\ \hline
\multicolumn{1}{c|}{\textbf{WFG2}} & 2 & 1 & 0 & 0.014 & 1 & 2 & 0 & 0.022 & 0 & 3 & 0 & 0.027 & 3 & 0 & 0 & 0.000 \\ \hline
\multicolumn{1}{c|}{\textbf{WFG3}} & 3 & 0 & 0 & 0.000 & 0 & 3 & 0 & 0.008 & 2 & 1 & 0 & 0.001 & 1 & 2 & 0 & 0.004 \\ \hline
\multicolumn{1}{c|}{\textbf{WFG4}} & 0 & 3 & 0 & 0.055 & 1 & 2 & 0 & 0.052 & 2 & 1 & 0 & 0.015 & 3 & 0 & 0 & 0.000 \\ \hline
\multicolumn{1}{c|}{\textbf{WFG5}} & 0 & 3 & 0 & 0.055 & 1 & 2 & 0 & 0.032 & 2 & 1 & 0 & 0.005 & 3 & 0 & 0 & 0.000 \\ \hline
\multicolumn{1}{c|}{\textbf{WFG6}} & 0 & 2 & 1 & 0.048 & 0 & 2 & 1 & 0.043 & 3 & 0 & 0 & 0.000 & 2 & 1 & 0 & 0.022 \\ \hline
\multicolumn{1}{c|}{\textbf{WFG7}} & 0 & 3 & 0 & 0.055 & 1 & 2 & 0 & 0.044 & 2 & 1 & 0 & 0.016 & 3 & 0 & 0 & 0.000 \\ \hline
\multicolumn{1}{c|}{\textbf{WFG8}} & 1 & 2 & 0 & 0.121 & 0 & 3 & 0 & 0.187 & 2 & 1 & 0 & 0.084 & 3 & 0 & 0 & 0.000 \\ \hline
\multicolumn{1}{c|}{\textbf{WFG9}} & 1 & 2 & 0 & 0.055 & 0 & 3 & 0 & 0.128 & 2 & 1 & 0 & 0.002 & 3 & 0 & 0 & 0.000 \\ \hline
\multicolumn{1}{c|}{\textbf{DTLZ1}} & 2 & 1 & 0 & 0.014 & 1 & 2 & 0 & 0.022 & 0 & 3 & 0 & 0.024 & 3 & 0 & 0 & 0.000 \\ \hline
\multicolumn{1}{c|}{\textbf{DTLZ2}} & 1 & 2 & 0 & 0.031 & 0 & 3 & 0 & 0.044 & 2 & 1 & 0 & 0.015 & 3 & 0 & 0 & 0.000 \\ \hline
\multicolumn{1}{c|}{\textbf{DTLZ3}} & 1 & 2 & 0 & 0.031 & 0 & 3 & 0 & 0.039 & 2 & 1 & 0 & 0.015 & 3 & 0 & 0 & 0.000 \\ \hline
\multicolumn{1}{c|}{\textbf{DTLZ4}} & 0 & 2 & 1 & 0.117 & 1 & 1 & 1 & 0.041 & 0 & 1 & 2 & 0.254 & 3 & 0 & 0 & 0.000 \\ \hline
\multicolumn{1}{c|}{\textbf{DTLZ5}} & 0 & 3 & 0 & 0.007 & 1 & 2 & 0 & 0.003 & 2 & 0 & 1 & 0.000 & 2 & 0 & 1 & 0.000 \\ \hline
\multicolumn{1}{c|}{\textbf{DTLZ6}} & 1 & 2 & 0 & 0.395 & 0 & 3 & 0 & 0.743 & 2 & 1 & 0 & 0.314 & 3 & 0 & 0 & 0.000 \\ \hline
\multicolumn{1}{c|}{\textbf{DTLZ7}} & 1 & 2 & 0 & 0.050 & 2 & 1 & 0 & 0.040 & 0 & 3 & 0 & 0.104 & 3 & 0 & 0 & 0.000 \\ \hline
\multicolumn{1}{c|}{\textbf{UF8}} & 1 & 2 & 0 & 0.145 & 0 & 3 & 0 & 0.455 & 2 & 1 & 0 & 0.020 & 3 & 0 & 0 & 0.000 \\ \hline
\multicolumn{1}{c|}{\textbf{UF9}} & 1 & 1 & 1 & 0.117 & 0 & 3 & 0 & 0.235 & 1 & 1 & 1 & 0.119 & 3 & 0 & 0 & 0.000 \\ \hline
\multicolumn{1}{c|}{\textbf{UF10}} & 0 & 2 & 1 & 0.287 & 0 & 2 & 1 & 0.386 & 2 & 1 & 0 & 0.214 & 3 & 0 & 0 & 0.000 \\ \hline
\multicolumn{1}{c|}{\textbf{Total}} & \textbf{16} & \textbf{37} & \textbf{4} & \textbf{1.615} & \textbf{9} & \textbf{45} & \textbf{3} & \textbf{2.571} & \textbf{30} & \textbf{23} & \textbf{4} & \textbf{1.237} & \textbf{53} & \textbf{3} & \textbf{1} & \textbf{0.026} \\ \hline
\end{tabular}%
%}
\end{table*}


%%% Please add the following required packages to your document preamble:
%% \usepackage{graphicx}
%\begin{table*}[t]
%\caption{Statistical Tests of IGD+ with Two Objectives}
%\label{tab:Tests_IGDP_2obj}
%\centering
%%\resizebox{\textwidth}{!}{%
%\begin{tabular}{c|c|c|c|c|c|c|c|c|c|c|c|c|c|c|c|c|}
%\cline{2-17}
%\textbf{} & \multicolumn{4}{c|}{\textbf{MOEA/D}} & \multicolumn{4}{c|}{\textbf{NSGA-II}} & \multicolumn{4}{c|}{\textbf{R2-MOEA}} & \multicolumn{4}{c|}{\textbf{VSD-MOEA}} \\ \cline{2-17} 
% & \textbf{$\uparrow$} & \textbf{$\downarrow$} & \textbf{$\leftrightarrow$} & \textbf{Diff} & \textbf{$\uparrow$} & \textbf{$\downarrow$} & \textbf{$\leftrightarrow$} & \textbf{Diff} & \textbf{$\uparrow$} & \textbf{$\downarrow$} & \textbf{$\leftrightarrow$} & \textbf{Diff} & \textbf{$\uparrow$} & \textbf{$\downarrow$} & \textbf{$\leftrightarrow$} & \textbf{Diff} \\ \hline
%\multicolumn{1}{|c|}{\textbf{WFG1}} & 1 & 1 & 1 & 0.000 & 0 & 1 & 2 & 0.001 & 0 & 2 & 1 & 0.006 & 3 & 0 & 0 & 0.000 \\ \hline
%\multicolumn{1}{|c|}{\textbf{WFG2}} & 1 & 2 & 0 & 0.049 & 2 & 1 & 0 & 0.037 & 0 & 3 & 0 & 0.051 & 3 & 0 & 0 & 0.000 \\ \hline
%\multicolumn{1}{|c|}{\textbf{WFG3}} & 2 & 1 & 0 & 0.000 & 0 & 3 & 0 & 0.005 & 1 & 2 & 0 & 0.001 & 3 & 0 & 0 & 0.000 \\ \hline
%\multicolumn{1}{|c|}{\textbf{WFG4}} & 1 & 2 & 0 & 0.001 & 0 & 3 & 0 & 0.003 & 3 & 0 & 0 & 0.000 & 2 & 1 & 0 & 0.001 \\ \hline
%\multicolumn{1}{|c|}{\textbf{WFG5}} & 0 & 1 & 2 & 0.026 & 0 & 2 & 1 & 0.026 & 1 & 1 & 1 & 0.025 & 3 & 0 & 0 & 0.000 \\ \hline
%\multicolumn{1}{|c|}{\textbf{WFG6}} & 1 & 0 & 2 & 0.000 & 1 & 0 & 2 & 0.001 & 1 & 0 & 2 & 0.002 & 0 & 3 & 0 & 0.030 \\ \hline
%\multicolumn{1}{|c|}{\textbf{WFG7}} & 1 & 2 & 0 & 0.001 & 0 & 3 & 0 & 0.003 & 3 & 0 & 0 & 0.000 & 2 & 1 & 0 & 0.001 \\ \hline
%\multicolumn{1}{|c|}{\textbf{WFG8}} & 1 & 1 & 1 & 0.083 & 0 & 3 & 0 & 0.096 & 1 & 1 & 1 & 0.082 & 3 & 0 & 0 & 0.000 \\ \hline
%\multicolumn{1}{|c|}{\textbf{WFG9}} & 1 & 1 & 1 & 0.056 & 0 & 3 & 0 & 0.090 & 1 & 1 & 1 & 0.055 & 3 & 0 & 0 & 0.000 \\ \hline
%\multicolumn{1}{|c|}{\textbf{DTLZ1}} & 3 & 0 & 0 & 0.000 & 0 & 3 & 0 & 0.000 & 1 & 1 & 1 & 0.000 & 1 & 1 & 1 & 0.000 \\ \hline
%\multicolumn{1}{|c|}{\textbf{DTLZ2}} & 1 & 2 & 0 & 0.000 & 0 & 3 & 0 & 0.001 & 3 & 0 & 0 & 0.000 & 2 & 1 & 0 & 0.000 \\ \hline
%\multicolumn{1}{|c|}{\textbf{DTLZ3}} & 1 & 2 & 0 & 0.000 & 0 & 3 & 0 & 0.001 & 3 & 0 & 0 & 0.000 & 2 & 1 & 0 & 0.000 \\ \hline
%\multicolumn{1}{|c|}{\textbf{DTLZ4}} & 0 & 2 & 1 & 0.103 & 1 & 1 & 1 & 0.062 & 0 & 0 & 3 & 0.165 & 2 & 0 & 1 & 0.000 \\ \hline
%\multicolumn{1}{|c|}{\textbf{DTLZ5}} & 1 & 2 & 0 & 0.000 & 0 & 3 & 0 & 0.001 & 3 & 0 & 0 & 0.000 & 2 & 1 & 0 & 0.000 \\ \hline
%\multicolumn{1}{|c|}{\textbf{DTLZ6}} & 1 & 1 & 1 & 0.073 & 0 & 3 & 0 & 0.203 & 1 & 1 & 1 & 0.076 & 3 & 0 & 0 & 0.000 \\ \hline
%\multicolumn{1}{|c|}{\textbf{DTLZ7}} & 0 & 3 & 0 & 0.001 & 1 & 1 & 1 & 0.001 & 3 & 0 & 0 & 0.000 & 1 & 1 & 1 & 0.001 \\ \hline
%\multicolumn{1}{|c|}{\textbf{UF1}} & 1 & 1 & 1 & 0.001 & 0 & 3 & 0 & 0.003 & 1 & 1 & 1 & 0.001 & 3 & 0 & 0 & 0.000 \\ \hline
%\multicolumn{1}{|c|}{\textbf{UF2}} & 3 & 0 & 0 & 0.000 & 0 & 3 & 0 & 0.006 & 2 & 1 & 0 & 0.001 & 1 & 2 & 0 & 0.003 \\ \hline
%\multicolumn{1}{|c|}{\textbf{UF3}} & 0 & 2 & 1 & 0.129 & 2 & 1 & 0 & 0.032 & 0 & 2 & 1 & 0.131 & 3 & 0 & 0 & 0.000 \\ \hline
%\multicolumn{1}{|c|}{\textbf{UF4}} & 1 & 2 & 0 & 0.004 & 0 & 3 & 0 & 0.012 & 3 & 0 & 0 & 0.000 & 2 & 1 & 0 & 0.001 \\ \hline
%\multicolumn{1}{|c|}{\textbf{UF5}} & 0 & 3 & 0 & 0.147 & 1 & 1 & 1 & 0.024 & 1 & 1 & 1 & 0.096 & 3 & 0 & 0 & 0.000 \\ \hline
%\multicolumn{1}{|c|}{\textbf{UF6}} & 0 & 3 & 0 & 0.354 & 2 & 1 & 0 & 0.168 & 1 & 2 & 0 & 0.240 & 3 & 0 & 0 & 0.000 \\ \hline
%\multicolumn{1}{|c|}{\textbf{UF7}} & 2 & 0 & 1 & 0.000 & 1 & 2 & 0 & 0.003 & 0 & 3 & 0 & 0.042 & 2 & 0 & 1 & 0.000 \\ \hline
%\multicolumn{1}{|c|}{\textbf{Total}} & \textbf{23} & \textbf{34} & \textbf{12} & \textbf{1.033} & \textbf{11} & \textbf{50} & \textbf{8} & \textbf{0.779} & \textbf{33} & \textbf{22} & \textbf{14} & \textbf{0.976} & \textbf{52} & \textbf{13} & \textbf{4} & \textbf{0.036} \\ \hline
%\end{tabular}%
%%}
%\end{table*}
%

%% Please add the following required packages to your document preamble:
% \usepackage{graphicx}
\begin{table}[t]
\centering
\caption{Statistical Tests and Deterioration Level of the IGD+ for three objectives}
\label{tab:Tests_HV_3obj}
%\resizebox{\textwidth}{!}{%
\begin{tabular}{c c|c|c|c}
\cline{2-5}
                                        & \textbf{$\uparrow$} & \textbf{$\downarrow$} & \textbf{$\leftrightarrow$} & \textbf{Deterioration} \\ \hline
\multicolumn{1}{c|}{\textbf{MOEA/D}}   & 15                  & 37                    & 5                          & 0.787         \\ \hline
\multicolumn{1}{c|}{\textbf{NSGA-II}}  & 6                   & 46                    & 5                          & 1.214         \\ \hline
\multicolumn{1}{c|}{\textbf{R2-EMOA}}  & 35                  & 16                    & 6                          & 0.669         \\ \hline
\multicolumn{1}{c|}{\textbf{VSD-MOEA}} & 49                  & 6                     & 2                          & 0.039         \\ \hline
\end{tabular}%
%}
\end{table}

%%% Please add the following required packages to your document preamble:
%%% \usepackage{graphicx}
%%\begin{table*}[t]
%%\caption{Statistical Tests of IGD+ with Three Objectives}
%%\label{tab:Tests_IGDP_3obj}
%%\centering
%%%\resizebox{\textwidth}{!}{%
%%\begin{tabular}{c|c|c|c|c|c|c|c|c|c|c|c|c|c|c|c|c|}
%%\cline{2-17}
%%\textbf{} & \multicolumn{4}{c|}{\textbf{MOEA/D}} & \multicolumn{4}{c|}{\textbf{NSGA-II}} & \multicolumn{4}{c|}{\textbf{R2-MOEA}} & \multicolumn{4}{c|}{\textbf{VSD-MOEA}} \\ \cline{2-17} 
%% & \textbf{$\uparrow$} & \textbf{$\downarrow$} & \textbf{$\leftrightarrow$} & \textbf{Diff} & \textbf{$\uparrow$} & \textbf{$\downarrow$} & \textbf{$\leftrightarrow$} & \textbf{Diff} & \textbf{$\uparrow$} & \textbf{$\downarrow$} & \textbf{$\leftrightarrow$} & \textbf{Diff} & \textbf{$\uparrow$} & \textbf{$\downarrow$} & \textbf{$\leftrightarrow$} & \textbf{Diff} \\ \hline
%%\multicolumn{1}{|c|}{\textbf{WFG1}} & 1 & 2 & 0 & 0.034 & 0 & 3 & 0 & 0.104 & 2 & 1 & 0 & 0.022 & 3 & 0 & 0 & 0.000 \\ \hline
%%\multicolumn{1}{|c|}{\textbf{WFG2}} & 2 & 1 & 0 & 0.025 & 1 & 2 & 0 & 0.060 & 0 & 3 & 0 & 0.065 & 3 & 0 & 0 & 0.000 \\ \hline
%%\multicolumn{1}{|c|}{\textbf{WFG3}} & 2 & 1 & 0 & 0.000 & 0 & 3 & 0 & 0.017 & 3 & 0 & 0 & 0.000 & 1 & 2 & 0 & 0.010 \\ \hline
%%\multicolumn{1}{|c|}{\textbf{WFG4}} & 1 & 2 & 0 & 0.034 & 0 & 3 & 0 & 0.039 & 2 & 1 & 0 & 0.004 & 3 & 0 & 0 & 0.000 \\ \hline
%%\multicolumn{1}{|c|}{\textbf{WFG5}} & 0 & 3 & 0 & 0.034 & 1 & 2 & 0 & 0.023 & 2 & 1 & 0 & 0.006 & 3 & 0 & 0 & 0.000 \\ \hline
%%\multicolumn{1}{|c|}{\textbf{WFG6}} & 0 & 2 & 1 & 0.034 & 0 & 2 & 1 & 0.037 & 3 & 0 & 0 & 0.000 & 2 & 1 & 0 & 0.022 \\ \hline
%%\multicolumn{1}{|c|}{\textbf{WFG7}} & 0 & 3 & 0 & 0.033 & 1 & 2 & 0 & 0.030 & 2 & 1 & 0 & 0.004 & 3 & 0 & 0 & 0.000 \\ \hline
%%\multicolumn{1}{|c|}{\textbf{WFG8}} & 1 & 2 & 0 & 0.086 & 0 & 3 & 0 & 0.150 & 2 & 1 & 0 & 0.058 & 3 & 0 & 0 & 0.000 \\ \hline
%%\multicolumn{1}{|c|}{\textbf{WFG9}} & 1 & 2 & 0 & 0.048 & 0 & 3 & 0 & 0.118 & 2 & 0 & 1 & 0.013 & 2 & 0 & 1 & 0.000 \\ \hline
%%\multicolumn{1}{|c|}{\textbf{DTLZ1}} & 1 & 2 & 0 & 0.001 & 0 & 3 & 0 & 0.004 & 3 & 0 & 0 & 0.000 & 2 & 1 & 0 & 0.000 \\ \hline
%%\multicolumn{1}{|c|}{\textbf{DTLZ2}} & 1 & 2 & 0 & 0.004 & 0 & 3 & 0 & 0.009 & 3 & 0 & 0 & 0.000 & 2 & 1 & 0 & 0.001 \\ \hline
%%\multicolumn{1}{|c|}{\textbf{DTLZ3}} & 1 & 2 & 0 & 0.004 & 0 & 3 & 0 & 0.007 & 3 & 0 & 0 & 0.000 & 2 & 1 & 0 & 0.001 \\ \hline
%%\multicolumn{1}{|c|}{\textbf{DTLZ4}} & 0 & 2 & 1 & 0.068 & 1 & 1 & 1 & 0.007 & 0 & 0 & 3 & 0.165 & 2 & 0 & 1 & 0.000 \\ \hline
%%\multicolumn{1}{|c|}{\textbf{DTLZ5}} & 0 & 3 & 0 & 0.001 & 1 & 2 & 0 & 0.001 & 2 & 1 & 0 & 0.000 & 3 & 0 & 0 & 0.000 \\ \hline
%%\multicolumn{1}{|c|}{\textbf{DTLZ6}} & 1 & 2 & 0 & 0.085 & 0 & 3 & 0 & 0.185 & 2 & 1 & 0 & 0.067 & 3 & 0 & 0 & 0.000 \\ \hline
%%\multicolumn{1}{|c|}{\textbf{DTLZ7}} & 1 & 1 & 1 & 0.017 & 1 & 1 & 1 & 0.016 & 0 & 3 & 0 & 0.051 & 3 & 0 & 0 & 0.000 \\ \hline
%%\multicolumn{1}{|c|}{\textbf{UF8}} & 1 & 2 & 0 & 0.040 & 0 & 3 & 0 & 0.149 & 2 & 0 & 1 & 0.004 & 2 & 0 & 1 & 0.000 \\ \hline
%%\multicolumn{1}{|c|}{\textbf{UF9}} & 1 & 1 & 1 & 0.062 & 0 & 2 & 1 & 0.114 & 0 & 1 & 2 & 0.069 & 3 & 0 & 0 & 0.000 \\ \hline
%%\multicolumn{1}{|c|}{\textbf{UF10}} & 0 & 2 & 1 & 0.195 & 0 & 2 & 1 & 0.163 & 2 & 1 & 0 & 0.158 & 3 & 0 & 0 & 0.000 \\ \hline
%%\multicolumn{1}{|c|}{\textbf{Total}} & \textbf{15} & \textbf{37} & \textbf{5} & \textbf{0.806} & \textbf{6} & \textbf{46} & \textbf{5} & \textbf{1.232} & \textbf{35} & \textbf{15} & \textbf{7} & \textbf{0.687} & \textbf{48} & \textbf{6} & \textbf{3} & \textbf{0.036} \\ \hline
%%\end{tabular}%
%%%}
%%\end{table*}
%%

