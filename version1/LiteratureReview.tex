This section is devoted to review some of the most important papers that are closely related to our proposal.
%
First, some of the most popular ways of managing diversity in \EAS{} are presented.
%
Then, the state-of-the-art in \MOEAS{} is summarized.

\subsection{Diversity Management in Evolutionary Algorithms}

%In the \EAS{} field is well known that a proper balance between exploration and exploitation is one of the keys to attain quality solutions.
The proper balance between exploration and exploitation is one of the keys to success in the design of \EAS{}.
%
In the single-objective domain it is known that properly managing the diversity in the variable space is a way to control such balance,
and as a consequence, a large amount of diversity management techniques have been devised~\cite{Mohan:14}.
%
Particularly, these methods are classified depending on the component(s) of the \EA{} that is modified to alter the 
amount of maintained diversity.
%
A popular taxonomy identifies the following groups~\cite{Joel:Crepinsek}: \textit{selection-based}, \textit{population-based}, 
\textit{crossover/mutation-based}, \textit{fitness-based}, and \textit{replacement-based}.
%
Additionally, the methods are referred to as \textit{uniprocess-driven} when a single component is altered, whereas the term
\textit{multiprocess-driven} is used to refer to those methods that act on more than one component.

Among the previous proposals, the replacement-based methods have attained very high-quality results in last years~\cite{Segura:17}, so
this alternative was selected with the aim of designing a novel \MOEA{} incorporating an explicit way to control the diversity 
in the variable space.
%
The basic principle of these methods is to bias the level of exploration in successive generations by 
controlling the diversity of the survivors of the population~\cite{Segura:17}.
%
Since premature convergence is one of the most commond drawbacks in the design of \EAS{}, 
modifications are usually performed with the aim of slowing down the convergence.
%
One of the most popular proposals belonging to this group is the \textit{crowding} method which
is based on the principle that offspring should replace similar individuals from the previous generation~\cite{Mengshoel:14}.
%
Several replacement strategies that do not rely on crowding have also been devised.
%
In some methods, diversity is considered as an objective.
%
For instance, in the hybrid genetic search with adaptive diversity control (\HGSADC{})~\cite{Vidal:13}, individuals are sorted 
by their contribution to diversity and by their original cost.
%
Then, the rankings of the individuals are used in the fitness assignment phase.
%
A more recent proposal~\cite{Segura:17} incorporates a penalty approach to alter gradually the amount of diversity 
maintained in the population.
%
Particularly, initial phases preserve a larger amount of diversity than the final phases of the optimization.
%
This last method has inspired the design of the novel proposal put forth in this paper for multi-objective optimization.
%
%Thus, our proposal can be classified as an \textit{uniprocess-driven} and \textit{replacement-based} method.

It is important to remark that in the case of multi-objective optimization, few works related to the maintenance of 
diversity in the variable space have been developed.
%
The following section reviews some of the most important \MOEAS{} and introduces some of the works that consider
the maintenance of diversity in the variable space.

\subsection{Multi-objective Evolutionary Algorithms}

In recent decades, several \MOEAS{} have been proposed. 
%
While the purpose of most of them is to provide a well-spread set of solutions close to the Pareto front,
several ways of facing this purpose have been devised.
%
Therefore, several taxonomies have been proposed with the aim of better classifying the different 
schemes~\cite{Joel:BOOK_MOEAs}.
%
Particularly, a \MOEA{} can be designed based on Pareto dominance, indicators and/or decomposition~\cite{Joel:StateArt}.
%
Since none of the groups has a remarkable superiority over the others, in this work all of them are taken into account to validate
our proposal.
%
This section introduces the three types of schemes and some of the most popular approaches belonging to each category.
%
Then, one \MOEA{} of each category is selected to carry out the validation of \VSDMOEA{}.

The dominance-based category includes those schemes where the Pareto dominance relation is used to guide the 
design of some of its components such as the fitness assignment, parent selection and replacement phase.
%
The dominance relation does not inherently promotes the preservation of diversity in the objective space, 
therefore additional techniques such as niching, crowding and/or clustering are usually integrated with the aim of 
obtaining a proper spread and convergence to the Pareto front.
%
The most popular dominance-based \MOEA{} is the Non-Dominated Sorting Genetic Algorithm II (NSGA-II)~\cite{Joel:NSGAII}.
%
%Another quite popular proposal that belongs to this category is the Generalized Differential Evolution (\GDEIII{})~\cite{Joel:GDE3}, 
%which integrates a crowding mechanism, and is an extension of the single-objective Differential Evolution.

In order to assess the performance of \MOEAS{}, several quality indicators have been devised.
%
In the indicator-based \MOEAS{}, the use of the Pareto dominance relation is substituted by some quality indicators 
to guide the decisions performed by the \MOEA{}.
%
An advantage of indicator-based algorithms is that the indicators usually take into account both the quality and 
diversity in objective space, so incorporating additional mechanisms to promote diversity in the objective 
space is not required.
%
Among the different indicators, hypervolume is a widely accepted Pareto-compliance quality indicator.
%
The Indicator-Based Evolutionary Algorithm (\IBEA{})~\cite{Joel:IBEA} was the first method belonging to this category.
%
A more recent one is the R2-Indicator-Based Evolutionary Multi-objective Algorithm (\RMOEA{})~\cite{trautmann2013r2}, 
%
%
which has reported a quite promising performance in multi-objective problems.
%
%Thereafter the Many-Objective Genetic Algorithm Based on the R2 Indicator (MOMBI-II)~\cite{Joel:MOMBI-II}.
%which has reported quite promising performance both with multi-objective and many-objective problems.
%
%MOMBI-II might be considered as a hybrid between indicator-based and dominance-based because it also uses the 
%Pareto dominance. 
%
Its most important feature is the use of the R2 indicator, which computes the mean difference in utilities through a set 
of weight vectors.
%In any case, its most important feature is the use of the R2 indicator. % \cite{Joel:R2_Indicator}.
%
%MOMBI-II has been selected as the indicator-based \MOEA{} to validate our proposal.

Finally, decomposition-based \MOEAS{}~\cite{Joel:MOEAD_AMS} are based on transforming the \MOP{} into a set of 
single-objective optimization problems that are tackled simultaneously.
%
This transformation can be performed in several ways, e.g. with a linear weighted sum or with a weighted Tchebycheff function. 
%
Given a set of weights to establish different single-objective functions, the \MOEA{} searches for a single 
high-quality solution for each of them. 
%
The weight vectors should be selected with the aim of obtaining a well-spread set of solutions~\cite{Joel:Kalyanmoy}.
%
The Multi-objective Evolutionary Algorithm Based on Decomposition (\MOEAD{})~\cite{Joel:MOEAD} is the most popular 
decomposition-based \MOEA{}. 
%
Its main principles include problem decomposition, weighted aggregation of objectives and mating restrictions 
through the use of neighborhoods. 
%
Different ways of aggregating the objectives have been tested with \MOEAD{}.
%
Among them, the use of the Tchebycheff approach is quite popular. 
%
%\MOEAD{} based on the Tchebycheff approach has been used to validate our proposal.
 
It is important to stand out that none of the most popular algorithms in the multi-objective field introduce special 
mechanisms to promote diversity in variable space.
%
However, some efforts have been dedicated to this principle.
%
A popular approach to promote the diversity in the decision space is the application of fitness sharing~\cite{Joel:NPGA} 
in a similar way than in single-objective optimization.
%
Although, fitness sharing might be used to promote diversity both in objective and decision variable space, most
popular variants consider only distances in the objective space.
%
Another \MOEA{} designed to promote diversity in both the decision and the objective space is the Genetic
Diversity Evolutionary Algorithm (GDEA)~\cite{toffolo2003genetic}.
%
In this case, each individual is assigned with a diversity-based objective which is calculated as the
Euclidean distance in the genotype space to the remaining individuals in the population.
%
Then, a ranking that considers both the original objectives and the diversity objectiv is used
to sort individuals.
%
Another somewhat popular approach is to calculate distances between candidate solutions by taking
into account both the objective and variable space~\cite{deb2005omni,shir2009enhancing} with the aim
of promoting diversity in both spaces.
%
A different proposal combines the use of two selection operators~\cite{chan2005evolutionary}.
%
The first one promotes diversity and quality in the objective space whereas the second one promotes diversity in the decision space.
%
In the same line, modifying the hypervolume to integrate the decision space diversity in a single metric was proposed in~\cite{ulrich2010integrating}.
%
In this approach, the proposed metric is used to guide the selection in the \MOEA{}.
%
Finally, some indirect mechanisms that might affect the diversity have also been introduced in some schemes.
%
Probably, the most popular one is the use of mating restrictions~\cite{Joel:STUDY_MATTING_RESTRICTION,Joel:MOEAD_AMS}.

Attending to the analyses of the previous approaches, it is clear that they might bring benefits to decision makers
because the final solutions obtained by these methods present a larger decision space diversity than the ones obtained
by traditional approaches~\cite{deb2005omni, rudolph2007capabilities}.
%
Thus, while clear improvements are obtained when taking into account metrics related to the Pareto set, the benefits in terms of the 
obtained Pareto front are not so clear.
%
We claim that one of the reasons of this behavior might be that the diversity in the variable space is considered 
in the whole optimization process.
%
However, in a similar way that in the single objective domain, reducing the importance granted to the diversity in the decision space as the generations progress is really important~\cite{Joel:MULTI_DYNAMIC}.
%
Currently, no \MOEA{} considers this idea, so this principle has guided the design of our novel \MOEA{}.
