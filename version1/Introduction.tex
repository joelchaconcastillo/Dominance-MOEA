\IEEEPARstart{M}{ulti-objective} Optimization Problems (\MOPS{}) %have a number of objective functions which are to be minimized or maximized.
%
%Thus, differently to single-objective optimization, \MOPS{} 
involve the simultaneous optimization of several objective functions that are usually in conflict~\cite{Joel:Kalyanmoy}. 
%
A continuous box-constrained minimization \MOP{}, which is the kind of problem addressed in this paper, can be defined as follows:
\begin{equation}
   \begin{split}
    minimize \quad & \vec{F} = [f_1(\vec{\mathbf{x}}), f_2(\vec{\mathbf{x}}), ..., f_M(\vec{\mathbf{x}})] \\
   subject \quad to \quad &  x_i^{(L)} \leq x_i \leq x_i^{(U)}, \quad i=1,2,..., n. \\
   %& \vec{\mathbf{x}} \in \Omega
   \end{split}
\end{equation}
where $n$ corresponds to the dimension of the variable space, $\vec{\mathbf{x}}$ is a vector of $n$ 
decision variables $\vec{\mathbf{x}}=(x_1, ..., x_n) \in R^n$, which are constrained by $x_i^{(L)}$ 
and $x_i^{(U)}$, i.e. the lower bound and upper bound, and $M$ is the number of objective functions
to optimize.
%
The feasible space bounded by $x_i^{(L)}$ and $x_i^{(U)}$ is denoted by $\Omega$, 
each solution is mapped to the objective space with the function $F : \Omega \rightarrow R^M$, 
which consist of $M$ real-valued objective functions and $R^M$ is called the \textit{objective space}. 

Given two solutions $\vec{\mathbf{x}}$, $\vec{\mathbf{y}}$ $\in \Omega$, $\vec{\mathbf{x}}$ dominates $\vec{\mathbf{y}}$, 
mathematically denoted by $\vec{\mathbf{x}} \prec \vec{\mathbf{y}}$, iff $\forall m \in {1,2,...,M} : 
f_m(\vec{\mathbf{x}}) \leq f_m(\vec{\mathbf{y}})$ and $\exists  m \in {1,2,...,M} : f_m(\vec{\mathbf{x}}) < f_m(\vec{\mathbf{y}})$.
%
The best solutions of a \MOP{} are those whose objective vectors are not dominated by any other feasible vector.
%
These solutions are known as the Pareto optimal solutions.
%
The Pareto set is the set of all Pareto optimal solutions, and the Pareto front are the images of the Pareto set. 
%
The goal of multi-objective optimization approaches is to obtain a proper approximation of the Pareto front, i.e., 
a set of well distributed solutions that are close to the Pareto front.

One of the most popular metaheuristics used to deal with \MOPS{} is the Evolutionary Algorithm (\EA{}).
%
In single-objective \EAS{}, it has been shown that taking into account the diversity of the variable space
to properly balance between exploration and exploitation is highly important to attain high quality 
solutions~\cite{Joel:BALANCE_DIVERSITY}.
%
Diversity can be taken into account in the design of several components such as in the variation 
stage~\cite{Joel:FUZZY_ADAPTIVE_GA,Joel:CROSSOVER_DIVERSITY}, replacement phase~\cite{Joel:MULTI_DYNAMIC} 
and/or population model~\cite{Joel:SAWTOOTH}.
%In the uniprocess-driven proposals only one component is affected, whereas in the multiprocess-driven ones, several components are redesigned~\cite{Joel:Crepinsek}. 
%
%One of the key issues that affects the performance of population-based metaheuristics is the premature convergence, which appears
%when most of the population members are placed in a small region of the search space and the components selected do not allow escaping from this region.
%
The explicit consideration of diversity leads to improvements in terms of premature convergence avoidance, 
meaning that taking into account the diversity in the design of \EAS{} is specially important when dealing 
with long-term executions.
%
Recently, some diversity management algorithms that combine the information of diversity, stopping criterion and elapsed 
generations have been devised.
%
They have allowed to provide a gradual loss of diversity that depends on the time or evaluations granted to the 
execution~\cite{Joel:MULTI_DYNAMIC}.
%
Particularly the aim of such methodology is to promote exploration in the initial generations and gradually alter the 
behaviour towards intensification.
%
These schemes have provided really promising results.
%
For instance, new best-known solutions for some well-known variants of the frequency assignment problem~\cite{Segura:17} 
and for a two-dimensional packing problem~\cite{Joel:MULTI_DYNAMIC} have been attained using such methodology.
%
Additionally, this principle guided the design of the winning strategy of the Second Wind Farm Layout Optimization 
Competition\footnote{https://www.irit.fr/wind-competition/}, which was held in the Genetic and Evolutionary 
Computation Conference.
%
Thus, the benefits of such methodology have been shown in several different single-objective optimization problems.

One of the goals in the design of Multi-objective Evolutionary Algorithms (\MOEAS{}) is to obtain a well-spread 
set of solutions in the objective space.
%
The maintenance of some degree of diversity in the objective space implies that complete convergence 
does not appear in the variable space~\cite{Joel:GDE3_CEC09}.
%
In some way, the variable space inherits some degree of diversity due to the way in which the objective space is 
taken into account. 
%
However, this is just an indirect way of preserving the diversity in the variable space, so 
in some cases the level of diversity might not be large enough to ensure a high degree of exploration.
%
For instance, it has been shown that with some of the \WFG{} benchmarks, in most of the state-of-the-art \MOEAS{} 
the distance parameters quickly converge, meaning that the approach focuses just on optimizing the 
position parameters for a long period of the optimization process~\cite{Joel:GDE3_CEC09}.
%
Thus, while some degree of diversity is maintained, a similar situation to premature convergence is presented
meaning that genetic operators might not be able to generate better trade-offs. 

Attending to the differences between state-of-the-art single-objective \EAS{} and \MOEAS{}, 
this paper proposes a novel \MOEA{}, the Variable-Space-Diversity based \MOEA{} (\VSDMOEA{}), 
that is based on controlling the amount of diversity in the variable space in an explicit way.
%
Similarly to the successful methodology applied in single-objective optimization, the stopping criterion and the 
amount of evaluations performed are used to vary the amount of desired diversity.
%
The main difference with respect to the single-objective case is that the objective space is simultaneously considered.
%
Particularly, the approach grants more importance to the diversity of the variable space in the initial stages, whereas 
as the generations evolve, it gradually grants more importance to the diversity of the objective space.
%
In fact, at the end of the execution, the diversity of the variable space is neglected, so in the last phases the proposal is similar to current state-of-the-art approaches.
%
To our knowledge, this is the first \MOEA{} whose design follows this principle.
%
Since there exist currently a quite large amount of different \MOEAS{}~\cite{Joel:MOEA_APPLICATIONS_BOOK_KCTAN}, 
four popular schemes have been selected to validate our proposal.
%
%Most current taxonomies classify \MOEAS{} into three categories: non-domination based, based on decomposition and based on metrics.
%
%An approach of each of these categories was selected.
%
%Particularly, they are the Non-Dominated Sorting Genetic Algorithm II (NSGA-II)~\cite{Joel:NSGAII}, the \MOEA{} based on decomposition (\MOEAD{})~\cite{Joel:MOEAD}, 
%and the Many-Objective Genetic Algorithm Based on the R2 Indicator (MOMBI-II)~\cite{Joel:MOMBI-II}.
%
%Additionally, the Generalized Differential Evolution (GDE3)~\cite{Joel:GDE3}, which have a particularly different variation stage that has reported much benefits in several cases,
%is also included.
%
This validation has been performed with several well-known benchmarks and proper quality metrics.
%
The important benefits of properly taking into account the diversity of the variable space is
clearly shown in this paper.
%
Particularly, the advantages are clearer in the most complex problems.
%
Note that this is consistent with the single-objective case, where the most important benefits have been obtained
in complex multi-modal cases~\cite{Segura:17}.

The rest of this paper is organized as follows. 
%
Section~\ref{Sec:LiteratureReview} provides a review of related papers.
%
Some key components related to diversity and the \VSDMOEA{} proposal are detailed in section~\ref{Sec:Proposal}.
%
Section~\ref{Sec:ExperimentalValidation} is devoted to the experimental validation of the novel proposal.
%
Finally, conclusions and some lines of future work are given in Section~\ref{Sec:Conclusion}.
%
Note also that some supplementary materials are given.
%
They include details of the experimental results with additional metrics as well as some
explanatory videos.
