The evolutionary algorithms have been one of the most popular approaches to deal with complex optimization problems.
%
The performance of multi-objective algorithms is measured directly in the objective space, therefore the majority of algorithms incorporate sophisticated mechanisms to attain well spread and converged solutions among the Pareto front.
%
Nevertheless, the multi-objective problems are build with some difficulties which mislead several algorithms and drive to premature convergence or stagnation.
%
Mainly, in difficult problems and with long-term executions stagnation could be present wasting important computational resources.
%
In evolutionary algorithms, the diversity of the decision variable space has showed a critical role to avoid premature convergence or stagnation.
%
Particularly, stagnation could converge in sub-optimal regions provoking that the operators lose its exploratiory strength.
%
For this reason, a potential strategy to mitigate those drawbacks dwell in leading the search process through different diversity stages, which change as the stopping criterion is reached.

Mainly, in this work two contributions applied in the multi-objective field are described.
%
First, is presented a mechanism to simultaneoulsy manage the diversity in the variable space and the objective space.
%
This strategy is incorporated in the replacement phase, specifically the diversity maintained in the decision variable space is decremented according the elapsed time and the stopping criterion.
%
Second, a novel density estimator of the objective space, which is based in the IGD+ indicator, is suggested.
%
In the experimental validation carried out, is shown that our proposal not only improves the state-of-the-art algorithms in long-term executions, also offers a competitive performance in short-term executions.
%
%
In addition, some scalability experiments in the decision variable space are carried out with middle-term executions, results indicate the superiority and stability measured with the hypervolume indicator.
%
In the same line, the evolution of diversity in the variable space with some specific problems is analysed.
%
Showing that the problems of two objectives maintain implicitly less diveristy in decision variable space than the problems of three objectives.
%


In the future, we plan to develop an adaptive scheme for the initial threshold value, as well the development of alternatives in short-term executions to provide improved solutions.
%
In order, to attain even better results, these strategies are going to be incorporated in a multi-objective memetic algorithm.
%
Finally, the integration of the replacement phase in a different multi-objective paradigm (indicator or decomposition) seems to be promising to give an emphasis of the diversity implications in the multi-objective field.

