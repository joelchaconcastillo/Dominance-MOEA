The evolutionary algorithms have been one of the most popular approaches to deal with complex optimization problems.
%
The performance of multi-objective algorithms is measured directly in the objective space, therefore the majority of algorithms incorporate sophisticated mechanisms to attain well spread and converged solutions among the Pareto front.
%
Nevertheless, the multi-objective problems are build with some difficulties which mislead several algorithms and drive to premature convergence or stagnation.
%
Mainly in difficult problems and with long-term executions stagnation could be present wasting important resources.
%
In evolutionary algorithms, the diversity of the decision variable space has showed a critical role to avoid premature convergence or stagnation.
%
However stagnation could converge in sub-optimal regions provoking that the operators lose its exploratiory strength.
%
For this reason, a potential strategy to mitigate those drawbacks dwell in leading the search process through different diversity levels, which change as the stopping criterion is reached.

Mainly, in this work two contributions applied in the multi-objective field are described.
%
First, is presented a mechanism to simultaneoulsy manage the diversity in the variable space and the objective space.
%
This strategy is incorporated in the replacement phase, specifically the diversity maintained in the decision variable space is decrement according the time elapsed and the stopping criterion.
%
Second, a novel density estimator of the objective space, which is based in the IGD+ indicator, is suggested.
%
In the experimental validation carried out, is showed that our proposal not only improves the state-of-the-art algorithms in long-term execution, also offers a competitive performance in short-term executions.
%
%This validation shows that the VSD-MOEA is able to properly solve several test problems, also it attained better results than the remaining \MOEAS{} in the problems with the most difficult characteristics.
%
In addition, some scalability experiments in the decision variable space are carried out with middle-term executions, results indicate the superiority and stability measured with the hypervolume indicator.
%
In the same line, the evolution of diversity in the variable space with some specific problems is analyzed.
%
Showing that the problem with two objectives maintin less diveristy implicitly than the problems woth three objectives.
%


In the future, we plan to develop an adaptive scheme for the initial distance factor, as well the development of alternatives in short-term executions to provide quality solutions.
%
In order, to attain even better results, these strategies are going to be incorporated in a multi-objective memetic algorithm.
%
Finally, the integration of the replacement phase in a different multi-objective paradigm (indicator or decomposition) seems to be promising.

%The quality of the solutions in a multi-objective problem is directly measured in the objective space, therefore the main 
%
%Given that the quality of the \MOEAS{} is directly measured in the objective space, the majority of them incorporate sophisticated mechanisms with the aim to attain diverse and converged solutions to the Pareto Front.
%
%Since this, the diversity in the decision variables tend to be neglected.
%
%However, equivalently than in single-objective problems, where the diversity in variable space has a critical role, the multi-objective problems should take into consideration the diversity in both spaces to avoid premature convergence in sub-optimal regions.
%
%In this work we have provided an algorithm with a particular replacement phase.
%
%This phase considers the diversity in both spaces, specifically in the variable space the diversity is based in a decremented dynamic concept.
%
%Thus, at the first stages the diversity in variables space is promoted, as the generations elapses this diversity is gradually reduced, thus at the last stages the replacement phase works as a classic \MOEA{}.
%

%
%
