\EAS{} have been one of the most popular approaches to deal with complex optimization problems.
%
Its design is a quite complex task that require the definition of several components.
%
Attending to the differences between single-objective and multi-objective optimizers, it is remarkable
that several state-of-the-art single-objective optimizers consider explicitly the diversity of the variable space specially
when dealing with long-term executions, whereas this is not the case for \MOEAS{}.

This paper propose a novel \MOEA{}, which is called \VSDMOEA{} that takes into account both the diversity of the variable space
and the diversity of the objective space.
%
The main novelty is that the importance given to the different diversities is adapted during the optimization process.
%
Particularly, in \VSDMOEA{} more importance is given to the diversity of the variable space in the initial stages
but, as the generations evolves, it gradually grants more importance to the diversity of the objective space.
%
This is performed with a penalty method that is integrated in the replacement phase.
%
Additionally, a novel density estimator based on the IGD+ is integrated.
%
This is used to select among the non-penalized individuals.

The experimental validation carried out shows a remarkable improvement of \VSDMOEA{} when compared to state-of-the-art \MOEAS{} both in
two-objective and three-objective problems.
%
Moreover, our proposal not only improves the state-of-the-art algorithms in long-term and mid-term executions,
but also offers a competitive performance in short-term executions.
%
The scalability analyses show that as the number of objectives and variables increases the implicit variable space maintained by state-of-the-art
\MOEAS{} also increases.
%
Thus, for large enough objectives and variables, explicitly considering the diversity of the variable space is not so helpful.
%
Finally, the analysis of the initial threshold distance, which is an additional parameter required by \VSDMOEA{}, shows that finding a proper
value for this parameter is not a difficult task.

In the future we plan to integrate the principles studied in this paper with other categories of \MOEAS{}.
%
For instance, including the management of diversity put forth in this paper into decomposition-based and indicator-based \MOEAS{} seems plausible.
%
Additionally, we would like to develop an adaptive scheme for avoiding the setting of the initial threshold value.
%
Finally, in order to attain even better results, these strategies are going to be incorporated into a multi-objective memetic algorithm.
%
