% Please add the following required packages to your document preamble:
% \usepackage{graphicx}
\begin{table*}[t]
\caption{Statistics IGD+ with two objectives}
\label{tab:StatisticsIGDP_2obj}
\resizebox{\textwidth}{!}{%
\begin{tabular}{c|c|c|c|c|c|c|c|c|}
\cline{2-9}
                                     & \multicolumn{2}{c|}{\textbf{MOEA/D}}              & \multicolumn{2}{c|}{\textbf{NSGA-II}}             & \multicolumn{2}{c|}{\textbf{R2-MOEA}}             & \multicolumn{2}{c|}{\textbf{VSD-MOEA}}            \\ \cline{2-9} 
                                     & \textbf{{[}Min, Max{]}} & \textbf{Mean $\pm$ Std} & \textbf{{[}Min, Max{]}} & \textbf{Mean $\pm$ Std} & \textbf{{[}Min, Max{]}} & \textbf{Mean $\pm$ Std} & \textbf{{[}Min, Max{]}} & \textbf{Mean $\pm$ Std} \\ \hline
\multicolumn{1}{|c|}{\textbf{WFG1}}  & 0.006, 0.015            & $0.008 \pm 0.002$       & 0.006, 0.014            & $0.008 \pm 0.002$       & 0.881, 0.902            & $0.890 \pm 0.005$       & 0.006, 0.025            & $0.007 \pm 0.003$       \\ \hline
\multicolumn{1}{|c|}{\textbf{WFG2}}  & 0.006, 0.055            & $0.052 \pm 0.011$       & 0.003, 0.053            & $0.040 \pm 0.022$       & 0.054, 0.198            & $0.075 \pm 0.050$       & 0.003, 0.003            & $0.003 \pm 0.000$       \\ \hline
\multicolumn{1}{|c|}{\textbf{WFG3}}  & 0.008, 0.008            & $0.008 \pm 0.000$       & 0.011, 0.013            & $0.012 \pm 0.000$       & 0.008, 0.009            & $0.009 \pm 0.000$       & 0.007, 0.007            & $0.007 \pm 0.000$       \\ \hline
\multicolumn{1}{|c|}{\textbf{WFG4}}  & 0.007, 0.007            & $0.007 \pm 0.000$       & 0.007, 0.010            & $0.008 \pm 0.001$       & 0.005, 0.005            & $0.005 \pm 0.000$       & 0.006, 0.006            & $0.006 \pm 0.000$       \\ \hline
\multicolumn{1}{|c|}{\textbf{WFG5}}  & 0.060, 0.069            & $0.065 \pm 0.002$       & 0.060, 0.068            & $0.066 \pm 0.002$       & 0.067, 0.067            & $0.067 \pm 0.000$       & 0.033, 0.053            & $0.040 \pm 0.005$       \\ \hline
\multicolumn{1}{|c|}{\textbf{WFG6}}  & 0.034, 0.073            & $0.050 \pm 0.010$       & 0.034, 0.064            & $0.051 \pm 0.007$       & 0.120, 0.120            & $0.120 \pm 0.000$       & 0.068, 0.090            & $0.081 \pm 0.005$       \\ \hline
\multicolumn{1}{|c|}{\textbf{WFG7}}  & 0.007, 0.007            & $0.007 \pm 0.000$       & 0.008, 0.010            & $0.009 \pm 0.000$       & 0.005, 0.006            & $0.005 \pm 0.000$       & 0.006, 0.006            & $0.006 \pm 0.000$       \\ \hline
\multicolumn{1}{|c|}{\textbf{WFG8}}  & 0.103, 0.120            & $0.112 \pm 0.005$       & 0.116, 0.139            & $0.125 \pm 0.005$       & 0.096, 0.126            & $0.110 \pm 0.006$       & 0.024, 0.035            & $0.029 \pm 0.003$       \\ \hline
\multicolumn{1}{|c|}{\textbf{WFG9}}  & 0.011, 0.125            & $0.067 \pm 0.053$       & 0.014, 0.127            & $0.101 \pm 0.046$       & 0.122, 0.126            & $0.125 \pm 0.001$       & 0.009, 0.015            & $0.011 \pm 0.001$       \\ \hline
\multicolumn{1}{|c|}{\textbf{DTLZ1}} & 0.001, 0.001            & $0.001 \pm 0.000$       & 0.002, 0.002            & $0.002 \pm 0.000$       & 0.001, 0.001            & $0.001 \pm 0.000$       & 0.001, 0.001            & $0.001 \pm 0.000$       \\ \hline
\multicolumn{1}{|c|}{\textbf{DTLZ2}} & 0.002, 0.002            & $0.002 \pm 0.000$       & 0.002, 0.003            & $0.003 \pm 0.000$       & 0.002, 0.002            & $0.002 \pm 0.000$       & 0.002, 0.002            & $0.002 \pm 0.000$       \\ \hline
\multicolumn{1}{|c|}{\textbf{DTLZ3}} & 0.002, 0.002            & $0.002 \pm 0.000$       & 0.002, 0.003            & $0.002 \pm 0.000$       & 0.002, 0.002            & $0.002 \pm 0.000$       & 0.002, 0.002            & $0.002 \pm 0.000$       \\ \hline
\multicolumn{1}{|c|}{\textbf{DTLZ4}} & 0.002, 0.363            & $0.105 \pm 0.163$       & 0.002, 0.363            & $0.064 \pm 0.136$       & 0.002, 0.363            & $0.146 \pm 0.177$       & 0.002, 0.002            & $0.002 \pm 0.000$       \\ \hline
\multicolumn{1}{|c|}{\textbf{DTLZ5}} & 0.002, 0.002            & $0.002 \pm 0.000$       & 0.002, 0.003            & $0.003 \pm 0.000$       & 0.002, 0.002            & $0.002 \pm 0.000$       & 0.002, 0.002            & $0.002 \pm 0.000$       \\ \hline
\multicolumn{1}{|c|}{\textbf{DTLZ6}} & 0.022, 0.149            & $0.076 \pm 0.027$       & 0.126, 0.315            & $0.205 \pm 0.036$       & 0.002, 0.002            & $0.002 \pm 0.000$       & 0.002, 0.002            & $0.002 \pm 0.000$       \\ \hline
\multicolumn{1}{|c|}{\textbf{DTLZ7}} & 0.003, 0.003            & $0.003 \pm 0.000$       & 0.002, 0.003            & $0.003 \pm 0.000$       & 0.002, 0.361            & $0.094 \pm 0.157$       & 0.003, 0.003            & $0.003 \pm 0.000$       \\ \hline
\multicolumn{1}{|c|}{\textbf{UF1}}   & 0.004, 0.004            & $0.004 \pm 0.000$       & 0.005, 0.006            & $0.006 \pm 0.000$       & 0.028, 0.084            & $0.049 \pm 0.022$       & 0.003, 0.003            & $0.003 \pm 0.000$       \\ \hline
\multicolumn{1}{|c|}{\textbf{UF2}}   & 0.003, 0.005            & $0.004 \pm 0.000$       & 0.008, 0.010            & $0.010 \pm 0.000$       & 0.005, 0.026            & $0.010 \pm 0.005$       & 0.005, 0.008            & $0.006 \pm 0.001$       \\ \hline
\multicolumn{1}{|c|}{\textbf{UF3}}   & 0.141, 0.237            & $0.180 \pm 0.022$       & 0.052, 0.127            & $0.084 \pm 0.020$       & 0.111, 0.604            & $0.242 \pm 0.145$       & 0.043, 0.077            & $0.052 \pm 0.006$       \\ \hline
\multicolumn{1}{|c|}{\textbf{UF4}}   & 0.024, 0.031            & $0.026 \pm 0.001$       & 0.027, 0.039            & $0.033 \pm 0.003$       & 0.061, 0.076            & $0.070 \pm 0.004$       & 0.021, 0.024            & $0.022 \pm 0.001$       \\ \hline
\multicolumn{1}{|c|}{\textbf{UF5}}   & 0.079, 0.593            & $0.265 \pm 0.120$       & 0.091, 0.254            & $0.142 \pm 0.033$       & 0.437, 0.761            & $0.607 \pm 0.079$       & 0.083, 0.145            & $0.118 \pm 0.015$       \\ \hline
\multicolumn{1}{|c|}{\textbf{UF6}}   & 0.066, 0.529            & $0.380 \pm 0.108$       & 0.037, 0.542            & $0.193 \pm 0.114$       & 0.229, 0.530            & $0.374 \pm 0.096$       & 0.019, 0.034            & $0.026 \pm 0.005$       \\ \hline
\multicolumn{1}{|c|}{\textbf{UF7}}   & 0.003, 0.005            & $0.004 \pm 0.000$       & 0.007, 0.008            & $0.007 \pm 0.000$       & 0.010, 0.175            & $0.029 \pm 0.041$       & 0.003, 0.005            & $0.004 \pm 0.000$       \\ \hline
\multicolumn{1}{|c|}{\textbf{Mean}}  & \textbf{0.026, 0.105}   & \textbf{0.062}          & \textbf{0.027, 0.095}   & \textbf{0.051}          & \textbf{0.098, 0.198}   & \textbf{0.132}          & \textbf{0.015, 0.024}   & \textbf{0.019}          \\ \hline
\end{tabular}%
}
\end{table*}

% Please add the following required packages to your document preamble:
% \usepackage{graphicx}
\begin{table*}[t]
\caption{Statistical Tests of IGD+ with Two Objectives}
\label{tab:Tests_IGDP_2obj}
\resizebox{\textwidth}{!}{%
\begin{tabular}{c|c|c|c|c|c|c|c|c|c|c|c|c|c|c|c|c|}
\cline{2-17}
\textbf{}                            & \multicolumn{4}{c|}{\textbf{MOEA/D}}                                                       & \multicolumn{4}{c|}{\textbf{NSGA-II}}                                                      & \multicolumn{4}{c|}{\textbf{R2-MOEA}}                                                      & \multicolumn{4}{c|}{\textbf{VSD-MOEA}}                                                     \\ \cline{2-17} 
                                     & \textbf{$\uparrow$} & \textbf{$\downarrow$} & \textbf{$\leftrightarrow$} & \textbf{Diff}   & \textbf{$\uparrow$} & \textbf{$\downarrow$} & \textbf{$\leftrightarrow$} & \textbf{Diff}   & \textbf{$\uparrow$} & \textbf{$\downarrow$} & \textbf{$\leftrightarrow$} & \textbf{Diff}   & \textbf{$\uparrow$} & \textbf{$\downarrow$} & \textbf{$\leftrightarrow$} & \textbf{Diff}   \\ \hline
\multicolumn{1}{|c|}{\textbf{WFG1}}  & 1                   & 1                     & 1                          & 0.0004          & 1                   & 1                     & 1                          & 0.0009          & 0                   & 3                     & 0                          & 0.8828          & 3                   & 0                     & 0                          & 0.0000          \\ \hline
\multicolumn{1}{|c|}{\textbf{WFG2}}  & 1                   & 2                     & 0                          & 0.0495          & 2                   & 1                     & 0                          & 0.0374          & 0                   & 3                     & 0                          & 0.0721          & 3                   & 0                     & 0                          & 0.0000          \\ \hline
\multicolumn{1}{|c|}{\textbf{WFG3}}  & 2                   & 1                     & 0                          & 0.0005          & 0                   & 3                     & 0                          & 0.0048          & 1                   & 2                     & 0                          & 0.0013          & 3                   & 0                     & 0                          & 0.0000          \\ \hline
\multicolumn{1}{|c|}{\textbf{WFG4}}  & 1                   & 2                     & 0                          & 0.0015          & 0                   & 3                     & 0                          & 0.0030          & 3                   & 0                     & 0                          & 0.0000          & 2                   & 1                     & 0                          & 0.0006          \\ \hline
\multicolumn{1}{|c|}{\textbf{WFG5}}  & 0                   & 1                     & 2                          & 0.0259          & 1                   & 1                     & 1                          & 0.0265          & 0                   & 2                     & 1                          & 0.0271          & 3                   & 0                     & 0                          & 0.0000          \\ \hline
\multicolumn{1}{|c|}{\textbf{WFG6}}  & 2                   & 0                     & 1                          & 0.0000          & 2                   & 0                     & 1                          & 0.0010          & 0                   & 3                     & 0                          & 0.0699          & 1                   & 2                     & 0                          & 0.0300          \\ \hline
\multicolumn{1}{|c|}{\textbf{WFG7}}  & 1                   & 2                     & 0                          & 0.0014          & 0                   & 3                     & 0                          & 0.0033          & 3                   & 0                     & 0                          & 0.0000          & 2                   & 1                     & 0                          & 0.0005          \\ \hline
\multicolumn{1}{|c|}{\textbf{WFG8}}  & 1                   & 1                     & 1                          & 0.0830          & 0                   & 3                     & 0                          & 0.0963          & 1                   & 1                     & 1                          & 0.0817          & 3                   & 0                     & 0                          & 0.0000          \\ \hline
\multicolumn{1}{|c|}{\textbf{WFG9}}  & 2                   & 1                     & 0                          & 0.0561          & 0                   & 2                     & 1                          & 0.0895          & 0                   & 2                     & 1                          & 0.1138          & 3                   & 0                     & 0                          & 0.0000          \\ \hline
\multicolumn{1}{|c|}{\textbf{DTLZ1}} & 3                   & 0                     & 0                          & 0.0000          & 0                   & 3                     & 0                          & 0.0004          & 2                   & 1                     & 0                          & 0.0001          & 1                   & 2                     & 0                          & 0.0002          \\ \hline
\multicolumn{1}{|c|}{\textbf{DTLZ2}} & 1                   & 2                     & 0                          & 0.0003          & 0                   & 3                     & 0                          & 0.0008          & 3                   & 0                     & 0                          & 0.0000          & 2                   & 1                     & 0                          & 0.0003          \\ \hline
\multicolumn{1}{|c|}{\textbf{DTLZ3}} & 1                   & 2                     & 0                          & 0.0003          & 0                   & 3                     & 0                          & 0.0005          & 3                   & 0                     & 0                          & 0.0000          & 2                   & 1                     & 0                          & 0.0002          \\ \hline
\multicolumn{1}{|c|}{\textbf{DTLZ4}} & 1                   & 2                     & 0                          & 0.1033          & 1                   & 1                     & 1                          & 0.0624          & 0                   & 1                     & 2                          & 0.1444          & 2                   & 0                     & 1                          & 0.0000          \\ \hline
\multicolumn{1}{|c|}{\textbf{DTLZ5}} & 1                   & 2                     & 0                          & 0.0003          & 0                   & 3                     & 0                          & 0.0008          & 3                   & 0                     & 0                          & 0.0000          & 2                   & 1                     & 0                          & 0.0003          \\ \hline
\multicolumn{1}{|c|}{\textbf{DTLZ6}} & 1                   & 2                     & 0                          & 0.0737          & 0                   & 3                     & 0                          & 0.2030          & 3                   & 0                     & 0                          & 0.0000          & 2                   & 1                     & 0                          & 0.0003          \\ \hline
\multicolumn{1}{|c|}{\textbf{DTLZ7}} & 1                   & 2                     & 0                          & 0.0006          & 2                   & 0                     & 1                          & 0.0000          & 0                   & 3                     & 0                          & 0.0917          & 2                   & 0                     & 1                          & 0.0000          \\ \hline
\multicolumn{1}{|c|}{\textbf{UF1}}   & 2                   & 1                     & 0                          & 0.0011          & 1                   & 2                     & 0                          & 0.0030          & 0                   & 3                     & 0                          & 0.0458          & 3                   & 0                     & 0                          & 0.0000          \\ \hline
\multicolumn{1}{|c|}{\textbf{UF2}}   & 3                   & 0                     & 0                          & 0.0000          & 0                   & 2                     & 1                          & 0.0061          & 0                   & 2                     & 1                          & 0.0066          & 2                   & 1                     & 0                          & 0.0026          \\ \hline
\multicolumn{1}{|c|}{\textbf{UF3}}   & 0                   & 2                     & 1                          & 0.1286          & 2                   & 1                     & 0                          & 0.0322          & 0                   & 2                     & 1                          & 0.1899          & 3                   & 0                     & 0                          & 0.0000          \\ \hline
\multicolumn{1}{|c|}{\textbf{UF4}}   & 2                   & 1                     & 0                          & 0.0036          & 1                   & 2                     & 0                          & 0.0109          & 0                   & 3                     & 0                          & 0.0479          & 3                   & 0                     & 0                          & 0.0000          \\ \hline
\multicolumn{1}{|c|}{\textbf{UF5}}   & 1                   & 2                     & 0                          & 0.1472          & 2                   & 1                     & 0                          & 0.0238          & 0                   & 3                     & 0                          & 0.4884          & 3                   & 0                     & 0                          & 0.0000          \\ \hline
\multicolumn{1}{|c|}{\textbf{UF6}}   & 0                   & 2                     & 1                          & 0.3542          & 2                   & 1                     & 0                          & 0.1677          & 0                   & 2                     & 1                          & 0.3489          & 3                   & 0                     & 0                          & 0.0000          \\ \hline
\multicolumn{1}{|c|}{\textbf{UF7}}   & 2                   & 0                     & 1                          & 0.0000          & 1                   & 2                     & 0                          & 0.0034          & 0                   & 3                     & 0                          & 0.0254          & 2                   & 0                     & 1                          & 0.0000          \\ \hline
\multicolumn{1}{|c|}{\textbf{Total}} & \textbf{30}         & \textbf{31}           & \textbf{8}                 & \textbf{1.0315} & \textbf{18}         & \textbf{44}           & \textbf{7}                 & \textbf{0.7778} & \textbf{22}         & \textbf{39}           & \textbf{8}                 & \textbf{2.6378} & \textbf{55}         & \textbf{11}           & \textbf{3}                 & \textbf{0.0350} \\ \hline
\end{tabular}%
}
\end{table*}

% Please add the following required packages to your document preamble:
% \usepackage{graphicx}
\begin{table*}[t]
\caption{Statistical Tests of IGD+ with Three Objectives}
\label{tab:Tests_IGDP_3obj}
\resizebox{\textwidth}{!}{%
\begin{tabular}{c|c|c|c|c|c|c|c|c|c|c|c|c|c|c|c|c|}
\cline{2-17}
                                     & \multicolumn{4}{c|}{\textbf{MOEA/D}}                                                       & \multicolumn{4}{c|}{\textbf{NSGA-II}}                                                      & \multicolumn{4}{c|}{\textbf{R2-MOEA}}                                                      & \multicolumn{4}{c|}{\textbf{VSD-MOEA}}                                                     \\ \cline{2-17} 
                                     & \textbf{$\uparrow$} & \textbf{$\downarrow$} & \textbf{$\leftrightarrow$} & \textbf{Diff}   & \textbf{$\uparrow$} & \textbf{$\downarrow$} & \textbf{$\leftrightarrow$} & \textbf{Diff}   & \textbf{$\uparrow$} & \textbf{$\downarrow$} & \textbf{$\leftrightarrow$} & \textbf{Diff}   & \textbf{$\uparrow$} & \textbf{$\downarrow$} & \textbf{$\leftrightarrow$} & \textbf{Diff}   \\ \hline
\multicolumn{1}{|c|}{\textbf{WFG1}}  & 2                   & 1                     & 0                          & 0.0341          & 1                   & 2                     & 0                          & 0.1041          & 0                   & 3                     & 0                          & 1.0762          & 3                   & 0                     & 0                          & 0.0000          \\ \hline
\multicolumn{1}{|c|}{\textbf{WFG2}}  & 2                   & 1                     & 0                          & 0.0254          & 1                   & 2                     & 0                          & 0.0598          & 0                   & 3                     & 0                          & 0.1187          & 3                   & 0                     & 0                          & 0.0000          \\ \hline
\multicolumn{1}{|c|}{\textbf{WFG3}}  & 3                   & 0                     & 0                          & 0.0000          & 0                   & 3                     & 0                          & 0.0167          & 2                   & 1                     & 0                          & 0.0017          & 1                   & 2                     & 0                          & 0.0099          \\ \hline
\multicolumn{1}{|c|}{\textbf{WFG4}}  & 1                   & 2                     & 0                          & 0.0339          & 0                   & 3                     & 0                          & 0.0388          & 2                   & 1                     & 0                          & 0.0046          & 3                   & 0                     & 0                          & 0.0000          \\ \hline
\multicolumn{1}{|c|}{\textbf{WFG5}}  & 0                   & 3                     & 0                          & 0.0339          & 1                   & 2                     & 0                          & 0.0233          & 2                   & 1                     & 0                          & 0.0079          & 3                   & 0                     & 0                          & 0.0000          \\ \hline
\multicolumn{1}{|c|}{\textbf{WFG6}}  & 1                   & 1                     & 1                          & 0.0119          & 1                   & 1                     & 1                          & 0.0143          & 0                   & 3                     & 0                          & 0.0433          & 3                   & 0                     & 0                          & 0.0000          \\ \hline
\multicolumn{1}{|c|}{\textbf{WFG7}}  & 0                   & 3                     & 0                          & 0.0333          & 1                   & 2                     & 0                          & 0.0297          & 2                   & 1                     & 0                          & 0.0083          & 3                   & 0                     & 0                          & 0.0000          \\ \hline
\multicolumn{1}{|c|}{\textbf{WFG8}}  & 1                   & 2                     & 0                          & 0.0862          & 0                   & 3                     & 0                          & 0.1501          & 2                   & 1                     & 0                          & 0.0692          & 3                   & 0                     & 0                          & 0.0000          \\ \hline
\multicolumn{1}{|c|}{\textbf{WFG9}}  & 2                   & 1                     & 0                          & 0.0479          & 0                   & 3                     & 0                          & 0.1177          & 1                   & 2                     & 0                          & 0.1032          & 3                   & 0                     & 0                          & 0.0000          \\ \hline
\multicolumn{1}{|c|}{\textbf{DTLZ1}} & 1                   & 2                     & 0                          & 0.0005          & 0                   & 3                     & 0                          & 0.0042          & 3                   & 0                     & 0                          & 0.0000          & 2                   & 1                     & 0                          & 0.0002          \\ \hline
\multicolumn{1}{|c|}{\textbf{DTLZ2}} & 1                   & 2                     & 0                          & 0.0041          & 0                   & 3                     & 0                          & 0.0086          & 3                   & 0                     & 0                          & 0.0000          & 2                   & 1                     & 0                          & 0.0010          \\ \hline
\multicolumn{1}{|c|}{\textbf{DTLZ3}} & 1                   & 2                     & 0                          & 0.0030          & 0                   & 3                     & 0                          & 0.0054          & 2                   & 1                     & 0                          & 0.0004          & 3                   & 0                     & 0                          & 0.0000          \\ \hline
\multicolumn{1}{|c|}{\textbf{DTLZ4}} & 0                   & 2                     & 1                          & 0.0678          & 2                   & 1                     & 0                          & 0.0073          & 0                   & 2                     & 1                          & 0.1546          & 3                   & 0                     & 0                          & 0.0000          \\ \hline
\multicolumn{1}{|c|}{\textbf{DTLZ5}} & 0                   & 3                     & 0                          & 0.0011          & 1                   & 2                     & 0                          & 0.0008          & 2                   & 1                     & 0                          & 0.0000          & 3                   & 0                     & 0                          & 0.0000          \\ \hline
\multicolumn{1}{|c|}{\textbf{DTLZ6}} & 1                   & 2                     & 0                          & 0.0847          & 0                   & 3                     & 0                          & 0.1847          & 2                   & 1                     & 0                          & 0.0000          & 3                   & 0                     & 0                          & 0.0000          \\ \hline
\multicolumn{1}{|c|}{\textbf{DTLZ7}} & 1                   & 1                     & 1                          & 0.0169          & 1                   & 1                     & 1                          & 0.0160          & 0                   & 3                     & 0                          & 0.2544          & 3                   & 0                     & 0                          & 0.0000          \\ \hline
\multicolumn{1}{|c|}{\textbf{UF8}}   & 1                   & 2                     & 0                          & 0.0410          & 0                   & 3                     & 0                          & 0.1503          & 3                   & 0                     & 0                          & 0.0000          & 2                   & 1                     & 0                          & 0.0013          \\ \hline
\multicolumn{1}{|c|}{\textbf{UF9}}   & 1                   & 2                     & 0                          & 0.0617          & 0                   & 3                     & 0                          & 0.1144          & 2                   & 1                     & 0                          & 0.0169          & 3                   & 0                     & 0                          & 0.0000          \\ \hline
\multicolumn{1}{|c|}{\textbf{UF10}}  & 0                   & 1                     & 2                          & 0.1950          & 0                   & 1                     & 2                          & 0.1627          & 0                   & 1                     & 2                          & 0.1827          & 3                   & 0                     & 0                          & 0.0000          \\ \hline
\multicolumn{1}{|c|}{\textbf{Total}} & \textbf{19}         & \textbf{33}           & \textbf{5}                 & \textbf{0.7823} & \textbf{9}          & \textbf{44}           & \textbf{4}                 & \textbf{1.2089} & \textbf{28}         & \textbf{26}           & \textbf{3}                 & \textbf{2.0421} & \textbf{52}         & \textbf{5}            & \textbf{0}                 & \textbf{0.0124} \\ \hline
\end{tabular}%
}
\end{table*}



This section presents the results obtained by \VSDMOEA{} and state-of-the-art schemes in terms of
the IGD+\cite{Joel:Inverted_Generational_Distance_Plus}.
%
Specifically, we present the results for the long-term executions, meaning
the stopping criterion was set to $250,000$ generations.
%
The structure of the tables is the same as in the main document.
%
Thus, the only modification is that instead of using the hypervolume, the IGD+ is used.

Table \ref{tab:StatisticsIGDP_2obj} shows the IGD+ obtained for the benchmark functions with two objectives.
%
Specifically, the minimum, maximum, mean and standard deviation of the IGD+ is given for each method and function tested.
%
The last row shows the results considering all the functions together.
%
For each function, the data for the method that yielded the lowest mean is shown in bold.
%
Additionally, all the methods that were not statistically inferior to said method are shown in bold.
%
From here on, the methods shown in bold for a given problem are referred to as the winning methods.
%
Based on the number of functions where each method is in the group of the winning methods for the cases 
with two objectives, the best methods are \VSDMOEA{} and \RMOEA{} with 13 and 8, respectively.
%
Thus, \VSDMOEA{} is the most competitive method in terms of this metric.
%
More impressive is the fact that the mean IGD+ attained by \VSDMOEA{}, when all the problems are considered simultaneously, is much lower 
than that attained by \RMOEA{}.
%
In fact, the total means of \RMOEA{} ($0.060$), \NSGAII{} ($0.051$) and \MOEAD{} ($0.062$) are quite similar.
%
In contrast, \VSDMOEA{} yielded a much lower value ($0.021$).
%
When the data is inspected carefully, it is clear that in the cases where \VSDMOEA{} loses, the difference with respect to the
best method is not very large.
%
For instance, the difference between the mean IGD+ attained by \VSDMOEA{} and by the best method was never larger
than $0.05$.
%
However, all the other methods exhibited a deterioration greater than $0.05$ in several cases.
%
Specifically, it happened in $7$, $5$ and $8$ problems for \MOEAD{}, \NSGAII{} and \RMOEA{}, respectively.
%
This means that even if \VSDMOEA{} loses in some cases, its deterioration is always small, exhibiting a much more 
robust behavior than any other method.
%
Exactly the same situation appeared when analyzing the data in terms of hypervolume.

In order to better clarify these findings, pair-wise statistical tests were done between each method tested in each 
function.
%
Table~\ref{tab:Tests_IGDP_2obj} shows the results, with the same meaning as in the main document.
%
The calculated data confirms that although \VSDMOEA{} loses in some cases, the overall numbers of wins and losses favor \VSDMOEA{}.
%
More importantly, the total deterioration is considerably lower in the case of \VSDMOEA{}, confirming that when \VSDMOEA{} loses, the deterioration is not 
very high.


Tables~\ref{tab:Tests_IGDP_3obj} and~\ref{tab:StatisticsIGDP_3obj} show the same information for the problems with three objectives.
%
In this case, the superiority of \VSDMOEA{} is even clearer.
%
Taking into account the mean of all the functions, \VSDMOEA{} again yielded a much lower mean IGD+ than the other methods.
%
Specifically, \VSDMOEA{} obtained a value of $0.059$, whereas the second-ranked algorithm (\RMOEA{}) obtained a value of $0.093$.
%
Once again, the difference between the mean IGD+ obtained by \VSDMOEA{} and by the best method was never greater
than $0.05$.
%
However, all the other methods exhibited a deterioration greater than $0.05$ in several cases.
%
In particular, this happened in $5$, $8$ and $7$ problems for \MOEAD{}, \NSGAII{}, \RMOEA{}, respectively.
%
Moreover, \VSDMOEA{} is much more superior than the other methods not only in terms of total deterioration, but also
in terms of total wins and losses.
%
\VSDMOEA{} was in the group of winning methods for 14 out of 19 functions, whereas the second best-ranked algorithm (\RMOEA{})
was in the group of winning methods for only 5 functions.
%
These conclusions are again quite similar to those drawn for the hypervolume in the main document.




% Please add the following required packages to your document preamble:
% \usepackage{graphicx}
\begin{table*}[t]
\caption{Statistics IGD+ with three objectives}
\label{tab:StatisticsIGDP_3obj}
\resizebox{\textwidth}{!}{%
\begin{tabular}{c|c|c|c|c|c|c|c|c|}
\cline{2-9}
\textbf{}                            & \multicolumn{2}{c|}{\textbf{MOEA/D}}              & \multicolumn{2}{c|}{\textbf{NSGA-II}}             & \multicolumn{2}{c|}{\textbf{R2-MOEA}}             & \multicolumn{2}{c|}{\textbf{VSD-MOEA}}            \\ \cline{2-9} 
\textbf{}                            & \textbf{{[}Min, Max{]}} & \textbf{Mean $\pm$ Std} & \textbf{{[}Min, Max{]}} & \textbf{Mean $\pm$ Std} & \textbf{{[}Min, Max{]}} & \textbf{Mean $\pm$ Std} & \textbf{{[}Min, Max{]}} & \textbf{Mean $\pm$ Std} \\ \hline
\multicolumn{1}{|c|}{\textbf{WFG1}}  & 0.080, 0.100            & $0.090 \pm 0.005$       & 0.142, 0.179            & $0.160 \pm 0.010$       & 1.116, 1.143            & $1.132 \pm 0.006$       & 0.050, 0.066            & $0.056 \pm 0.004$       \\ \hline
\multicolumn{1}{|c|}{\textbf{WFG2}}  & 0.057, 0.068            & $0.063 \pm 0.002$       & 0.073, 0.133            & $0.097 \pm 0.014$       & 0.111, 0.185            & $0.156 \pm 0.024$       & 0.031, 0.044            & $0.038 \pm 0.003$       \\ \hline
\multicolumn{1}{|c|}{\textbf{WFG3}}  & 0.023, 0.023            & $0.023 \pm 0.000$       & 0.031, 0.061            & $0.039 \pm 0.005$       & 0.023, 0.026            & $0.024 \pm 0.001$       & 0.033, 0.033            & $0.033 \pm 0.000$       \\ \hline
\multicolumn{1}{|c|}{\textbf{WFG4}}  & 0.127, 0.127            & $0.127 \pm 0.000$       & 0.121, 0.144            & $0.132 \pm 0.005$       & 0.094, 0.101            & $0.097 \pm 0.002$       & 0.091, 0.094            & $0.093 \pm 0.001$       \\ \hline
\multicolumn{1}{|c|}{\textbf{WFG5}}  & 0.177, 0.184            & $0.181 \pm 0.002$       & 0.160, 0.186            & $0.170 \pm 0.005$       & 0.153, 0.159            & $0.155 \pm 0.001$       & 0.143, 0.155            & $0.147 \pm 0.002$       \\ \hline
\multicolumn{1}{|c|}{\textbf{WFG6}}  & 0.155, 0.205            & $0.175 \pm 0.012$       & 0.159, 0.196            & $0.177 \pm 0.009$       & 0.202, 0.209            & $0.206 \pm 0.002$       & 0.143, 0.173            & $0.163 \pm 0.008$       \\ \hline
\multicolumn{1}{|c|}{\textbf{WFG7}}  & 0.127, 0.127            & $0.127 \pm 0.000$       & 0.113, 0.138            & $0.123 \pm 0.007$       & 0.096, 0.106            & $0.102 \pm 0.003$       & 0.092, 0.094            & $0.093 \pm 0.001$       \\ \hline
\multicolumn{1}{|c|}{\textbf{WFG8}}  & 0.189, 0.194            & $0.192 \pm 0.001$       & 0.244, 0.274            & $0.256 \pm 0.008$       & 0.163, 0.196            & $0.175 \pm 0.009$       & 0.101, 0.121            & $0.106 \pm 0.005$       \\ \hline
\multicolumn{1}{|c|}{\textbf{WFG9}}  & 0.130, 0.240            & $0.154 \pm 0.036$       & 0.138, 0.246            & $0.224 \pm 0.025$       & 0.205, 0.212            & $0.209 \pm 0.002$       & 0.101, 0.162            & $0.106 \pm 0.010$       \\ \hline
\multicolumn{1}{|c|}{\textbf{DTLZ1}} & 0.014, 0.014            & $0.014 \pm 0.000$       & 0.017, 0.020            & $0.018 \pm 0.001$       & 0.014, 0.014            & $0.014 \pm 0.000$       & 0.014, 0.014            & $0.014 \pm 0.000$       \\ \hline
\multicolumn{1}{|c|}{\textbf{DTLZ2}} & 0.027, 0.027            & $0.027 \pm 0.000$       & 0.030, 0.036            & $0.032 \pm 0.001$       & 0.023, 0.024            & $0.023 \pm 0.000$       & 0.024, 0.025            & $0.024 \pm 0.000$       \\ \hline
\multicolumn{1}{|c|}{\textbf{DTLZ3}} & 0.027, 0.027            & $0.027 \pm 0.000$       & 0.027, 0.032            & $0.030 \pm 0.001$       & 0.024, 0.025            & $0.025 \pm 0.000$       & 0.024, 0.025            & $0.024 \pm 0.000$       \\ \hline
\multicolumn{1}{|c|}{\textbf{DTLZ4}} & 0.027, 0.595            & $0.092 \pm 0.181$       & 0.028, 0.036            & $0.032 \pm 0.001$       & 0.023, 0.595            & $0.179 \pm 0.165$       & 0.024, 0.025            & $0.024 \pm 0.000$       \\ \hline
\multicolumn{1}{|c|}{\textbf{DTLZ5}} & 0.003, 0.003            & $0.003 \pm 0.000$       & 0.003, 0.003            & $0.003 \pm 0.000$       & 0.002, 0.002            & $0.002 \pm 0.000$       & 0.002, 0.002            & $0.002 \pm 0.000$       \\ \hline
\multicolumn{1}{|c|}{\textbf{DTLZ6}} & 0.022, 0.163            & $0.087 \pm 0.032$       & 0.126, 0.224            & $0.187 \pm 0.027$       & 0.002, 0.002            & $0.002 \pm 0.000$       & 0.002, 0.002            & $0.002 \pm 0.000$       \\ \hline
\multicolumn{1}{|c|}{\textbf{DTLZ7}} & 0.045, 0.045            & $0.045 \pm 0.000$       & 0.038, 0.052            & $0.044 \pm 0.003$       & 0.059, 0.693            & $0.283 \pm 0.194$       & 0.027, 0.029            & $0.028 \pm 0.000$       \\ \hline
\multicolumn{1}{|c|}{\textbf{UF8}}   & 0.048, 0.365            & $0.069 \pm 0.051$       & 0.093, 0.220            & $0.178 \pm 0.031$       & 0.025, 0.033            & $0.028 \pm 0.002$       & 0.026, 0.034            & $0.029 \pm 0.002$       \\ \hline
\multicolumn{1}{|c|}{\textbf{UF9}}   & 0.041, 0.151            & $0.086 \pm 0.049$       & 0.106, 0.314            & $0.139 \pm 0.049$       & 0.023, 0.137            & $0.042 \pm 0.039$       & 0.022, 0.030            & $0.025 \pm 0.002$       \\ \hline
\multicolumn{1}{|c|}{\textbf{UF10}}  & 0.163, 0.565            & $0.294 \pm 0.125$       & 0.198, 0.658            & $0.261 \pm 0.080$       & 0.117, 0.558            & $0.281 \pm 0.122$       & 0.061, 0.168            & $0.099 \pm 0.026$       \\ \hline
\multicolumn{1}{|c|}{\textbf{Mean}}  & \textbf{0.078, 0.170}   & \textbf{0.099}          & \textbf{0.0972, 0.166}  & \textbf{0.121}          & \textbf{0.130, 0.233}   & \textbf{0.165}          & \textbf{0.053, 0.068}   & \textbf{0.0582}         \\ \hline
\end{tabular}%
}
\end{table*}



