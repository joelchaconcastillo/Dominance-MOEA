% Please add the following required packages to your document preamble:
% \usepackage{graphicx}
\begin{table*}[t]
\centering
\caption{Statistics IGD+ with two objectives}
\label{tab:StatisticsIGDP_2obj}
%\resizebox{\textwidth}{!}{%
\begin{tabular}{cc|c|c|c|c|c|c|c|c|c|c|c|c|c|c|c}
\cline{2-17}
 & \multicolumn{4}{c|}{\textbf{MOEA/D}} & \multicolumn{4}{c|}{\textbf{NSGA-II}} & \multicolumn{4}{c|}{\textbf{R2-EMOA}} & \multicolumn{4}{c}{\textbf{VSD-MOEA}} \\ \cline{2-17} 
 & \textbf{Min} & \textbf{Max} & \textbf{Mean} & \textbf{Std} & \textbf{Min} & \textbf{Max} & \textbf{Mean} & \textbf{Std} & \textbf{Min} & \textbf{Max} & \textbf{Mean} & \textbf{Std} & \textbf{Min} & \textbf{Max} & \textbf{Mean} & \textbf{Std} \\ \hline
\multicolumn{1}{c|}{\textbf{WFG1}} & 0.006 & 0.015 & 0.008 & 0.002 & 0.006 & 0.014 & 0.008 & 0.002 & 0.006 & 0.061 & 0.013 & 0.014 & 0.006 & 0.019 & 0.008 & 0.003 \\ \hline
\multicolumn{1}{c|}{\textbf{WFG2}} & 0.006 & 0.055 & 0.052 & 0.011 & 0.003 & 0.053 & 0.040 & 0.022 & 0.053 & 0.055 & 0.054 & 0.000 & 0.003 & 0.003 & 0.003 & 0.000 \\ \hline
\multicolumn{1}{c|}{\textbf{WFG3}} & 0.008 & 0.008 & 0.008 & 0.000 & 0.011 & 0.013 & 0.012 & 0.000 & 0.008 & 0.009 & 0.008 & 0.000 & 0.007 & 0.007 & 0.007 & 0.000 \\ \hline
\multicolumn{1}{c|}{\textbf{WFG4}} & 0.007 & 0.007 & 0.007 & 0.000 & 0.007 & 0.010 & 0.008 & 0.001 & 0.005 & 0.005 & 0.005 & 0.000 & 0.006 & 0.006 & 0.006 & 0.000 \\ \hline
\multicolumn{1}{c|}{\textbf{WFG5}} & 0.060 & 0.069 & 0.065 & 0.002 & 0.060 & 0.068 & 0.066 & 0.002 & 0.064 & 0.066 & 0.065 & 0.000 & 0.038 & 0.057 & 0.047 & 0.006 \\ \hline
\multicolumn{1}{c|}{\textbf{WFG6}} & 0.034 & 0.073 & 0.050 & 0.010 & 0.034 & 0.064 & 0.051 & 0.007 & 0.034 & 0.076 & 0.053 & 0.010 & 0.068 & 0.088 & 0.081 & 0.004 \\ \hline
\multicolumn{1}{c|}{\textbf{WFG7}} & 0.007 & 0.007 & 0.007 & 0.000 & 0.008 & 0.010 & 0.009 & 0.000 & 0.005 & 0.006 & 0.005 & 0.000 & 0.006 & 0.006 & 0.006 & 0.000 \\ \hline
\multicolumn{1}{c|}{\textbf{WFG8}} & 0.103 & 0.120 & 0.112 & 0.005 & 0.116 & 0.139 & 0.125 & 0.005 & 0.103 & 0.120 & 0.110 & 0.004 & 0.026 & 0.099 & 0.043 & 0.025 \\ \hline
\multicolumn{1}{c|}{\textbf{WFG9}} & 0.011 & 0.125 & 0.067 & 0.053 & 0.014 & 0.127 & 0.101 & 0.046 & 0.009 & 0.125 & 0.067 & 0.053 & 0.009 & 0.014 & 0.011 & 0.001 \\ \hline
\multicolumn{1}{c|}{\textbf{DTLZ1}} & 0.001 & 0.001 & 0.001 & 0.000 & 0.002 & 0.002 & 0.002 & 0.000 & 0.001 & 0.001 & 0.001 & 0.000 & 0.001 & 0.001 & 0.001 & 0.000 \\ \hline
\multicolumn{1}{c|}{\textbf{DTLZ2}} & 0.002 & 0.002 & 0.002 & 0.000 & 0.002 & 0.003 & 0.003 & 0.000 & 0.002 & 0.002 & 0.002 & 0.000 & 0.002 & 0.002 & 0.002 & 0.000 \\ \hline
\multicolumn{1}{c|}{\textbf{DTLZ3}} & 0.002 & 0.002 & 0.002 & 0.000 & 0.002 & 0.003 & 0.002 & 0.000 & 0.002 & 0.002 & 0.002 & 0.000 & 0.002 & 0.002 & 0.002 & 0.000 \\ \hline
\multicolumn{1}{c|}{\textbf{DTLZ4}} & 0.002 & 0.363 & 0.105 & 0.163 & 0.002 & 0.363 & 0.064 & 0.136 & 0.002 & 0.363 & 0.167 & 0.180 & 0.002 & 0.002 & 0.002 & 0.000 \\ \hline
\multicolumn{1}{c|}{\textbf{DTLZ5}} & 0.002 & 0.002 & 0.002 & 0.000 & 0.002 & 0.003 & 0.003 & 0.000 & 0.002 & 0.002 & 0.002 & 0.000 & 0.002 & 0.002 & 0.002 & 0.000 \\ \hline
\multicolumn{1}{c|}{\textbf{DTLZ6}} & 0.022 & 0.149 & 0.076 & 0.027 & 0.126 & 0.315 & 0.205 & 0.036 & 0.019 & 0.128 & 0.078 & 0.027 & 0.002 & 0.002 & 0.002 & 0.000 \\ \hline
\multicolumn{1}{c|}{\textbf{DTLZ7}} & 0.003 & 0.003 & 0.003 & 0.000 & 0.002 & 0.003 & 0.003 & 0.000 & 0.002 & 0.002 & 0.002 & 0.000 & 0.003 & 0.003 & 0.003 & 0.000 \\ \hline
\multicolumn{1}{c|}{\textbf{UF1}} & 0.004 & 0.004 & 0.004 & 0.000 & 0.005 & 0.006 & 0.006 & 0.000 & 0.003 & 0.005 & 0.004 & 0.001 & 0.003 & 0.003 & 0.003 & 0.000 \\ \hline
\multicolumn{1}{c|}{\textbf{UF2}} & 0.003 & 0.005 & 0.004 & 0.000 & 0.008 & 0.010 & 0.010 & 0.000 & 0.004 & 0.006 & 0.005 & 0.001 & 0.004 & 0.007 & 0.005 & 0.001 \\ \hline
\multicolumn{1}{c|}{\textbf{UF3}} & 0.141 & 0.237 & 0.180 & 0.022 & 0.052 & 0.127 & 0.084 & 0.020 & 0.119 & 0.210 & 0.183 & 0.021 & 0.038 & 0.095 & 0.057 & 0.013 \\ \hline
\multicolumn{1}{c|}{\textbf{UF4}} & 0.024 & 0.031 & 0.026 & 0.001 & 0.027 & 0.039 & 0.033 & 0.003 & 0.019 & 0.023 & 0.021 & 0.001 & 0.020 & 0.024 & 0.022 & 0.001 \\ \hline
\multicolumn{1}{c|}{\textbf{UF5}} & 0.079 & 0.593 & 0.265 & 0.120 & 0.091 & 0.254 & 0.142 & 0.033 & 0.079 & 0.521 & 0.215 & 0.131 & 0.088 & 0.154 & 0.132 & 0.014 \\ \hline
\multicolumn{1}{c|}{\textbf{UF6}} & 0.066 & 0.529 & 0.380 & 0.108 & 0.037 & 0.542 & 0.193 & 0.114 & 0.064 & 0.432 & 0.266 & 0.103 & 0.021 & 0.065 & 0.038 & 0.011 \\ \hline
\multicolumn{1}{c|}{\textbf{UF7}} & 0.003 & 0.005 & 0.004 & 0.000 & 0.007 & 0.008 & 0.007 & 0.000 & 0.003 & 0.242 & 0.046 & 0.082 & 0.003 & 0.009 & 0.004 & 0.001 \\ \hline
\multicolumn{1}{c|}{\textbf{Mean}} & 0.026 & 0.105 & 0.062 & 0.023 & 0.027 & 0.095 & 0.051 & 0.019 & 0.026 & 0.107 & 0.060 & 0.027 & 0.016 & 0.029 & 0.021 & 0.003 \\ \hline
\end{tabular}%
%}
\end{table*}



%%% Please add the following required packages to your document preamble:
%%% \usepackage{graphicx}
%%
%%\begin{table*}[t]
%%\caption{Statistics IGD+ with two objectives}
%%\label{tab:StatisticsIGDP_2obj}
%%%\resizebox{\textwidth}{!}{%
%%\begin{tabular}{c|c|c|c|c|c|c|c|c|c|c|c|c|c|c|c|c|}
%%\cline{2-17}
%% & \multicolumn{4}{c|}{\textbf{MOEA/D}} & \multicolumn{4}{c|}{\textbf{NSGA-II}} & \multicolumn{4}{c|}{\textbf{R2-MOEA}} & \multicolumn{4}{c|}{\textbf{VSD-MOEA}} \\ \cline{2-17} 
%% & \textbf{Min} & \textbf{Max} & \textbf{Mean} & \textbf{Std} & \textbf{Min} & \textbf{Max} & \textbf{Mean} & \textbf{Std} & \textbf{Min} & \textbf{Max} & \textbf{Mean} & \textbf{Std} & \textbf{Min} & \textbf{Max} & \textbf{Mean} & \textbf{Std} \\ \hline
%%\multicolumn{1}{|c|}{\textbf{WFG1}} & 0.006 & 0.015 & 0.008 & 0.002 & 0.006 & 0.014 & 0.008 & 0.002 & 0.006 & 0.061 & 0.013 & 0.014 & 0.006 & 0.025 & 0.007 & 0.003 \\ \hline
%%\multicolumn{1}{|c|}{\textbf{WFG2}} & 0.006 & 0.055 & 0.052 & 0.011 & 0.003 & 0.053 & 0.040 & 0.022 & 0.053 & 0.055 & 0.054 & 0.000 & 0.003 & 0.003 & 0.003 & 0.000 \\ \hline
%%\multicolumn{1}{|c|}{\textbf{WFG3}} & 0.008 & 0.008 & 0.008 & 0.000 & 0.011 & 0.013 & 0.012 & 0.000 & 0.008 & 0.009 & 0.008 & 0.000 & 0.007 & 0.007 & 0.007 & 0.000 \\ \hline
%%\multicolumn{1}{|c|}{\textbf{WFG4}} & 0.007 & 0.007 & 0.007 & 0.000 & 0.007 & 0.010 & 0.008 & 0.001 & 0.005 & 0.005 & 0.005 & 0.000 & 0.006 & 0.006 & 0.006 & 0.000 \\ \hline
%%\multicolumn{1}{|c|}{\textbf{WFG5}} & 0.060 & 0.069 & 0.065 & 0.002 & 0.060 & 0.068 & 0.066 & 0.002 & 0.064 & 0.066 & 0.065 & 0.000 & 0.033 & 0.053 & 0.040 & 0.005 \\ \hline
%%\multicolumn{1}{|c|}{\textbf{WFG6}} & 0.034 & 0.073 & 0.050 & 0.010 & 0.034 & 0.064 & 0.051 & 0.007 & 0.034 & 0.076 & 0.053 & 0.010 & 0.068 & 0.090 & 0.081 & 0.005 \\ \hline
%%\multicolumn{1}{|c|}{\textbf{WFG7}} & 0.007 & 0.007 & 0.007 & 0.000 & 0.008 & 0.010 & 0.009 & 0.000 & 0.005 & 0.006 & 0.005 & 0.000 & 0.006 & 0.006 & 0.006 & 0.000 \\ \hline
%%\multicolumn{1}{|c|}{\textbf{WFG8}} & 0.103 & 0.120 & 0.112 & 0.005 & 0.116 & 0.139 & 0.125 & 0.005 & 0.103 & 0.120 & 0.110 & 0.004 & 0.024 & 0.035 & 0.029 & 0.003 \\ \hline
%%\multicolumn{1}{|c|}{\textbf{WFG9}} & 0.011 & 0.125 & 0.067 & 0.053 & 0.014 & 0.127 & 0.101 & 0.046 & 0.009 & 0.125 & 0.067 & 0.053 & 0.009 & 0.015 & 0.011 & 0.001 \\ \hline
%%\multicolumn{1}{|c|}{\textbf{DTLZ1}} & 0.001 & 0.001 & 0.001 & 0.000 & 0.002 & 0.002 & 0.002 & 0.000 & 0.001 & 0.001 & 0.001 & 0.000 & 0.001 & 0.001 & 0.001 & 0.000 \\ \hline
%%\multicolumn{1}{|c|}{\textbf{DTLZ2}} & 0.002 & 0.002 & 0.002 & 0.000 & 0.002 & 0.003 & 0.003 & 0.000 & 0.002 & 0.002 & 0.002 & 0.000 & 0.002 & 0.002 & 0.002 & 0.000 \\ \hline
%%\multicolumn{1}{|c|}{\textbf{DTLZ3}} & 0.002 & 0.002 & 0.002 & 0.000 & 0.002 & 0.003 & 0.002 & 0.000 & 0.002 & 0.002 & 0.002 & 0.000 & 0.002 & 0.002 & 0.002 & 0.000 \\ \hline
%%\multicolumn{1}{|c|}{\textbf{DTLZ4}} & 0.002 & 0.363 & 0.105 & 0.163 & 0.002 & 0.363 & 0.064 & 0.136 & 0.002 & 0.363 & 0.167 & 0.180 & 0.002 & 0.002 & 0.002 & 0.000 \\ \hline
%%\multicolumn{1}{|c|}{\textbf{DTLZ5}} & 0.002 & 0.002 & 0.002 & 0.000 & 0.002 & 0.003 & 0.003 & 0.000 & 0.002 & 0.002 & 0.002 & 0.000 & 0.002 & 0.002 & 0.002 & 0.000 \\ \hline
%%\multicolumn{1}{|c|}{\textbf{DTLZ6}} & 0.022 & 0.149 & 0.076 & 0.027 & 0.126 & 0.315 & 0.205 & 0.036 & 0.019 & 0.128 & 0.078 & 0.027 & 0.002 & 0.002 & 0.002 & 0.000 \\ \hline
%%\multicolumn{1}{|c|}{\textbf{DTLZ7}} & 0.003 & 0.003 & 0.003 & 0.000 & 0.002 & 0.003 & 0.003 & 0.000 & 0.002 & 0.002 & 0.002 & 0.000 & 0.003 & 0.003 & 0.003 & 0.000 \\ \hline
%%\multicolumn{1}{|c|}{\textbf{UF1}} & 0.004 & 0.004 & 0.004 & 0.000 & 0.005 & 0.006 & 0.006 & 0.000 & 0.003 & 0.005 & 0.004 & 0.001 & 0.003 & 0.003 & 0.003 & 0.000 \\ \hline
%%\multicolumn{1}{|c|}{\textbf{UF2}} & 0.003 & 0.005 & 0.004 & 0.000 & 0.008 & 0.010 & 0.010 & 0.000 & 0.004 & 0.006 & 0.005 & 0.001 & 0.005 & 0.008 & 0.006 & 0.001 \\ \hline
%%\multicolumn{1}{|c|}{\textbf{UF3}} & 0.141 & 0.237 & 0.180 & 0.022 & 0.052 & 0.127 & 0.084 & 0.020 & 0.119 & 0.210 & 0.183 & 0.021 & 0.043 & 0.077 & 0.052 & 0.006 \\ \hline
%%\multicolumn{1}{|c|}{\textbf{UF4}} & 0.024 & 0.031 & 0.026 & 0.001 & 0.027 & 0.039 & 0.033 & 0.003 & 0.019 & 0.023 & 0.021 & 0.001 & 0.021 & 0.024 & 0.022 & 0.001 \\ \hline
%%\multicolumn{1}{|c|}{\textbf{UF5}} & 0.079 & 0.593 & 0.265 & 0.120 & 0.091 & 0.254 & 0.142 & 0.033 & 0.079 & 0.521 & 0.215 & 0.131 & 0.083 & 0.145 & 0.118 & 0.015 \\ \hline
%%\multicolumn{1}{|c|}{\textbf{UF6}} & 0.066 & 0.529 & 0.380 & 0.108 & 0.037 & 0.542 & 0.193 & 0.114 & 0.064 & 0.432 & 0.266 & 0.103 & 0.019 & 0.034 & 0.026 & 0.005 \\ \hline
%%\multicolumn{1}{|c|}{\textbf{UF7}} & 0.003 & 0.005 & 0.004 & 0.000 & 0.007 & 0.008 & 0.007 & 0.000 & 0.003 & 0.242 & 0.046 & 0.082 & 0.003 & 0.005 & 0.004 & 0.000 \\ \hline
%%\multicolumn{1}{|c|}{\textbf{Mean}} & \textbf{0.026} & \textbf{0.105} & \textbf{0.062} & \textbf{0.023} & \textbf{0.027} & \textbf{0.095} & \textbf{0.051} & \textbf{0.019} & \textbf{0.026} & \textbf{0.107} & \textbf{0.060} & \textbf{0.027} & \textbf{0.015} & \textbf{0.024} & \textbf{0.019} & \textbf{0.002} \\ \hline
%%\end{tabular}%
%%%}
%%\end{table*}
%%

%% Please add the following required packages to your document preamble:
%% \usepackage{graphicx}
%\begin{table*}[t]
%\caption{Statistical Tests of IGD+ with Two Objectives}
%\label{tab:Tests_IGDP_2obj}
%\centering
%%\resizebox{\textwidth}{!}{%
%\begin{tabular}{c|c|c|c|c|c|c|c|c|c|c|c|c|c|c|c|c|}
%\cline{2-17}
%\textbf{} & \multicolumn{4}{c|}{\textbf{MOEA/D}} & \multicolumn{4}{c|}{\textbf{NSGA-II}} & \multicolumn{4}{c|}{\textbf{R2-MOEA}} & \multicolumn{4}{c|}{\textbf{VSD-MOEA}} \\ \cline{2-17} 
% & \textbf{$\uparrow$} & \textbf{$\downarrow$} & \textbf{$\leftrightarrow$} & \textbf{Diff} & \textbf{$\uparrow$} & \textbf{$\downarrow$} & \textbf{$\leftrightarrow$} & \textbf{Diff} & \textbf{$\uparrow$} & \textbf{$\downarrow$} & \textbf{$\leftrightarrow$} & \textbf{Diff} & \textbf{$\uparrow$} & \textbf{$\downarrow$} & \textbf{$\leftrightarrow$} & \textbf{Diff} \\ \hline
%\multicolumn{1}{|c|}{\textbf{WFG1}} & 1 & 1 & 1 & 0.000 & 0 & 1 & 2 & 0.001 & 0 & 2 & 1 & 0.006 & 3 & 0 & 0 & 0.000 \\ \hline
%\multicolumn{1}{|c|}{\textbf{WFG2}} & 1 & 2 & 0 & 0.049 & 2 & 1 & 0 & 0.037 & 0 & 3 & 0 & 0.051 & 3 & 0 & 0 & 0.000 \\ \hline
%\multicolumn{1}{|c|}{\textbf{WFG3}} & 2 & 1 & 0 & 0.000 & 0 & 3 & 0 & 0.005 & 1 & 2 & 0 & 0.001 & 3 & 0 & 0 & 0.000 \\ \hline
%\multicolumn{1}{|c|}{\textbf{WFG4}} & 1 & 2 & 0 & 0.001 & 0 & 3 & 0 & 0.003 & 3 & 0 & 0 & 0.000 & 2 & 1 & 0 & 0.001 \\ \hline
%\multicolumn{1}{|c|}{\textbf{WFG5}} & 0 & 1 & 2 & 0.026 & 0 & 2 & 1 & 0.026 & 1 & 1 & 1 & 0.025 & 3 & 0 & 0 & 0.000 \\ \hline
%\multicolumn{1}{|c|}{\textbf{WFG6}} & 1 & 0 & 2 & 0.000 & 1 & 0 & 2 & 0.001 & 1 & 0 & 2 & 0.002 & 0 & 3 & 0 & 0.030 \\ \hline
%\multicolumn{1}{|c|}{\textbf{WFG7}} & 1 & 2 & 0 & 0.001 & 0 & 3 & 0 & 0.003 & 3 & 0 & 0 & 0.000 & 2 & 1 & 0 & 0.001 \\ \hline
%\multicolumn{1}{|c|}{\textbf{WFG8}} & 1 & 1 & 1 & 0.083 & 0 & 3 & 0 & 0.096 & 1 & 1 & 1 & 0.082 & 3 & 0 & 0 & 0.000 \\ \hline
%\multicolumn{1}{|c|}{\textbf{WFG9}} & 1 & 1 & 1 & 0.056 & 0 & 3 & 0 & 0.090 & 1 & 1 & 1 & 0.055 & 3 & 0 & 0 & 0.000 \\ \hline
%\multicolumn{1}{|c|}{\textbf{DTLZ1}} & 3 & 0 & 0 & 0.000 & 0 & 3 & 0 & 0.000 & 1 & 1 & 1 & 0.000 & 1 & 1 & 1 & 0.000 \\ \hline
%\multicolumn{1}{|c|}{\textbf{DTLZ2}} & 1 & 2 & 0 & 0.000 & 0 & 3 & 0 & 0.001 & 3 & 0 & 0 & 0.000 & 2 & 1 & 0 & 0.000 \\ \hline
%\multicolumn{1}{|c|}{\textbf{DTLZ3}} & 1 & 2 & 0 & 0.000 & 0 & 3 & 0 & 0.001 & 3 & 0 & 0 & 0.000 & 2 & 1 & 0 & 0.000 \\ \hline
%\multicolumn{1}{|c|}{\textbf{DTLZ4}} & 0 & 2 & 1 & 0.103 & 1 & 1 & 1 & 0.062 & 0 & 0 & 3 & 0.165 & 2 & 0 & 1 & 0.000 \\ \hline
%\multicolumn{1}{|c|}{\textbf{DTLZ5}} & 1 & 2 & 0 & 0.000 & 0 & 3 & 0 & 0.001 & 3 & 0 & 0 & 0.000 & 2 & 1 & 0 & 0.000 \\ \hline
%\multicolumn{1}{|c|}{\textbf{DTLZ6}} & 1 & 1 & 1 & 0.073 & 0 & 3 & 0 & 0.203 & 1 & 1 & 1 & 0.076 & 3 & 0 & 0 & 0.000 \\ \hline
%\multicolumn{1}{|c|}{\textbf{DTLZ7}} & 0 & 3 & 0 & 0.001 & 1 & 1 & 1 & 0.001 & 3 & 0 & 0 & 0.000 & 1 & 1 & 1 & 0.001 \\ \hline
%\multicolumn{1}{|c|}{\textbf{UF1}} & 1 & 1 & 1 & 0.001 & 0 & 3 & 0 & 0.003 & 1 & 1 & 1 & 0.001 & 3 & 0 & 0 & 0.000 \\ \hline
%\multicolumn{1}{|c|}{\textbf{UF2}} & 3 & 0 & 0 & 0.000 & 0 & 3 & 0 & 0.006 & 2 & 1 & 0 & 0.001 & 1 & 2 & 0 & 0.003 \\ \hline
%\multicolumn{1}{|c|}{\textbf{UF3}} & 0 & 2 & 1 & 0.129 & 2 & 1 & 0 & 0.032 & 0 & 2 & 1 & 0.131 & 3 & 0 & 0 & 0.000 \\ \hline
%\multicolumn{1}{|c|}{\textbf{UF4}} & 1 & 2 & 0 & 0.004 & 0 & 3 & 0 & 0.012 & 3 & 0 & 0 & 0.000 & 2 & 1 & 0 & 0.001 \\ \hline
%\multicolumn{1}{|c|}{\textbf{UF5}} & 0 & 3 & 0 & 0.147 & 1 & 1 & 1 & 0.024 & 1 & 1 & 1 & 0.096 & 3 & 0 & 0 & 0.000 \\ \hline
%\multicolumn{1}{|c|}{\textbf{UF6}} & 0 & 3 & 0 & 0.354 & 2 & 1 & 0 & 0.168 & 1 & 2 & 0 & 0.240 & 3 & 0 & 0 & 0.000 \\ \hline
%\multicolumn{1}{|c|}{\textbf{UF7}} & 2 & 0 & 1 & 0.000 & 1 & 2 & 0 & 0.003 & 0 & 3 & 0 & 0.042 & 2 & 0 & 1 & 0.000 \\ \hline
%\multicolumn{1}{|c|}{\textbf{Total}} & \textbf{23} & \textbf{34} & \textbf{12} & \textbf{1.033} & \textbf{11} & \textbf{50} & \textbf{8} & \textbf{0.779} & \textbf{33} & \textbf{22} & \textbf{14} & \textbf{0.976} & \textbf{52} & \textbf{13} & \textbf{4} & \textbf{0.036} \\ \hline
%\end{tabular}%
%%}
%\end{table*}
%

% Please add the following required packages to your document preamble:
% \usepackage{graphicx}
\begin{table}[t]
\centering
\caption{Statistical Tests and Deterioration Level of the IGD+ for three objectives}
\label{tab:Tests_HV_3obj}
%\resizebox{\textwidth}{!}{%
\begin{tabular}{c c|c|c|c}
\cline{2-5}
                                        & \textbf{$\uparrow$} & \textbf{$\downarrow$} & \textbf{$\leftrightarrow$} & \textbf{Deterioration} \\ \hline
\multicolumn{1}{c|}{\textbf{MOEA/D}}   & 15                  & 37                    & 5                          & 0.787         \\ \hline
\multicolumn{1}{c|}{\textbf{NSGA-II}}  & 6                   & 46                    & 5                          & 1.214         \\ \hline
\multicolumn{1}{c|}{\textbf{R2-EMOA}}  & 35                  & 16                    & 6                          & 0.669         \\ \hline
\multicolumn{1}{c|}{\textbf{VSD-MOEA}} & 49                  & 6                     & 2                          & 0.039         \\ \hline
\end{tabular}%
%}
\end{table}

%%% Please add the following required packages to your document preamble:
%%% \usepackage{graphicx}
%%\begin{table*}[t]
%%\caption{Statistical Tests of IGD+ with Three Objectives}
%%\label{tab:Tests_IGDP_3obj}
%%\centering
%%%\resizebox{\textwidth}{!}{%
%%\begin{tabular}{c|c|c|c|c|c|c|c|c|c|c|c|c|c|c|c|c|}
%%\cline{2-17}
%%\textbf{} & \multicolumn{4}{c|}{\textbf{MOEA/D}} & \multicolumn{4}{c|}{\textbf{NSGA-II}} & \multicolumn{4}{c|}{\textbf{R2-MOEA}} & \multicolumn{4}{c|}{\textbf{VSD-MOEA}} \\ \cline{2-17} 
%% & \textbf{$\uparrow$} & \textbf{$\downarrow$} & \textbf{$\leftrightarrow$} & \textbf{Diff} & \textbf{$\uparrow$} & \textbf{$\downarrow$} & \textbf{$\leftrightarrow$} & \textbf{Diff} & \textbf{$\uparrow$} & \textbf{$\downarrow$} & \textbf{$\leftrightarrow$} & \textbf{Diff} & \textbf{$\uparrow$} & \textbf{$\downarrow$} & \textbf{$\leftrightarrow$} & \textbf{Diff} \\ \hline
%%\multicolumn{1}{|c|}{\textbf{WFG1}} & 1 & 2 & 0 & 0.034 & 0 & 3 & 0 & 0.104 & 2 & 1 & 0 & 0.022 & 3 & 0 & 0 & 0.000 \\ \hline
%%\multicolumn{1}{|c|}{\textbf{WFG2}} & 2 & 1 & 0 & 0.025 & 1 & 2 & 0 & 0.060 & 0 & 3 & 0 & 0.065 & 3 & 0 & 0 & 0.000 \\ \hline
%%\multicolumn{1}{|c|}{\textbf{WFG3}} & 2 & 1 & 0 & 0.000 & 0 & 3 & 0 & 0.017 & 3 & 0 & 0 & 0.000 & 1 & 2 & 0 & 0.010 \\ \hline
%%\multicolumn{1}{|c|}{\textbf{WFG4}} & 1 & 2 & 0 & 0.034 & 0 & 3 & 0 & 0.039 & 2 & 1 & 0 & 0.004 & 3 & 0 & 0 & 0.000 \\ \hline
%%\multicolumn{1}{|c|}{\textbf{WFG5}} & 0 & 3 & 0 & 0.034 & 1 & 2 & 0 & 0.023 & 2 & 1 & 0 & 0.006 & 3 & 0 & 0 & 0.000 \\ \hline
%%\multicolumn{1}{|c|}{\textbf{WFG6}} & 0 & 2 & 1 & 0.034 & 0 & 2 & 1 & 0.037 & 3 & 0 & 0 & 0.000 & 2 & 1 & 0 & 0.022 \\ \hline
%%\multicolumn{1}{|c|}{\textbf{WFG7}} & 0 & 3 & 0 & 0.033 & 1 & 2 & 0 & 0.030 & 2 & 1 & 0 & 0.004 & 3 & 0 & 0 & 0.000 \\ \hline
%%\multicolumn{1}{|c|}{\textbf{WFG8}} & 1 & 2 & 0 & 0.086 & 0 & 3 & 0 & 0.150 & 2 & 1 & 0 & 0.058 & 3 & 0 & 0 & 0.000 \\ \hline
%%\multicolumn{1}{|c|}{\textbf{WFG9}} & 1 & 2 & 0 & 0.048 & 0 & 3 & 0 & 0.118 & 2 & 0 & 1 & 0.013 & 2 & 0 & 1 & 0.000 \\ \hline
%%\multicolumn{1}{|c|}{\textbf{DTLZ1}} & 1 & 2 & 0 & 0.001 & 0 & 3 & 0 & 0.004 & 3 & 0 & 0 & 0.000 & 2 & 1 & 0 & 0.000 \\ \hline
%%\multicolumn{1}{|c|}{\textbf{DTLZ2}} & 1 & 2 & 0 & 0.004 & 0 & 3 & 0 & 0.009 & 3 & 0 & 0 & 0.000 & 2 & 1 & 0 & 0.001 \\ \hline
%%\multicolumn{1}{|c|}{\textbf{DTLZ3}} & 1 & 2 & 0 & 0.004 & 0 & 3 & 0 & 0.007 & 3 & 0 & 0 & 0.000 & 2 & 1 & 0 & 0.001 \\ \hline
%%\multicolumn{1}{|c|}{\textbf{DTLZ4}} & 0 & 2 & 1 & 0.068 & 1 & 1 & 1 & 0.007 & 0 & 0 & 3 & 0.165 & 2 & 0 & 1 & 0.000 \\ \hline
%%\multicolumn{1}{|c|}{\textbf{DTLZ5}} & 0 & 3 & 0 & 0.001 & 1 & 2 & 0 & 0.001 & 2 & 1 & 0 & 0.000 & 3 & 0 & 0 & 0.000 \\ \hline
%%\multicolumn{1}{|c|}{\textbf{DTLZ6}} & 1 & 2 & 0 & 0.085 & 0 & 3 & 0 & 0.185 & 2 & 1 & 0 & 0.067 & 3 & 0 & 0 & 0.000 \\ \hline
%%\multicolumn{1}{|c|}{\textbf{DTLZ7}} & 1 & 1 & 1 & 0.017 & 1 & 1 & 1 & 0.016 & 0 & 3 & 0 & 0.051 & 3 & 0 & 0 & 0.000 \\ \hline
%%\multicolumn{1}{|c|}{\textbf{UF8}} & 1 & 2 & 0 & 0.040 & 0 & 3 & 0 & 0.149 & 2 & 0 & 1 & 0.004 & 2 & 0 & 1 & 0.000 \\ \hline
%%\multicolumn{1}{|c|}{\textbf{UF9}} & 1 & 1 & 1 & 0.062 & 0 & 2 & 1 & 0.114 & 0 & 1 & 2 & 0.069 & 3 & 0 & 0 & 0.000 \\ \hline
%%\multicolumn{1}{|c|}{\textbf{UF10}} & 0 & 2 & 1 & 0.195 & 0 & 2 & 1 & 0.163 & 2 & 1 & 0 & 0.158 & 3 & 0 & 0 & 0.000 \\ \hline
%%\multicolumn{1}{|c|}{\textbf{Total}} & \textbf{15} & \textbf{37} & \textbf{5} & \textbf{0.806} & \textbf{6} & \textbf{46} & \textbf{5} & \textbf{1.232} & \textbf{35} & \textbf{15} & \textbf{7} & \textbf{0.687} & \textbf{48} & \textbf{6} & \textbf{3} & \textbf{0.036} \\ \hline
%%\end{tabular}%
%%%}
%%\end{table*}
%%


This section presents the results obtained by \VSDMOEA{} and state-of-the-art schemes in terms of
the IGD+\cite{Joel:Inverted_Generational_Distance_Plus}.
%
Specifically, we present the results for the long-term executions, meaning
the stopping criterion was set to $250,000$ generations.
%
The structure of the tables is the same as in the main document.
%
Thus, the only modification is that instead of using the hypervolume, the IGD+ is used.

Table \ref{tab:StatisticsIGDP_2obj} shows the IGD+ obtained for the benchmark functions with two objectives.
%
Specifically, the minimum, maximum, mean and standard deviation of the IGD+ is given for each method and function tested.
%
The last row shows the results considering all the functions together.
%
For each function, the data for the method that yielded the lowest mean is shown in bold.
%
Additionally, all the methods that were not statistically inferior to said method are shown in bold.
%
From here on, the methods shown in bold for a given problem are referred to as the winning methods.
%
Based on the number of functions where each method is in the group of the winning methods for the cases 
with two objectives, the best methods are \VSDMOEA{} and \RMOEA{} with 13 and 8, respectively.
%
Thus, \VSDMOEA{} is the most competitive method in terms of this metric.
%
More impressive is the fact that the mean IGD+ attained by \VSDMOEA{}, when all the problems are considered simultaneously, is much lower 
than that attained by \RMOEA{}.
%
In fact, the total means of \RMOEA{} ($0.060$), \NSGAII{} ($0.051$) and \MOEAD{} ($0.062$) are quite similar.
%
In contrast, \VSDMOEA{} yielded a much lower value ($0.021$).
%
When the data is inspected carefully, it is clear that in the cases where \VSDMOEA{} loses, the difference with respect to the
best method is not very large.
%
For instance, the difference between the mean IGD+ attained by \VSDMOEA{} and by the best method was never larger
than $0.05$.
%
However, all the other methods exhibited a deterioration greater than $0.05$ in several cases.
%
Specifically, it happened in $7$, $5$ and $8$ problems for \MOEAD{}, \NSGAII{} and \RMOEA{}, respectively.
%
This means that even if \VSDMOEA{} loses in some cases, its deterioration is always small, exhibiting a much more 
robust behavior than any other method.
%
Exactly the same situation appeared when analyzing the data in terms of hypervolume.

In order to better clarify these findings, pair-wise statistical tests were done between each method tested in each 
function.
%
Table~\ref{tab:Tests_IGDP_2obj} shows the results, with the same meaning as in the main document.
%
The calculated data confirms that although \VSDMOEA{} loses in some cases, the overall numbers of wins and losses favor \VSDMOEA{}.
%
More importantly, the total deterioration is considerably lower in the case of \VSDMOEA{}, confirming that when \VSDMOEA{} loses, the deterioration is not 
very high.


Tables~\ref{tab:Tests_IGDP_3obj} and~\ref{tab:StatisticsIGDP_3obj} show the same information for the problems with three objectives.
%
In this case, the superiority of \VSDMOEA{} is even clearer.
%
Taking into account the mean of all the functions, \VSDMOEA{} again yielded a much lower mean IGD+ than the other methods.
%
Specifically, \VSDMOEA{} obtained a value of $0.059$, whereas the second-ranked algorithm (\RMOEA{}) obtained a value of $0.093$.
%
Once again, the difference between the mean IGD+ obtained by \VSDMOEA{} and by the best method was never greater
than $0.05$.
%
However, all the other methods exhibited a deterioration greater than $0.05$ in several cases.
%
In particular, this happened in $5$, $8$ and $7$ problems for \MOEAD{}, \NSGAII{}, \RMOEA{}, respectively.
%
Moreover, \VSDMOEA{} is much more superior than the other methods not only in terms of total deterioration, but also
in terms of total wins and losses.
%
\VSDMOEA{} was in the group of winning methods for 14 out of 19 functions, whereas the second best-ranked algorithm (\RMOEA{})
was in the group of winning methods for only 5 functions.
%
These conclusions are again quite similar to those drawn for the hypervolume in the main document.




% Please add the following required packages to your document preamble:
% \usepackage{graphicx}
\begin{table*}[t]
\caption{Summary of the IGD+ results attained for problems with three objectives}
\label{tab:StatisticsIGDP_3obj}
\centering
%\resizebox{\textwidth}{!}{%
\begin{tabular}{cc|c|c|c|c|c|c|c|c|c|c|c|c|c|c|c}
\cline{2-17}
\textbf{}                           & \multicolumn{4}{c|}{\textbf{MOEA/D}}                       & \multicolumn{4}{c|}{\textbf{NSGA-II}}                      & \multicolumn{4}{c|}{\textbf{R2-EMOA}}                             & \multicolumn{4}{c}{\textbf{VSD-MOEA}}                            \\ \cline{2-17} 
                                    & \textbf{Min} & \textbf{Max} & \textbf{Mean} & \textbf{Std} & \textbf{Min} & \textbf{Max} & \textbf{Mean} & \textbf{Std} & \textbf{Min}   & \textbf{Max}   & \textbf{Mean}  & \textbf{Std}   & \textbf{Min}   & \textbf{Max}   & \textbf{Mean}  & \textbf{Std}   \\ \hline
\multicolumn{1}{c|}{\textbf{WFG1}}  & 0.080        & 0.100        & 0.090         & 0.005        & 0.142        & 0.179        & 0.160         & 0.010        & 0.058          & 0.098          & 0.079          & 0.010          & \textbf{0.049} & \textbf{0.070} & \textbf{0.058} & \textbf{0.006} \\ \hline
\multicolumn{1}{c|}{\textbf{WFG2}}  & 0.057        & 0.068        & 0.063         & 0.002        & 0.073        & 0.133        & 0.097         & 0.014        & 0.102          & 0.104          & 0.103          & 0.000          & \textbf{0.031} & \textbf{0.048} & \textbf{0.037} & \textbf{0.004} \\ \hline
\multicolumn{1}{c|}{\textbf{WFG3}}  & 0.023        & 0.023        & 0.023         & 0.000        & 0.031        & 0.061        & 0.039         & 0.005        & \textbf{0.022} & \textbf{0.023} & \textbf{0.022} & \textbf{0.000} & 0.033          & 0.033          & 0.033          & 0.000          \\ \hline
\multicolumn{1}{c|}{\textbf{WFG4}}  & 0.127        & 0.127        & 0.127         & 0.000        & 0.121        & 0.144        & 0.132         & 0.005        & 0.095          & 0.098          & 0.097          & 0.001          & \textbf{0.090} & \textbf{0.094} & \textbf{0.093} & \textbf{0.001} \\ \hline
\multicolumn{1}{c|}{\textbf{WFG5}}  & 0.177        & 0.184        & 0.181         & 0.002        & 0.160        & 0.186        & 0.170         & 0.005        & 0.147          & 0.158          & 0.153          & 0.003          & \textbf{0.140} & \textbf{0.150} & \textbf{0.146} & \textbf{0.003} \\ \hline
\multicolumn{1}{c|}{\textbf{WFG6}}  & 0.155        & 0.205        & 0.175         & 0.012        & 0.159        & 0.196        & 0.177         & 0.009        & \textbf{0.122} & \textbf{0.151} & \textbf{0.140} & \textbf{0.007} & 0.156          & 0.173          & 0.166          & 0.005          \\ \hline
\multicolumn{1}{c|}{\textbf{WFG7}}  & 0.127        & 0.127        & 0.127         & 0.000        & 0.113        & 0.138        & 0.123         & 0.007        & 0.094          & 0.102          & 0.097          & 0.001          & \textbf{0.092} & \textbf{0.094} & \textbf{0.094} & \textbf{0.001} \\ \hline
\multicolumn{1}{c|}{\textbf{WFG8}}  & 0.189        & 0.194        & 0.192         & 0.001        & 0.244        & 0.274        & 0.256         & 0.008        & 0.161          & 0.166          & 0.163          & 0.001          & \textbf{0.099} & \textbf{0.154} & \textbf{0.109} & \textbf{0.015} \\ \hline
\multicolumn{1}{c|}{\textbf{WFG9}}  & 0.130        & 0.240        & 0.154         & 0.036        & 0.138        & 0.246        & 0.224         & 0.025        & 0.099          & 0.211          & 0.119          & 0.037          & \textbf{0.099} & \textbf{0.210} & \textbf{0.118} & \textbf{0.036} \\ \hline
\multicolumn{1}{c|}{\textbf{DTLZ1}} & 0.014        & 0.014        & 0.014         & 0.000        & 0.017        & 0.020        & 0.018         & 0.001        & \textbf{0.013} & \textbf{0.014} & \textbf{0.014} & \textbf{0.000} & 0.014          & 0.014          & 0.014          & 0.000          \\ \hline
\multicolumn{1}{c|}{\textbf{DTLZ2}} & 0.027        & 0.027        & 0.027         & 0.000        & 0.030        & 0.036        & 0.032         & 0.001        & \textbf{0.023} & \textbf{0.024} & \textbf{0.023} & \textbf{0.000} & 0.024          & 0.025          & 0.024          & 0.000          \\ \hline
\multicolumn{1}{c|}{\textbf{DTLZ3}} & 0.027        & 0.027        & 0.027         & 0.000        & 0.027        & 0.032        & 0.030         & 0.001        & \textbf{0.023} & \textbf{0.023} & \textbf{0.023} & \textbf{0.000} & 0.024          & 0.025          & 0.024          & 0.000          \\ \hline
\multicolumn{1}{c|}{\textbf{DTLZ4}} & 0.027        & 0.595        & 0.092         & 0.181        & 0.028        & 0.036        & 0.032         & 0.001        & 0.023          & 0.595          & 0.190          & 0.225          & \textbf{0.024} & \textbf{0.025} & \textbf{0.024} & \textbf{0.000} \\ \hline
\multicolumn{1}{c|}{\textbf{DTLZ5}} & 0.003        & 0.003        & 0.003         & 0.000        & 0.003        & 0.003        & 0.003         & 0.000        & 0.002          & 0.002          & 0.002          & 0.000          & \textbf{0.002} & \textbf{0.002} & \textbf{0.002} & \textbf{0.000} \\ \hline
\multicolumn{1}{c|}{\textbf{DTLZ6}} & 0.022        & 0.163        & 0.087         & 0.032        & 0.126        & 0.224        & 0.187         & 0.027        & 0.003          & 0.136          & 0.069          & 0.033          & \textbf{0.002} & \textbf{0.002} & \textbf{0.002} & \textbf{0.000} \\ \hline
\multicolumn{1}{c|}{\textbf{DTLZ7}} & 0.045        & 0.045        & 0.045         & 0.000        & 0.038        & 0.052        & 0.044         & 0.003        & 0.060          & 0.087          & 0.079          & 0.008          & \textbf{0.027} & \textbf{0.029} & \textbf{0.028} & \textbf{0.000} \\ \hline
\multicolumn{1}{c|}{\textbf{UF8}}   & 0.048        & 0.365        & 0.069         & 0.051        & 0.093        & 0.220        & 0.178         & 0.031        & 0.027          & 0.159          & 0.033          & 0.022          & \textbf{0.025} & \textbf{0.034} & \textbf{0.029} & \textbf{0.002} \\ \hline
\multicolumn{1}{c|}{\textbf{UF9}}   & 0.041        & 0.151        & 0.086         & 0.049        & 0.106        & 0.314        & 0.139         & 0.049        & 0.025          & 0.137          & 0.094          & 0.053          & \textbf{0.022} & \textbf{0.028} & \textbf{0.024} & \textbf{0.001} \\ \hline
\multicolumn{1}{c|}{\textbf{UF10}}  & 0.163        & 0.565        & 0.294         & 0.125        & 0.198        & 0.658        & 0.261         & 0.080        & 0.159          & 0.553          & 0.257          & 0.131          & \textbf{0.070} & \textbf{0.187} & \textbf{0.103} & \textbf{0.026} \\ \hline
\multicolumn{1}{c|}{\textbf{Mean}}  & 0.078        & 0.170        & 0.099         & 0.026        & 0.097        & 0.166        & 0.121         & 0.015        & 0.066          & 0.150          & 0.093          & 0.028          & 0.054          & 0.074          & 0.059          & 0.005          \\ \hline
\end{tabular}%
%}
\end{table*}

%% Please add the following required packages to your document preamble:
%% \usepackage{graphicx}
%\begin{table*}[t]
%\caption{Statistics IGD+ with three objectives}
%\label{tab:StatisticsIGDP_3obj}
%%\resizebox{\textwidth}{!}{%
%\begin{tabular}{c|c|c|c|c|c|c|c|c|c|c|c|c|c|c|c|c|}
%\cline{2-17}
% & \multicolumn{4}{c|}{\textbf{MOEA/D}} & \multicolumn{4}{c|}{\textbf{NSGA-II}} & \multicolumn{4}{c|}{\textbf{R2-MOEA}} & \multicolumn{4}{c|}{\textbf{VSD-MOEA}} \\ \cline{2-17} 
% & \textbf{Min} & \textbf{Max} & \textbf{Mean} & \textbf{Std} & \textbf{Min} & \textbf{Max} & \textbf{Mean} & \textbf{Std} & \textbf{Min} & \textbf{Max} & \textbf{Mean} & \textbf{Std} & \textbf{Min} & \textbf{Max} & \textbf{Mean} & \textbf{Std} \\ \hline
%\multicolumn{1}{|c|}{\textbf{WFG1}} & 0.080 & 0.100 & 0.090 & 0.005 & 0.142 & 0.179 & 0.160 & 0.010 & 0.058 & 0.098 & 0.079 & 0.010 & 0.050 & 0.066 & 0.056 & 0.004 \\ \hline
%\multicolumn{1}{|c|}{\textbf{WFG2}} & 0.057 & 0.068 & 0.063 & 0.002 & 0.073 & 0.133 & 0.097 & 0.014 & 0.102 & 0.104 & 0.103 & 0.000 & 0.031 & 0.044 & 0.038 & 0.003 \\ \hline
%\multicolumn{1}{|c|}{\textbf{WFG3}} & 0.023 & 0.023 & 0.023 & 0.000 & 0.031 & 0.061 & 0.039 & 0.005 & 0.022 & 0.023 & 0.022 & 0.000 & 0.033 & 0.033 & 0.033 & 0.000 \\ \hline
%\multicolumn{1}{|c|}{\textbf{WFG4}} & 0.127 & 0.127 & 0.127 & 0.000 & 0.121 & 0.144 & 0.132 & 0.005 & 0.095 & 0.098 & 0.097 & 0.001 & 0.091 & 0.094 & 0.093 & 0.001 \\ \hline
%\multicolumn{1}{|c|}{\textbf{WFG5}} & 0.177 & 0.184 & 0.181 & 0.002 & 0.160 & 0.186 & 0.170 & 0.005 & 0.147 & 0.158 & 0.153 & 0.003 & 0.143 & 0.155 & 0.147 & 0.002 \\ \hline
%\multicolumn{1}{|c|}{\textbf{WFG6}} & 0.155 & 0.205 & 0.175 & 0.012 & 0.159 & 0.196 & 0.177 & 0.009 & 0.122 & 0.151 & 0.140 & 0.007 & 0.143 & 0.173 & 0.163 & 0.008 \\ \hline
%\multicolumn{1}{|c|}{\textbf{WFG7}} & 0.127 & 0.127 & 0.127 & 0.000 & 0.113 & 0.138 & 0.123 & 0.007 & 0.094 & 0.102 & 0.097 & 0.001 & 0.092 & 0.094 & 0.093 & 0.001 \\ \hline
%\multicolumn{1}{|c|}{\textbf{WFG8}} & 0.189 & 0.194 & 0.192 & 0.001 & 0.244 & 0.274 & 0.256 & 0.008 & 0.161 & 0.166 & 0.163 & 0.001 & 0.101 & 0.121 & 0.106 & 0.005 \\ \hline
%\multicolumn{1}{|c|}{\textbf{WFG9}} & 0.130 & 0.240 & 0.154 & 0.036 & 0.138 & 0.246 & 0.224 & 0.025 & 0.099 & 0.211 & 0.119 & 0.037 & 0.101 & 0.162 & 0.106 & 0.010 \\ \hline
%\multicolumn{1}{|c|}{\textbf{DTLZ1}} & 0.014 & 0.014 & 0.014 & 0.000 & 0.017 & 0.020 & 0.018 & 0.001 & 0.013 & 0.014 & 0.014 & 0.000 & 0.014 & 0.014 & 0.014 & 0.000 \\ \hline
%\multicolumn{1}{|c|}{\textbf{DTLZ2}} & 0.027 & 0.027 & 0.027 & 0.000 & 0.030 & 0.036 & 0.032 & 0.001 & 0.023 & 0.024 & 0.023 & 0.000 & 0.024 & 0.025 & 0.024 & 0.000 \\ \hline
%\multicolumn{1}{|c|}{\textbf{DTLZ3}} & 0.027 & 0.027 & 0.027 & 0.000 & 0.027 & 0.032 & 0.030 & 0.001 & 0.023 & 0.023 & 0.023 & 0.000 & 0.024 & 0.025 & 0.024 & 0.000 \\ \hline
%\multicolumn{1}{|c|}{\textbf{DTLZ4}} & 0.027 & 0.595 & 0.092 & 0.181 & 0.028 & 0.036 & 0.032 & 0.001 & 0.023 & 0.595 & 0.190 & 0.225 & 0.024 & 0.025 & 0.024 & 0.000 \\ \hline
%\multicolumn{1}{|c|}{\textbf{DTLZ5}} & 0.003 & 0.003 & 0.003 & 0.000 & 0.003 & 0.003 & 0.003 & 0.000 & 0.002 & 0.002 & 0.002 & 0.000 & 0.002 & 0.002 & 0.002 & 0.000 \\ \hline
%\multicolumn{1}{|c|}{\textbf{DTLZ6}} & 0.022 & 0.163 & 0.087 & 0.032 & 0.126 & 0.224 & 0.187 & 0.027 & 0.003 & 0.136 & 0.069 & 0.033 & 0.002 & 0.002 & 0.002 & 0.000 \\ \hline
%\multicolumn{1}{|c|}{\textbf{DTLZ7}} & 0.045 & 0.045 & 0.045 & 0.000 & 0.038 & 0.052 & 0.044 & 0.003 & 0.060 & 0.087 & 0.079 & 0.008 & 0.027 & 0.029 & 0.028 & 0.000 \\ \hline
%\multicolumn{1}{|c|}{\textbf{UF8}} & 0.048 & 0.365 & 0.069 & 0.051 & 0.093 & 0.220 & 0.178 & 0.031 & 0.027 & 0.159 & 0.033 & 0.022 & 0.026 & 0.034 & 0.029 & 0.002 \\ \hline
%\multicolumn{1}{|c|}{\textbf{UF9}} & 0.041 & 0.151 & 0.086 & 0.049 & 0.106 & 0.314 & 0.139 & 0.049 & 0.025 & 0.137 & 0.094 & 0.053 & 0.022 & 0.030 & 0.025 & 0.002 \\ \hline
%\multicolumn{1}{|c|}{\textbf{UF10}} & 0.163 & 0.565 & 0.294 & 0.125 & 0.198 & 0.658 & 0.261 & 0.080 & 0.159 & 0.553 & 0.257 & 0.131 & 0.061 & 0.168 & 0.099 & 0.026 \\ \hline
%\multicolumn{1}{|c|}{\textbf{Mean}} & \textbf{0.078} & \textbf{0.170} & \textbf{0.099} & \textbf{0.026} & \textbf{0.097} & \textbf{0.166} & \textbf{0.121} & \textbf{0.015} & \textbf{0.066} & \textbf{0.150} & \textbf{0.093} & \textbf{0.028} & \textbf{0.053} & \textbf{0.068} & \textbf{0.058} & \textbf{0.003} \\ \hline
%\end{tabular}%
%%}
%\end{table*}

