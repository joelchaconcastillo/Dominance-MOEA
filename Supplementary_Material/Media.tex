This section is devoted to provide a description of the video included as supplementary material\footnote{Alternatively, the video can be accessed 
in \url{https://youtu.be/dbk5DaFJ8y0}}.
%
The main aim of this video is to show the large differences between the way that \VSDMOEA{} and state of the art \MOEAS{} 
explore the search space.
%
The video shows a single execution of \VSDMOEA{} and \RMOEA{} for the optimization of the WFG5 problem.
%
Note that \NSGAII{} and \MOEAD{} were also executed and their behaviours were similar to the one of \RMOEA{}.
%
Thus, in order to avoid saturating the video, they were not included.
%
In order to allow the visuallization, the WFG5 problem was configured with two variables, meaning that one distance parameter
and one position parameter are used.
%
The main peculiarity of WFG5 is its deceptiveness~\cite{Joel:WFG}.
%
In this configuration, a solution $x$ is part of the Pareto set when $x_2 = 1.4$.
%
However, most of the descent directions points out towards two local optimal fronts, which appear when $x_2 = 0$ or when $x_2 = 4$.
%
The stopping criterion was set to $100,000$ function evaluations and the reamining parameters were set as in the main document.
%
It is worth noticing that \VSDMOEA{} explicitly promotes decision variable space diversity until the halfway point of the execution.

The video is split in two sides.
%
The left-side represents the objective space and the right-side represents the decision variable space.
%
In the decision variable space each local optimal region is remarked with a horizontal blue line, and the global optimal region is remarked with a horizontal red line.
%
The video shows that after just a few generations, \RMOEA{} has converged prematurely to the deceptive regions, 
contrarily to the \VSDMOEA{} which is still exploring.
%
Approximately at the $30\%$ of total generations, \VSDMOEA{} has located three individuals in the global optimal region, and few generations later ($40\%$)
the amount of solutions in such a region has increased significantly.
%
Thereafter, at the $50\%$ of total generations, \VSDMOEA{} has located most of the individuals in the global optimal region and just 
some individuals are located in the local optimal regions.
%
Thus, the results attained by \RMOEA{} are clearly outperformed by \VSDMOEA{}.
%
Note that both schemes place a lot of points in the knee of the Pareto front, at the cost of leaving some holes near the boundaries.
%
This is because both the R2 metric and the objective space density estimator used in \VSDMOEA{} prefer such regions.
%
In order to point out the superiority of \VSDMOEA{} note that most of the solutions attained by \RMOEA{} are dominated by the ones 
located by \VSDMOEA{}.
%
The main particularity of \VSDMOEA{}, which is to delay convergence, is clearly shown in this video.
