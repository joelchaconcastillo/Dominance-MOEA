This section provides a description of the video included as supplementary material\footnote{Alternatively, the video can be accessed 
at \url{https://youtu.be/dbk5DaFJ8y0}}.
%
The main aim of this video is to show the large differences between the way in which \VSDMOEA{} and state-of-the-art \MOEAS{} 
explore the search space.
%
The video shows a single execution of \VSDMOEA{} and \RMOEA{} to solve WFG5.
%
Note that \NSGAII{} and \MOEAD{} were also executed and their behaviors were similar to that of \RMOEA{}.
%
Thus, in order to avoid saturating the video, they were not included.
%
In order to allow a proper visualization, WFG5 was configured with two decision variables, meaning that one distance parameter
and one position parameter are used.
%
The main peculiarity of WFG5 is its deceptiveness~\cite{Joel:WFG}.
%
In this configuration, a solution $x$ is part of the Pareto set when $x_2 = 1.4$.
%
However, most of the descent directions point towards two local optimal fronts, which appear when $x_2 = 0$ or when $x_2 = 4$.
%
The stopping criterion was set to $100,000$ function evaluations and the remaining parameters were set as in the main document.
%
It is worth noting that \VSDMOEA{} explicitly promotes decision variable space diversity until the halfway point of the execution.

The video is divided into two sides.
%
The left-side represents the objective space and the right-side represents the decision variable space.
%
In the decision variable space, each local optimal region is highlighted with a horizontal blue line, and the global optimal region is highlighted 
with a horizontal red line.
%
The video shows that after just a few function evaluations, \RMOEA{} has converged prematurely to the deceptive regions, 
contrarily to the \VSDMOEA{}, which is still exploring.
%
At approximately $30\%$ of the total number of function evaluations, \VSDMOEA{} has 
located three individuals in the global optimal region, and a few later ($40\%$ of function evaluations),
the number of solutions in this region has significantly increased.
%
Thereafter, at the $50\%$ of all function evaluations, \VSDMOEA{} has located most of the individuals in the global optimal region and only 
some individuals are located in the local optimal regions.
%
Thus, the results obtained by \RMOEA{} are clearly outperformed by \VSDMOEA{}.
%
Note that both schemes place a large number of points in the knee of the Pareto front, at the cost of leaving some holes near the boundaries.
%
This is because both the $R2$ metric and the objective space density estimator used in \VSDMOEA{} prefer these regions.
%
To underscore the superiority of \VSDMOEA{}, note that most of the solutions obtained by \RMOEA{} are dominated by those 
located by \VSDMOEA{}.
%
The main feature of \VSDMOEA{}, which is to delay convergence, is clearly shown in this video.

