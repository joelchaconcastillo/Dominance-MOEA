This section is devoted to reviewing some of the most relevant work which is closely related to our proposal.
%
First, some of the most popular ways of managing diversity in \EAS{} are presented.
%
Then, the state of the art in \MOEAS{} is summarized.

\subsection{Diversity Management in Evolutionary Algorithms}

The proper balance between exploration and exploitation is one of the keys to designing successful \EAS{}.
%
In the single-objective domain, it is known that properly managing the diversity of decision 
variable space is a way to achieve this balance,
and as a consequence, a large number of diversity management techniques have been devised~\cite{Mohan:14}.
%
Specifically, these methods are classified depending on the component(s) of the \EA{} that is modified to alter the way in which 
diversity is maintained.
%
A popular taxonomy identifies the following groups~\cite{Joel:Crepinsek}: \textit{selection-based}, \textit{population-based}, 
\textit{crossover/mutation-based}, \textit{fitness-based}, and \textit{replacement-based}.
%
Additionally, the methods are referred to as \textit{uniprocess-driven} when a single component is altered, whereas the term
\textit{multiprocess-driven} is used to refer to those methods that act on more than one component.

Among the previous proposals, the replacement-based methods have yielded very high-quality results in recent years~\cite{Segura:17}, so
this alternative was selected with the aim of designing a novel \MOEA{} that explicitly incorporates a way to control the diversity 
of decision variable space.
%
The basic principle of these methods is to bias the level of exploration in successive generations by 
controlling the diversity of the survivors~\cite{Segura:17}.
%
Since premature convergence is one of the most common drawbacks in the application of \EAS{}, 
modifications are usually performed with the aim of slowing down convergence.
%
One of the most popular proposals belonging to this group is the \textit{crowding} method, which
is based on the principle that offspring should replace similar individuals to those from the previous generation~\cite{Mengshoel:14}.
%
Several replacement strategies that do not rely on crowding have also been devised.
%
In some methods, diversity is considered as an objective.
%
For instance, in the hybrid genetic search with adaptive diversity control (\HGSADC{})~\cite{Vidal:13}, individuals are sorted 
by their contribution to diversity and by their original cost.
%
Then, the rankings of the individuals are used in the fitness assignment phase.
%
A more recent proposal~\cite{Segura:17} incorporates a penalty approach to gradually alter the amount of diversity 
maintained in the population.
%
Specifically, the initial phases preserve a higher amount of diversity than the final phases of the optimization.
%
This last method has inspired the design of the novel proposal put forth in this paper for multi-objective optimization.
%

It is important to remark that in the case of multi-objective optimization, little work related to maintaining the 
diversity of decision variable space has been done.
%
The following section reviews some of the most important \MOEAS{} and introduces some of the works that consider
the maintenance of diversity of decision variable space.

\subsection{Multi-objective Evolutionary Algorithms}

In recent decades, several \MOEAS{} have been proposed. 
%
While most of them seek to provide a well-spread set of solutions close to the Pareto front,
several ways of achieving this purpose have been devised.
%
Therefore, several taxonomies have been proposed with the aim of better classifying the different 
schemes~\cite{Joel:BOOK_MOEAs}.
%
Particularly, a \MOEA{} can be designed based on Pareto dominance, indicators and/or decomposition~\cite{Joel:StateArt}.
%
Since none of the groups is significantly superior to the others, in this work all of them are taken into account to validate
our proposal.
%
This section introduces the three types of schemes and some of the most popular approaches belonging to each category.
%
Then, one \MOEA{} in each category is selected to validate the \VSDMOEA{}.

The dominance-based category includes those schemes where the Pareto dominance relationship is used to guide the 
design of some of its components, such as the fitness assignment, parent selection and replacement phase.
%
The dominance relationship does not inherently promote the preservation of diversity in objective function space; 
therefore, additional techniques such as objective space density estimators are usually integrated in order to obtain 
a proper spread and convergence to the Pareto front.
%
The most popular dominance-based \MOEA{} is the Non-Dominated Sorting Genetic Algorithm~II (\NSGAII{})~\cite{Joel:NSGAII}.
%

Several quality indicators have been devised to assess the performance of \MOEAS{}.
%
In indicator-based \MOEAS{}, the use of the Pareto dominance relationship is substituted by some quality indicators 
to guide the decisions made by the \MOEA{}.
%
An advantage of this kind of algorithm is that the indicators usually take into account both the quality and 
the diversity of objective function space. Thus, incorporating additional mechanisms to promote diversity in 
objective function space is not required.
%
The Indicator-Based Evolutionary Algorithm (\IBEA{})~\cite{Joel:IBEA} was the first method belonging to this category.
%
A more recent one is the R2-Indicator-Based Evolutionary Multi-objective Algorithm (\RMOEA{})~\cite{trautmann2013r2}, 
whose performance in \MOPS{} has been quite promising.
%
Its most important feature is the use of the R2 indicator.

Finally, decomposition-based \MOEAS{}~\cite{Joel:MOEAD_AMS} are based on transforming the \MOP{} into a set of 
single-objective optimization problems that are tackled simultaneously.
%
This transformation can be performed in several ways, e.g. with a linear weighted sum or with a weighted Tchebycheff function. 
%
Given a set of weights to establish different single-objective functions, the \MOEA{} searches for a single 
high-quality solution for each of them. 
%
The weight vectors should be selected with the aim of obtaining a well-spread set of solutions~\cite{Joel:Kalyanmoy}.
%
The Multi-objective Evolutionary Algorithm Based on Decomposition (\MOEAD{})~\cite{Joel:MOEAD} is the most popular 
decomposition-based \MOEA{}. 
%
Its main principles include problem decomposition, weighted aggregation of objectives and mating restrictions 
through the use of neighborhoods. 

It is important to note that none of the most popular algorithms in the multi-objective field introduces special 
mechanisms to promote diversity of decision variable space.
%
However, some efforts have been devoted to this principle.
%
A popular approach to promote the diversity of decision variable space is the application of fitness sharing~\cite{Joel:NPGA}, 
in a way similar to single-objective optimization.
%
Although fitness sharing might be used to promote the diversity of both objective and decision variable spaces, most
popular variants consider only distances in objective function space.
% 
Another \MOEA{} designed to promote diversity of both decision variable space and objective function space is the Genetic
Diversity Evolutionary Algorithm (GDEA)~\cite{toffolo2003genetic}.
%
In this case, each individual is assigned a diversity-based objective which is calculated as the
Euclidean distance in the genotype space to the remaining individuals in the population.
%
Then, a ranking that considers both the original objectives and the diversity objective is used
to sort individuals.
%
Another somewhat popular approach is to calculate distances between candidate solutions by taking
into account both objective and decision variable space~\cite{deb2005omni,shir2009enhancing} with the aim
of promoting diversity of both spaces.
%
A different proposal combines the use of two selection operators~\cite{chan2005evolutionary}.
%
The first one promotes diversity and quality in objective function space, whereas the second one promotes diversity 
in decision variable space.
%
A different approach involves modifying the hypervolume to integrate the decision variable space diversity 
into a single metric~\cite{ulrich2010integrating}.
% 
In this approach, the proposed metric is used to guide the selection in the \MOEA{}.
%
Finally, some indirect mechanisms that might affect diversity have also been taken into account.
%
The most popular one is probably the use of mating restrictions~\cite{Joel:STUDY_MATTING_RESTRICTION,Joel:MOEAD_AMS}.

In light of the results of the approaches described above, it is clear that considering the diversity of decision variable space
in the design phase might yield benefits for decision makers, since the final solutions obtained by these 
methods exhibit a higher decision variable space diversity than those obtained by traditional 
approaches~\cite{deb2005omni, rudolph2007capabilities}.
%
Thus, while clear improvements are obtained when metrics related to the decision 
variable space are taken into account, the benefits in terms of the 
objective function space are not so clear.
%
We claim that one of the reasons for this behavior might be that the diversity of decision variable space is considered 
in the whole optimization process.
%
However, in a similar way as in the single objective domain, reducing the importance granted to the diversity 
of decision variable space as the generations progress~\cite{Joel:MULTI_DYNAMIC} might be truly important for obtaining
better approximations of the Pareto front.
%
Currently, no \MOEA{} considers this idea, so it is this principle that has guided the design of our novel \MOEA{}.


