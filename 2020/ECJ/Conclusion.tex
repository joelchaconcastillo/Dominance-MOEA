\EAS{} have been one of the most popular approaches for dealing with complex optimization problems.
%
Their design is a highly complex task that requires defining several components.
%
Looking at the differences between single-objective and multi-objective optimizers, it is worth noting
that several state-of-the-art single-objective optimizers explicitly consider the diversity of the variable space, particularly
when dealing with long-term executions, whereas this is not the case for \MOEAS{}.

This paper proposes a novel \MOEA{}, called \VSDMOEA{}, that takes into account the diversity of both decision variable space
and objective function space.
%
The main novelty is that the importance given to the different diversities is adapted during the optimization process.
%
In particular, in \VSDMOEA{} more importance is given to the diversity of decision variable space at the initial stages,
but as evolution progresses it gradually grants more importance to the diversity of 
objective function space.
%
This is performed using a penalty method that is integrated into the replacement phase.
%
Also included is a novel density estimator based on IGD+ that is used to select from the non-penalized individuals.

The experimental validation carried out shows a remarkable improvement in \VSDMOEA{} when compared 
to state-of-the-art \MOEAS{} both in two-objective and three-objective problems.
%
Moreover, our proposal not only improves the state-of-the-art algorithms in long-term and medium-term executions,
but it also offers a competitive performance in short-term executions.
%
The scalability analyses show that as the number of objectives and decision variables increases, 
the implicit variable space maintained by state-of-the-art
\MOEAS{} also increases.
%
Thus, for large number of objectives and decision variables, explicitly considering the diversity of decision 
variable space is less helpful.
%
Finally, the analysis of the initial threshold distance, which is an additional parameter required by \VSDMOEA{}, 
shows that finding a proper value for this parameter is not a difficult task.

In the future, we plan to apply the principles studied in this paper to other categories of \MOEAS{}.
%
For instance, including the diversity management put forth in this paper in decomposition-based and indicator-based \MOEAS{} seems plausible.
%
Additionally, we would like to develop an adaptive scheme to avoid setting the initial threshold value.
%
Finally, in order to obtain even better results, these strategies are going to be incorporated into a multi-objective memetic algorithm.
%
